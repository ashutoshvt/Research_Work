%\documentclass[journal=jpccck,manuscript=article]{achemso}
%\usepackage[numbers,super]{natbib}
%
%\usepackage{graphicx}
%\usepackage{xcolor}
%\usepackage{todonotes}
%
%\def\ket#1{| #1 \rangle}
%\def\bra#1{\langle #1 |}
%
%\usepackage[utf8]{inputenc} % set input encoding (not needed with XeLaTeX)
%\usepackage{verbatim}
%\usepackage{amsfonts}
%\usepackage{graphicx}
%\newcommand{\overbar}[1]{\mkern 2.2mu\overline{\mkern-2.2mu#1\mkern-2.2mu}\mkern 2.2mu}
%\usepackage{multirow}
%\usepackage{array}
%\usepackage{varwidth}
%\usepackage{bm}
%
%\title{Pair Natural Orbitals ++ approach for Coupled Cluster Linear-Response Theory}
%\author{Ashutosh Kumar}
%\affiliation{Department of Chemistry, Virginia Tech, Blacksburg, Virginia 24061, U.S.A.}
%\author{T.\ Daniel Crawford}
%\email{crawdad@vt.edu}
%\affiliation{Department of Chemistry, Virginia Tech, Blacksburg, Virginia 24061, U.S.A.}
%
%\date{\today}
%
%\begin{document}
%
%\begin{abstract} We present here the the modified  pair natural-orbital (NO) 
%approach, which we call PNO++ for a compact representation of the 
%response of the coupled cluster wavefunction to external electric
%and magnectic fields. In our earlier work,%\cite{McAlexander15}
%we showed that the regular PNO method performs rather poorly for 
%%The frozen-virtual natural-orbital (NO) approach, whereby the
%%unoccupied-orbital space is constructed using a correlated density such as
%%that from many-body perturbation theory, has proven to yield compact wave
%%functions for determining ground-state correlation energies and associated
%%properties, with corresponding occupation numbers providing a guide to the
%%truncation of the virtual space.  In this work this approach is tested for the
%%first time for the calculation of higher-order response properties,
%%particularly frequency-dependent dipole polarizabilities using coupled-cluster
%%theory.  We find that such properties are much more sensitive to the
%%truncation of virtual space in the natural orbital (NO) basis than in the
%%original canonical molecular orbital (CMO) basis, with truncation errors
%%increasing linearly with respect to the number of frozen virtual NOs.  The
%%reasons behind this poor performance include the more diffuse nature of NOs
%%with low occupation numbers as well as the reduction in sparsity of the
%%perturbed singles amplitudes in the NO basis and the neglect of orbital
%%response.  We tested a number of approaches
%%to improve the performance of the NO space, including the use of a
%%field-perturbed density to define the virtual orbitals and various
%%external-space corrections.  The truncation of the CMO space, on the other
%%hand, yields errors in coupled-cluster dipole polarizabilities of less than
%%2\% even after removing as much as 50\% of the full virtual space. We find
%%that this positive performance of the CMO space results from a cancellation of
%%errors due to the truncation of the unperturbed and perturbed amplitudes, as
%%well as sparsity of the singles amplitudes.  We introduce a simple criterion
%%called a dipole amplitude to use as a threshold for truncating the CMO basis
%%for such property calculations.  \end{abstract}
%\end{abstract}
%\maketitle

\section{Introduction}
Accurate {\em ab initio} models like the coupled cluster theory have been used quite reliably to 
predict the chiroptical properties of molecules.\cite{} However, they have been limited to very 
small system sizes due to the heavy computational expenses associated with such methods. 
For example, the coupled cluster singles and doubles method (CCSD) has a high polynomial scaling of 
$O(N^6)$, where N is some measure of the system size. This is clearly unphysical as the phenomenon 
of dynamic electron-correlation, which these methods aim to capture, is local in nature.\cite{} This 
steep scaling can be attributed to the use of delocalized canonical Hartree-Fock MOs (CMOs) as the one 
electron basis for representing the wavefunction. Local correlation techniques try to exploit the intrinsic 
sparsity in correlated wavefunctions by unitarily localizing the occupied orbitals (LMOs) using methods 
like Boys-Foster, Pipek-Mezey etc\cite{}. and constructing excitation domains corresponding to each LMO. 
The sizes of these domains are usually very small compared to the full virtual space and become constant 
in the asymptotic limit, making these approaches reduced-scaling. Saebo and Pulay's introduction of projected 
atomic orbitals (PAO) as a representation of the virtual space stands as one of the earliest works in this 
regard\cite{}. As the name suggests, PAOs are formed by projecting out the occupied MO components from the 
AO basis. Since the PAOs are centered on atoms (just like AOs), an excitation domain for a given LMO can 
be constructed by including only those PAOs which are on or near the atoms associated with that LMO. 
Subsequently, domains corresponding to a pair of LMOs can be formed by taking a union of the PAO space of both 
the LMOs and so on. Werner, Sch{\"u}tz and co-workers \cite{} were the first to apply these concepts within 
the framework of CC theory. By employing truncated PAO domains coupled with other approximations like weak 
and distant pairs, they were able to achieve linear scaling while maintaining accurate description of 
ground-state properties like reaction enthalpies, thermodynamic constants etc.\cite{}  Crawford and King 
were the first to use PAOs with the equation of motion CCSD method to calculate excitation energies\cite{}. 
In a similar work, Korona and Werner\cite{} by constructing state specific PAO domains minimized the average 
localization errors for their test set to only 0.06 ev. Schutz and co-workers extended this approach with 
density-fitted second order coupled cluster (CC2) \cite{} method to calculate excited state properties of 
large systems.\cite{} However, these approaches could require very large domian sizes depending on the 
character of the excited state, for example, excitations in a charge-transfer excited state could be 
fairly non-local. PAOs have also been used within the context of CC linear response theory to calculate 
higher-order response properties like (hyper)polarizabilities, chiroptical response etc. albeit on a 
much smaller scale.\cite{} In 2004, Korona and Werner\cite{} used PAOs with local coupled cluster singles 
and doubles method (LCCSD) to calculate electric dipole moments and static polarizabilities where they got 
average errors of 1.61\% and 0.48\% respectively. Further minimization of the errors required building bigger 
domains leading to a significant increase in computational expenses. In the same year, Russ and Crawford\cite{} 
used a modified domain building procedure within the PAO framework, where they augmented the ground state 
orbital domains on the basis of first-order orbital response coefficients obtained by sloving the coupled-perturbed 
Hartree-Fock equations, to calculate CC static polarizabilities. A few years later, they extended this 
formalism to calculate dynamic polarizabilities and optical rotations\cite{}. While the localization errors 
for linear molecular structures were shown to be only a few percent of canonical results, three dimensional 
compounds required significantly large domains, specially for optical rotations. A similar conclusion was drawn 
when Friedrich and co-workers applied the incremental scheme, a fragmentation based local correlation technique 
to calculate CC dynamic polarizabilities.\cite{} For a more detailed overview of the applications of PAO based 
methods in this field, please refer to ref\cite{McAlexander15}. Within the framework of local correlation, 
the pair natural orbitals (PNOs) introduced in 1970s by Edimnston and Krauss\cite{}, Meyer\cite{}, Ahlrichs\cite{}, 
Kutzelnigg\cite{}, Staemmler and co-workers\cite{} and orbital specific virtuals (OSVs) developed by Chan and 
co-workers in 2011 \cite{}, are other popular alternatives to PAOs for a compact representation of the virtual 
space. In the local PNO (LPNO) approach, each pair of LMOs have their own virtual space which can be obtained 
by the diagonalization of the virtual-virtual block of an approximate 1-electron pair specific density. 
The OSV approach on the other hand involves diagonalization of a separate density for each LMO. Even though 
the PNOs were shown be to be quite useful as a wavefunction compression technique\cite{}, they were initially 
abandoned due to the significant computational costs involved in the transformation of the two electron 
integrals, onlyto be revived later by Neese and co-workers in 2008 by making use of the density-fitting procedure
\cite{}. They also proposed a variant of the LPNO approach, a scheme which they call domain based LPNO (DLPNO), 
where the PNOs are expressed in terms of PAOs, to achieve near-linear scaling behavior in MP2 and CC calculations
\cite{}. Following their pioneering works, the LPNO and the DLPNO methods has been used quite sucessfully to study 
ground state properties of molecular systems previously unreachable by canonical correlated methods\cite{}. 
\cite{} Hattig and co-workers were the first to extend this scheme to excited states by using state-specific 
PNOs generated from CIS(D) densities at CC2 level of theory\cite{} and recently with CCSD.\cite{}. Valeev and 
co-workers recently implemented a state-averaged L-PNO-CCSD simulation program by taking an average of the 
CIS(D) densities over a desired set of excited states\cite{}. Similar works employing state-specific natural 
transition orbitals and natural orbitals for calculating excitation energies have also been reported.
\cite{} Our group implemented the LPNO scheme in conjunction with CC2 and CCSD linear response (LR) theory  
to calculate dynamic polarizabilities and specific rotations of different chiral molecules ranging from linear
(H2)n helices to cagelike 1r-4r-norbornenone, where consistent with earlier studies, PNOs were found
to be much more compact than the PAOs and OSVs for representing the wavefunction. However, the truncation errors 
in specific rotations for 3-dimensional structures were quite large and the convergence of both the properties 
towards the canonical result turned out to be very slow\cite{}. The failure of the regular LPNO-CC2/CCSD-LR approach
for higher-order response properties could be attributed to the inability of the ground-state MP2 density to 
capture the response of the wavefunction, as shown in our earlier work with natural orbitals (NOs)\cite{}. However,
we have recently demonstrated that the NOs obtained from ``perturbation aware" densities are better suited
for calculating these response properties.\cite{} In this paper, we extend this approach which we have named 
PNO++ by constructing second-order pair specific perturbed densities, and compare its performance with 
the regular method for the same test systems as our earlier PNO paper.
% how about response properties
% only McAlexander and co-workers
% showed that PNO much more compact than PAO and OSV approach 
% However, rotations for 2-d and 3-d systems very sensitive to truncation
% inspired by our earlier work with natural orbitals,
% want to improve that usiing field perturbed densities 
% Here we explotre just that!!!! - exploration, that's why simulation code!
\section{Theory}
\subsection{Coupled Cluster Response Theory}
Coupled cluster (CC) response theory has been proven to be quite a reliable and accurate theoretical
tool to calculate higher-order response properties\cite{}. Within this formalism,
one expands the time-dependent expectation value of a time-independent operator 
in different orders of the perturbation. The response functions can then be identified 
as fourier analogues (FA) of the time-dependent expansion terms up to a given order, ex. the
linear response function (LRF) is the FA of terms appearing in the first order in the
expansion. The LRFs can also be seen as frequency-dependent first-order perturbed densities. 
The expression of CC-LRF in terms of these densities can be written as\cite{},
\begin{equation}
{\langle\langle A;B\rangle\rangle}_{\omega_1} =  \hat{P}(A,B)[\langle 0 | \
[\hat{Y}^{B}_{\omega_1}, \bar{A}]|0\rangle + \langle 0 | \
(1 + \hat{\Lambda})|[\bar{A},\hat{X}^{B}_{\omega_1}]|0\rangle]
\end{equation}
where $A$ and $B$ are time-independent one electron perturbation opeartors,
$\hat{P}(A,B)$ is a symmetrizing operator which simultaneously interchanges 
operators $A$ and $B$ and takes the complex conjugate of the expression, $\omega_1$ 
is the frequency of the external field, $|0\rangle$ is the reference wavefunction, $\hat{\Lambda}$ is
a linear de-excitation operator that parametrizes the CC left hand wavefunction, 
overbar on operator $A$ denotes a similarity transformation with the ground state 
$T$ operator, $\bar{A} = e^{-\hat{T}}\hat{A}e^{\hat{T}}$
and $\hat{X}^{B}_{\omega_1}$ and $\hat{Y}^{B}_{\omega_1}$ are the first-order right and left 
hand perturbed amplitudes corresponding to perturbation operator $B$ respectively.
These perturbed amplitudes can be obtained by solving a set of linear equations involving
CC Jacobian,


\subsection{Pair Natural Orbitals}
The regular PNO density can be written as,
\begin{equation}
\bm{D}^{ij} = \frac{2}{1+\delta_{ij}} (\bm{T}^{ij}\tilde{\bm{T}}^{ij\dagger} + \bm{T}^{ij\dagger}\tilde{\bm{T}}^{ij})
\end{equation}
where, 
\begin{equation}
\tilde{\bm{T}}^{ij} = 2\bm{T}^{ij} - \bm{T}^{ij\dagger}
\end{equation} 
\begin{equation}
T^{ij}_{ab} = \frac{\langle ab|ij \rangle}{f_{ii} + f_{jj} - \epsilon_a - \epsilon_b}
\end{equation} 
In the PNO++ approach, we create a perturbation specific density for each $ij$ pair. For a given
perturbation A, the PNO++ density is constructed by replacing the ground state $T^{ij}_{ab}$ amplitudes 
by perturbed amplitudes $T(A)^{ij}_{ab}$,
\begin{equation}
\bm{D}(A)^{ij} = \frac{2}{1+\delta_{ij}} (\bm{T}(A)^{ij}\bm{\tilde{T}}(A)^{ij\dagger} + \bm{T}(A)^{ij\dagger}\bm{\tilde{T}}(A)^{ij})
\end{equation}
The leading order contribution to these perturbed amplitudes come from $\bar{A}$ which is nothing but 
the similarity transformed perturbation operator $A$, $e^{-T}\hat{A}e^{T}$. So, we choose the following
form of the $T(A)^{ij}_{ab}$ amplitudes,
\begin{equation}
T(A)^{ij}_{ab} =  \frac{\bar{A}^{ij}_{ab}}{f_{ii} + f_{jj} - \epsilon_a - \epsilon_b} 
\end{equation} 
where,
\begin{equation}
\bar{A}^{ij}_{ab} = P_{ij}^{ab}\bigg[\sum_e t^{ij}_{eb}[A^a_e - t^m_a A^m_e] -\sum_m t^{mj}_{ab}[A^m_i - t^i_e A^m_e]\bigg]
\end{equation} 
\begin{equation}
P_{ij}^{ab} f_{ij}^{ab} = f_{ij}^{ab} + f_{ji}^{ba} .
\end{equation}
\section{Computational Details}
Here we are exploring !!!! that's why simulation code!
looking at the right behviour pattern mainly
look at harley's paper for more details.
\section{Results and Discussions}
whatever I have so far.
\section{Conclusions}

%The most popular approaches
%of inducing sparsity in the virtual orbital space 
%are based on the concept of Natural
%orbitals (NOs), introduced by L{\"o}wdin in 1955\cite{}as the eigenvectors of  
%the one electron reduced density matrices (1-RDMs), where he referred to 
%the corresponding eigenvalues as occupation numbers (ONs). L{\"o}wdin showed  
%the configuration interaction (CI) wavefunction expansion in the NO basis 
%converges quite rapidly compared to the CMO basis. Motivated by L{\"o}wdin's 
%findings, Bender, Barr and Davidson\cite{} used NOs obtained from 
%a few years later used approximate NOs 
%to construct optimized configurations in their CI calculations on first-row diatomic 
%hydrides. A Barr 
%and Davidson\cite{} employed what they called the frozen form of NOs 
%obtained by diagonalizing only the virtual-virtual block of the density 
%matrix to analyze the nature of the CI method through correlation energy 
%studies on Ne atom; Edminston and Krauss,\cite{}Meyer\cite{}, Ahlrichs and 
%co-workers\cite{} developed the concept of pair natural orbitals (PNOs) or 
%``pseudonatural orbitals" as the eigenvectors of the 1-RDM associated 
%with a single electron pair. 
%The ONs can be seen as a metric for estimating the importance of a virtual NO in describing
%electron correlation effects and hence all those virtual NOs which have ONs below a given threshold
%can be removed without causing any significant errors in correlation energies. 

%f. Recent work of FVNO and then PNO for excitation energies
%g. How about response properties?
%h. Properties are tough, cite all our papers, harley's paper, 
%Extension of harley's work, can the results be improved by making use of perturbed densities
%, cite my FVNO++ paper and explain the logic behind that
%  but that's only a reduced-prefactor method! how to make it reduced-scaling?
%i. Here we compare the performance of this approach for dynamic polarizabilities and specific 
