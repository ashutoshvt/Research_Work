%\section{Introduction}

%    a. chirality - same story as my Internal Seminar.
%    b. Drug thalidomide example
%    c. Chiroptical spectroscopy
%    d. Theory as a tool!
%    e. Some history of OR calculations (Harley's thesis)
%    f. TD-DFT not enough in most cases, CC needed (known problems of DFT)
%    g. Challenges of CC - (mention about MD as well.)
%    h. My work comes in here!
%	i. local correlatin - which chapter discusses what... end by conclusion..
\section{Chirality}
Lord Kelvin was the first to use the word `chiral' at a lecture he gave at Oxford 
University in 1894: ``I call any geometrical figure, or group of points, 
`chiral', and say that it has chirality if its image in a plane mirror, ideally realized, 
cannot be brought to coincide with itself"\cite{}. There are different ways in which a molecule 
could be chiral. The most common example being the presence of a chiral center 
like a carbon or nitrogen atom with four different substituents. Axial chirality, usually 
possessed by molecules with cumulated double bonds like allenes and chirality planes 
found in compounds like paracyclophanes are other popular examples of chiral elements 
found in nature. \\
A chiral molecule and its mirror image are known as enantiomers. Although the
enantiomers have idential phyical properties like boiling and melting points their
interactions in a chiral environment can be quite different. For example,
while the (R) enantiomer of the compound limonene smells like lemons and oranges,
as the name sugggests, the (S) enantiomer on the other hand smells like turpentine. 
Thus, the interaction of the (S) enantiomer with the chiral molecules constituting the olfactory receptors 
responsible for the sensation of smell is totally different from its fellow enantiomer.
The enantiomers also interact differently with the left- and right-hand circularly polarized (LCP and
RCP) light in processes of absorption, refraction etc. A given chiral molecule posesses the 
properties of circular dichroism (CD) and circular birefringence, i.e. it has different absorption 
coefficents and refractive indices for the LCP and RCP light respectively. As a result of circular 
birefringence, a chiral molecule rotates the plane of polarized light as it passes through it,
a phenomenon known as optical rotation. It should be noted that an enantiomeric pair of chiral
molecules has equal optical rotations but in opposite directions.\\
Since almost every part of the human body is composed of chiral molecules, its unsurprising that
more than 60\% of the pharmaceutical drugs are chiral as well. Hence, one of the most 
important aspect of the drug-design research is to be able to define the absolute configuration 
(AC) of the chiral molecules in order to study their behaviour in  biological systems as 
many biological activities are only associated with one specific AC. A very tragic example
in this regard is that of the drug thalidomide which was originally prescribed to pregnant women in 
Europe in 1950s to cure morning sickness but resulted in birth-defects in thousands of 
infants. It was found through a study done on rodents that the (R) enantiomer is indeed 
a sedative but the (S) enantiomer is a teratogen\cite{}. Furthemore, these enantiomers were 
found to rapidly interconvert in vivo in humans\cite{} because of which separating 
these two forms before use is useless.
\\
The first step in experimental determination of AC either involves the separation of 
a racemic mixture (equal concentration of both the enantiomers) into individual enantiomers
using chiral resolution methods like chiral column chromatography or a selective synthesis
of a given enantiomer using asymmetric synthesis. X-ray crystallography would then be the 
most popular procedure, provided one can crystallize the molecule first which usually 
requires the presence of heavy atom(s). Alternatively, one can measure the responses of 
the enantiomers like CD, optical rotation using corresponding chiroptical spectroscopy
techniques like electronic and vibrational CD i.e. ECD (UV-VIS) and VCD (IR), polarimetry etc.
and compare that against some suitable reference chiral substrate. If no such reference
exists, then accuarate {\em ab initio} methods can come to the rescuse and help validate
these experimental measurements. Sometimes, designing an asymmteric synthesis might be  
too complicated for molecules with large number of stereocenters and in these cases,
theory can take the lead and predict the AC of such compounds. In other words, theoretical
calculations of these chiroptical properties could serve as a very useful 
computational tool for modern organic chemists.\\\\
%Theoretical and computational chemistry has grown by leaps and bounds over the
%past several decades. {\em ab initio} quantum chemical methods can now accurately
%predict a variety of molecular properties. More recently, these methods have
%been extended to calculate chiroptical
%properties like optical rotation. Theoretical calculations of optical rotation
%can be very useful in solving experimental chemical problems.  Specifically, they
%can provide a computational tool to modern organic chemists to determine
%the absolute stereochemical configuration of a chiral compound, thus alleviating the experimental difficulties
%involved.\\ 
Rosenfeld developed a quantum mechanical recipe for calculating optical rotations in the year
1929\cite{Rosenfeld29}. He showed that in the presence of an external electromagnetic field,
the induced dipole moment in a molecule can be written as:
\begin{equation}\langle\vec{\mu}\rangle = \alpha\vec{E} + \frac{1}{\omega}\textbf{G}^\prime\frac{\partial\vec{B}}{\partial t}
\end{equation} where $\vec{E} $ and $\vec{B}$ are the time dependent electric and
magnetic field vectors respectively, $\omega$ is the frequency of the filed, $\alpha$ is the electric dipole
polarizability tensor and the $\textbf{G}^\prime$ tensor, also known as the Rosenfeld tensor is associated 
with the property of optical rotation. Invoking time-dependent Schr\"odinger's equation in conjunction 
with perturbation theory, expanding the expectation value of dipole operator in orders of perturbation 
and identifying the term linear in $\frac{\partial\vec{B}}{\partial t}$, 
for exact states, the $\textbf{G}^\prime$ tensor looks 
\\
\begin{equation}
%\textbf{G}^{\prime}(\omega) = -\frac{1}{\hbar} \text{Im}\sum_{n \neq 0}\left[\frac{\bra
%{\psi_o}\vec{\mu}\ket{\psi_n}\bra{\psi_n}\vec{m}\ket{\psi_o}}{\omega_{no} - \omega - i\Gamma_{no}} + \frac{\bra
%{\psi_o}\vec{m}\ket{\psi_n}\bra{\psi_n}\vec{\mu}\ket{\psi_o}}{\omega_{no} + \omega + i\Gamma_{no}} \right]
\textbf{G}^{\prime}(\omega) = -\frac{2}{\hbar} \text{Im}\sum_{n \neq 0}\frac{\omega \bra
{\psi_o}\vec{\mu}\ket{\psi_n}\bra{\psi_n}\vec{m}\ket{\psi_o}}{{\omega_{no}}^2 - {\omega}^2 }.
\end{equation}
\\
Here, $\vec{\mu}$ and $\vec{m}$ are electric and magnetic dipole operators, 
$\vec{\mu} = \sum\limits_i q_i \vec{{r}_i}$ and $\vec{m} = \sum\limits_i \frac{q_i}{2m_i} \vec{{r}_i} \times \vec{{p}_i}$, $\omega$ is the frequency of light, $\omega_{no}$ is the excitation energy of the state $\psi_n$.
%\cite{CrawfordTamJPA07}. 
Im means the imaginary part of the equation and the summation runs over all the excited
states $\psi_n$. Specific rotation, ${\lbrack\alpha\rbrack}_\omega$ is related to the trace of this tensor 
normalized by pathlength and concentration and is usually reported in units of in deg dm$^{\text{-1}}$(g/mL)$^{\text{-1}}$,\cite{Crawford06}
\\
\begin{equation}
{\lbrack\alpha\rbrack}_{\omega} = \frac{(72.0 \times 10^6){\hbar}^2 N_A\;\omega}{c^2{m_e}^2 M} \times \left[ \frac{1}{3}Tr(\textbf{G}^\prime)\right]
\end{equation}
\\
Here, $\textbf{G}^\prime$ and $\omega$ are in atomic units, c is the speed of light (m/s), m$_{\text{e}}$ is the 
mass of electron (kg), M is the molecular mass (amu) and $N_A$ is Avogadro's number. Calculating the 
$\textbf{G}^\prime$ tensor using equation (1.2) requires explicit calculations of a large number of excited states which is very computationally expensive. Most ab initio methods use the linear response formalism
\cite{Koch90,Kobayashi94} instead to calculate ${\lbrack\alpha\rbrack}_\omega$.
\begin{equation}
\textbf{G}^{\prime}(\omega) = Im\langle\langle\mu;m\rangle\rangle
\end{equation} The linear response approach focuses on the perturbation of the
ground state wavefunction in the presence of an external field, avoiding
excited state calculations. Different levels of theory have been employed
over the years for calculating optical rotation. Polavarapu was the first to
calculate ab-initio optical rotation of gas phase molecules using the time
dependent Hartree Fock method\cite{Polavarapu96}. The signs of the calculated optical
rotation values matched with that of the experiment for most of his structures,
while their magnitudes differed generally by a factor of two. Inspired by
Polavarapu's success, Cheeseman et al. \cite{Cheeseman00,Stephens01} included correlation
effects in the calculations by applying density functional theory.
Using large basis sets with diffuse functions like aug-cc-pVDZ and aug-cc-pVTZ,\cite{Dunning89} they were able to match the experimental values very closely, with a deviation of 20-25 degrees
for a set of 28 chiral molecules. Ruud et al. extended the calculations to the
coupled cluster level using the coupled cluster response
theory (CCLR) developed by Koch and J{\o}rgensen\cite{Koch90},
and obtained promising results\cite{Ruud03}. These initial breakthroughs have allowed optical
rotation calculations to be performed on a large number of chiral molecules over the years.  As
such, the gas phase calculations have become very reliable now and are being
increasingly used to determine absolute configurations of many chiral molecules.\cite{Kondru99} \\However, since most of the experimental measurements of optical rotation are
performed in solutions, we need to be able to calculate this property for solutions
as well. But a larger number of molecules and stronger intermolecular interactions
makes the modelling of solution phase optical rotation much more challenging and
complicated than that of the gas phase. In recent years, the ab inito methods
mentioned above have been combined with various solvation based
approaches\cite{Neugebauer05,Neugebauer09,Mennucci02,Tomasi05,JensenGordon96}
to calculate the optical rotation in solution. This review condiders some of
the popular approaches that are being used and describes their working
mechanisms and performances in detail. In the current/future work section of
this review, some reduced scaling techniques are proposed with the aim of
making these methods more efficient and computationally practical. Also, some
of the essential, underlying coupled cluster and response theory used for
optical rotation calculations are presented and discussed.

%\section{Introduction}

%    a. chirality - same story as my Internal Seminar.
