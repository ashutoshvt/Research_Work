%\section{Introduction}

Start talking about optical rotation as a property. Science behind this property!!
Points to talk:
1. Some history of optical rotation (MC Tam's thesis)
    a. chirality - same story as my Internal Seminar.
    b. Drug thalidomide example
    c. Chiroptical spectroscopy
    d. Theory as a tool!
    e. Some history of OR calculations (Harley's thesis)
    f. TD-DFT not enough in most cases, CC needed (known problems of DFT)
    g. Challenges of CC - (mention about MD as well.)
    h. My work comes in here!


Theoretical and computational chemistry has grown by leaps and bounds over the
past several decades. Ab initio quantum chemical methods can now accurately
predict a variety of molecular properties. More recently, these methods have
been extended to calculate chiroptical
properties like optical rotation. Theoretical calculations of optical rotation
can be very useful in solving experimental chemical problems.  Specifically, they
can provide a computational tool to modern organic chemists to determine
the absolute stereochemical configuration of a chiral compound, thus alleviating the experimental difficulties
involved.\\ Rosenfeld laid the theoretical foundations for calculating optical
roatation in 1929\cite{Rosenfeld29}. He showed that in the presence of an
external field, the induced dipole moment in a molecule can be written as:
\begin{equation}\langle\vec{\mu}\rangle = \alpha\vec{E} + \frac{1}{\omega}\textbf{G}^\prime\frac{\partial\vec{B}}{\partial t}
\end{equation} where $\vec{E} $ and $\vec{B}$ are the time dependent electric and
magnetic field vectors respectively, $\omega$ is the frequency of the filed, $\alpha$ is the electric dipole
polarizability tensor and the $\textbf{G}^\prime$ tensor, also known as the Rosenfeld
tensor contains the recipe for calculating optical rotation.
\begin{equation}
\textbf{G}^{\prime}(\omega) = -\frac{1}{\hbar} Im\sum_{n \neq 0}\left[\frac{\bra
{\psi_o}\vec{\mu}\ket{\psi_n}\bra{\psi_n}\vec{m}\ket{\psi_o}}{\omega_{no} - \omega - i\Gamma_{no}} + \frac{\bra
{\psi_o}\vec{m}\ket{\psi_n}\bra{\psi_n}\vec{\mu}\ket{\psi_o}}{\omega_{no} + \omega + i\Gamma_{no}} \right]
\end{equation}
Here, $\vec{\mu}$ and $\vec{m}$ are electric and magnetic dipole operators,
%: $\mu =
%\sum_ir_i,\;\; m= \sum_ir_i\times p_i$ (in atomic units).  
 $\omega$ is the frequency of light, $\omega_{no}$ is the excitation energy of
the state $\psi_n$ and $\Gamma_{no}$ is the dephasing rate between the states
$\psi_o$ and $\psi_n$.
% taken to be zero when the field frequency is far from
%resonance
\cite{CrawfordTamJPA07}. $Im$ means the imaginary part of the equation and the
summation runs over all the excited states $\psi_n$. The trace of this tensor
is related to the specific rotation, usually denoted as
${\lbrack\alpha\rbrack}_\omega$ in
deg dm$^{\text{-1}}$(g/mL)$^{\text{-1}}$
%After averaging over all possible
%orientations of the molecule,the following expression is obtained
\cite{Crawford06}.
\begin{equation}
{\lbrack\alpha\rbrack}_{\omega} = \frac{(72.0 \times 10^6){\hbar}^2 N_A\;\omega}{c^2{m_e}^2 M} \times \left[ \frac{1}{3}Tr(\textbf{G}^\prime)\right]
\end{equation}
Here, $\textbf{G}^\prime$ and $\omega$ are in atomic units, c is the speed of light (m/s), m$_{\text{e}}$ is the mass of
electron (kg), M is the molecular mass (amu) and $N_A$ is Avogadro's
number. Calculating the $\textbf{G}^\prime$ tensor using equation (2) requires explicit calculations of a large
number of excited states, which is very computationally expensive. Most ab initio methods
use the linear response formalism\cite{Koch90,Kobayashi94} instead to calculate
${\lbrack\alpha\rbrack}_\omega$.
\begin{equation}
\textbf{G}^{\prime}(\omega) = Im\langle\langle\mu;m\rangle\rangle
\end{equation} The linear response approach focuses on the perturbation of the
ground state wavefunction in the presence of an external field, avoiding
excited state calculations. Different levels of theory have been employed
over the years for calculating optical rotation. Polavarapu was the first to
calculate ab-initio optical rotation of gas phase molecules using the time
dependent Hartree Fock method\cite{Polavarapu96}. The signs of the calculated optical
rotation values matched with that of the experiment for most of his structures,
while their magnitudes differed generally by a factor of two. Inspired by
Polavarapu's success, Cheeseman et al. \cite{Cheeseman00,Stephens01} included correlation
effects in the calculations by applying density functional theory.
Using large basis sets with diffuse functions like aug-cc-pVDZ and aug-cc-pVTZ,\cite{Dunning89} they were able to match the experimental values very closely, with a deviation of 20-25 degrees
for a set of 28 chiral molecules. Ruud et al. extended the calculations to the
coupled cluster level using the coupled cluster response
theory (CCLR) developed by Koch and J{\o}rgensen\cite{Koch90},
and obtained promising results\cite{Ruud03}. These initial breakthroughs have allowed optical
rotation calculations to be performed on a large number of chiral molecules over the years.  As
such, the gas phase calculations have become very reliable now and are being
increasingly used to determine absolute configurations of many chiral molecules.\cite{Kondru99} \\However, since most of the experimental measurements of optical rotation are
performed in solutions, we need to be able to calculate this property for solutions
as well. But a larger number of molecules and stronger intermolecular interactions
makes the modelling of solution phase optical rotation much more challenging and
complicated than that of the gas phase. In recent years, the ab inito methods
mentioned above have been combined with various solvation based
approaches\cite{Neugebauer05,Neugebauer09,Mennucci02,Tomasi05,JensenGordon96}
to calculate the optical rotation in solution. This review condiders some of
the popular approaches that are being used and describes their working
mechanisms and performances in detail. In the current/future work section of
this review, some reduced scaling techniques are proposed with the aim of
making these methods more efficient and computationally practical. Also, some
of the essential, underlying coupled cluster and response theory used for
optical rotation calculations are presented and discussed.

