%\section{Introduction}

%    a. chirality - same story as my Internal Seminar.
%    b. Drug thalidomide example
%    c. Chiroptical spectroscopy
%    d. Theory as a tool!
%    e. Some history of OR calculations (Harley's thesis)
%    f. TD-DFT not enough in most cases, CC needed (known problems of DFT)
%    g. Challenges of CC - (mention about MD as well.)
%    h. My work comes in here!
%	i. local correlatin - which chapter discusses what... end by conclusion..
\section{Chirality}
Lord Kelvin was the first to use the word `chiral' in his lecture at Oxford 
University in 1894: ``I call any geometrical figure, or group of points, 
`chiral', and say that it has chirality if its image in a plane mirror, ideally realized, 
cannot be brought to coincide with itself"\cite{}. 
\\
There are different ways in which a molecule 
could be chiral. The most common example being the presence of a chiral center 
like a carbon or nitrogen atom with four different substituents. Axial chirality, usually 
possessed by molecules with cumulated double bonds like allenes and chirality planes 
found in compounds like paracyclophanes are other popular examples of chiral elements 
found in nature. \\
A chiral molecule and its mirror image are known as enantiomers. Although the
enantiomers have idential phyical properties like boiling and melting points their
interactions in a chiral environment can be quite different. For example,
while the (R) enantiomer of the compound limonene smells like lemons and oranges,
as the name sugggests, the (S) enantiomer on the other hand smells like turpentine. 
Thus, the interaction of the (S) enantiomer with the chiral molecules constituting the olfactory receptors 
responsible for the sensation of smell is totally different from its fellow enantiomer.
The enantiomers also interact differently with the left- and right-hand circularly polarized (LCP and
RCP) light in processes of absorption, refraction, etc. A chiral molecule posesses the 
properties of circular dichroism (CD) and circular birefringence, i.e. it has different absorption 
coefficents and refractive indices for the LCP and RCP light, respectively. As a result of circular 
birefringence, the plane of polarization of a plane-polarized light in a chiral medium gets rotated/shifted
by an amount known as the optical rotation of the medium. It should be noted that an enantiomeric pair of chiral
molecules has equal magnitudes of optical rotation but in opposite directions.\\
Since almost every part of the human body is composed of chiral molecules, its unsurprising that
more than 60\% of the pharmaceutical drugs are chiral as well. Hence, one of the most 
important aspect of the drug-design research is to be able to define the absolute configuration 
(AC) of the chiral molecules in order to study their behaviour in  biological systems as 
many biological activities are only associated with one specific AC. A very tragic example
in this regard is that of the drug thalidomide which was originally prescribed to pregnant women in 
Europe in 1950s to cure morning sickness but resulted in birth-defects in thousands of 
infants. It was found through a study done on rodents that the (R) enantiomer is indeed 
a sedative but the (S) enantiomer is a teratogen\cite{}. Furthemore, these enantiomers were 
found to rapidly interconvert in vivo in humans\cite{} because of which separating 
these two forms before use doesn't help.
\\
The first step in experimental determination of AC either involves the separation of 
a racemic mixture (equal concentration of both the enantiomers) into individual enantiomers
using chiral resolution methods like chiral column chromatography or a selective synthesis
of a given enantiomer using the process of asymmetric synthesis. X-ray crystallography would then be the 
most popular procedure, provided one can crystallize the molecule first which usually 
requires the presence of heavy atom(s). Alternatively, one can measure the responses of 
the enantiomers like CD and optical rotation using corresponding chiroptical spectroscopy
techniques like electronic and vibrational CD i.e. ECD (UV-VIS) and VCD (IR) and polarimetry,
followed by comparision against some suitable reference chiral substrate. However, in cases of 
ambiguity regarding appropriate references, accurate {\em ab initio} methods can come to the rescue and 
help validate these experimental measurements. Furthermore, designing an asymmteric synthesis might not
be straightforward for many chiral systems, especially the ones with large number of stereocenters. 
In such cases, theory can even take the lead and predict the corresponding AC. In other words, theoretical
calculations of these chiroptical properties could serve as a very useful computational tool for 
modern organic chemists. In particular, accurate optical rotation calculations are highly desired 
as the experimental techniques for measuring this property are fairly stable and robust in 
both gas and solution phases.
\\
%Theoretical and computational chemistry has grown by leaps and bounds over the
%past several decades. {\em ab initio} quantum chemical methods can now accurately
%predict a variety of molecular properties. More recently, these methods have
%been extended to calculate chiroptical
%properties like optical rotation. Theoretical calculations of optical rotation
%can be very useful in solving experimental chemical problems.  Specifically, they
%can provide a computational tool to modern organic chemists to determine
%the absolute stereochemical configuration of a chiral compound, thus alleviating the experimental difficulties
%involved.\\ 
\section{{\em Ab Initio} Optical Rotation Calculations}
Rosenfeld developed a quantum mechanical recipe for calculating optical rotations in the year
1929\cite{Rosenfeld29}. He showed that in the presence of an external electromagnetic field,
the induced dipole moment in a molecule can be written as:
\begin{equation}\langle\vec{\mu}\rangle = \bm{\alpha}\vec{E} + \frac{1}{\omega}\textbf{G}^\prime\frac{\partial\vec{B}}{\partial t}
\end{equation} where $\vec{E} $ and $\vec{B}$ are the time dependent electric and
magnetic field vectors respectively, $\omega$ is the frequency of the filed, $\alpha$ is the electric dipole
polarizability tensor and the $\textbf{G}^\prime$ tensor, also known as the Rosenfeld tensor is associated 
with the property of optical rotation. Invoking time-dependent Schr\"odinger's equation in conjunction 
with perturbation theory, expanding the expectation value of dipole operator in orders of perturbation 
and identifying the term linear in $\vec{E}$ and $\frac{\partial\vec{B}}{\partial t}$, 
for exact states, the $\bm{\alpha}$ and $\textbf{G}^\prime$ tensor look like,
\\
\begin{equation}
\begin{split}
& \bm{\alpha}(\omega) = -\frac{2}{\hbar} \sum_{n \neq 0}\frac{\omega_{no}\bra
{\psi_o}\vec{\mu}\ket{\psi_n}\bra{\psi_n}\vec{\mu}\ket{\psi_o}}{{\omega_{no}}^2 - {\omega}^2 }\\
& \textbf{G}^{\prime}(\omega) = -\frac{2}{\hbar} \text{Im}\sum_{n \neq 0}\frac{\omega \bra
{\psi_o}\vec{\mu}\ket{\psi_n}\bra{\psi_n}\vec{m}\ket{\psi_o}}{{\omega_{no}}^2 - {\omega}^2 }\\
\end{split}
\end{equation}
\\
Here, $\vec{\mu}$ and $\vec{m}$ are electric and magnetic dipole operators, 
$\vec{\mu} = \sum\limits_i q_i \vec{{r}_i}$ and $\vec{m} = \sum\limits_i \frac{q_i}{2m_i} \vec{{r}_i} \times \vec{{p}_i}$, $\omega$ is the frequency of light, $\omega_{no}$ is the excitation energy of the state $\psi_n$.
%\cite{CrawfordTamJPA07}. 
Im means the imaginary part of the equation and the summation runs over all the excited
states $\psi_n$. Specific rotation, ${\lbrack\alpha\rbrack}_\omega$ is related to the trace of the Rosenfeld tensor 
normalized by pathlength and concentration and is usually reported in the units of deg dm$^{\text{-1}}$(g/mL)$^{\text{-1}}$,\cite{Crawford06}
\\
\begin{equation}
{\lbrack\alpha\rbrack}_{\omega} = \frac{(72.0 \times 10^6){\hbar}^2 N_A\;\omega}{c^2{m_e}^2 M} \times \left[ \frac{1}{3}Tr(\textbf{G}^\prime)\right]
\end{equation}
\\
Here, $\textbf{G}^\prime$ and $\omega$ are in atomic units, c is the speed of light (m/s), m$_{\text{e}}$ is the 
mass of electron (kg), M is the molecular mass (amu) and $N_A$ is Avogadro's number. Calculating the 
$\textbf{G}^\prime$ tensor using equation (1.2) requires explicit calculations of a large number of 
excited states which is computationally prohibitive. A more practical approach is to invoke the response formalism
\cite{Koch90,Kobayashi94}, which casts these tensors as response functions,
\begin{equation}
\begin{split}
&\bm{\alpha}(\omega) = \langle\langle\vec{\mu};\vec{\mu}\rangle\rangle\\
&\textbf{G}^{\prime}(\omega) = -\text{Im}\langle\langle\vec{\mu};\vec{m}\rangle\rangle\\
\end{split}
\end{equation} 
The response theory focuses on the perturbation of the ground state wavefunction in the presence of an external 
field, avoiding explicit calculations of excited states. Polavarapu was the first to use this theory to calculate 
{\em ab initio} specific rotations of molecules using time dependent Hartree Fock (TDHF) method \cite{Polavarapu96}. 
The signs of the calculated values matched with that of the experiment for most of his structures, while their magnitudes 
differed generally by a factor of two. Inspired by Polavarapu's success, Cheeseman et al. \cite{Cheeseman00,Stephens01} 
included correlation effects in these calculations by using TD-DFT. Using large basis sets with diffuse functions like 
aug-cc-pVDZ and aug-cc-pVTZ,\cite{Dunning89} they were able to match the experimental values very closely, with a 
deviation of 20-25 degrees for a set of 28 chiral molecules. Ruud et al. extended the calculations to the coupled cluster 
level using coupled cluster response theory developed by Koch and J{\o}rgensen\cite{Koch90}, and obtained very 
promising results\cite{Ruud03}. Unsurprisingly, these response based simulations have been used as an accurate and 
reliable tool in the process of determining the AC of a large number of chiral molecules\cite{Kondru99}. However, all 
these calculations were done on single isolated molecules and thus ideally they should only be expected to match vapor phase 
optical rotation (OR) which can be measured by using cavity ring down spectrometry, thanks to the pioneering 
works of Vacaaro and co-workers.\cite{} Thus, in order to match the solution phase OR measurements, the response theory
needs to be combined with accurate solvation models. \\ 
However, works in this regard, as shown in chapter 2 of this manuscript have been less than satisfactory and 
the only robust way of accounting for solvent effects seem to be including the solvent molecules explicitly 
in the QM calculations. Since rotation is a dynamic property, one needs to average OR values over several 
configurations or snapshots (obtained by molecular dynamics (MD) simulations) to be able to match experimental 
values. Multiple calculations coupled with the polynomical scaling of coupled cluster (CC) methods, puts a heavy burden 
on the computational resources. An obvious way to make these compuations practical is to seek the help of CC reduced-scaling 
techniques like projected natural orbitals (PNO) approach\cite{} which has been really sucessfull in describing energetics of 
large molecular systems. However, works towards the extension of these methods to calculate chiroptical properties has been 
fairly limited\cite{}. McAlexander and Crawford in their recent work\cite{} demonstrated that the regular PNO approach
suffers from slow convergence towards the full canonical result. The main focus of this work is identifying the problems
associated with such approaches for calculating optical rotation and finding out ways in which they can be alleviated.
The theoretical tools needed to calculate chiroptical properties are derived in Chapter 2 within the context of coupled cluster 
linear response theory.
are derived in chapter 2.
and talks about the challenges of modelling optical rotation in solutions. Chapter 3 


describes application of a very useful 
technique to reduce the computational costs for response properties and identifies the inherent
problems associated with such approaches. Corrected in chapter 3 and extended to reduced-scaling
domain in chapter 4.
