Reduced-scaling techniques have extended the application of accurate coupled cluster (CC)
methods to molecules with hundreds of heavy atoms\cite{Neese09,NeeseCCSD09}. However, efforts towards extension 
of these schemes to simulate linear response properties have been fairly 
limited\cite{Gauss00,Korona04,Russ04,McAlexander12,Friedrich15,Russ08}. McAlexander and Crawford\cite{McAlexander15:LRCC} recently compared the performances of
reduced-scaling methods based on projected atomic orbitals (PAOs)\cite{Saebo86,Pulay83,PulaySaebo93}, orbital specific virtuals (OSVs)\cite{Yang12} and pair natural orbitals (PNOs)\cite{Edmiston66,Meyer73,Ahlrichs75,Neese09} for calculating optical rotations using the
coupled cluster linear response theory\cite{Koch90}, where they found the PNOs to be the most compact 
representation of the virtual space. However, very tight cutoffs needed to be employed in order to 
match the canonical results. \\
In our search for an optimal virtual space for response calculations,
we investigated the performance of the frozen virtual natural orbital (FVNO) scheme,
(chapter 3) which had been applied quite successfully to bring down the 
computational costs of CC calculations involving energies, geometry optimizations, 
ionization enthalpies, etc.\cite{Sosa89,Taube05,Taube08,Landau10,DePrince13:FNOs,DePrince13}, 
for calculating dynamic polarizabilities of 
small chiral compounds\cite{Kumar17}. It was found that the virtual natural orbitals (VNOs)
obtained from the ground state MP2 density (GSMP2D) are not suitable for calculating 
polarizabilities (and hence optical rotations) in the absence of orbital relaxation 
effects as the shift in their values with respect to the full canonical result grew 
almost linearly with the number of truncated VNOs. However, introduction of orbital
relaxation can introduce spurious or unphysical poles in the linear response function
because of which they are not included in the CC linear response calculations.
On a closer inspection, we found that all the diffuse 
VNOs had very low occupation numbers (ONs) and thus were removed first
in this scheme resulting in large errors. It should be noted that these 
diffuse orbitals are absolutely essential for calculating chiroptical properties 
as many chiral molecules posess low lying Rydeberg type excited states.
We also employed a CC2\cite{Christiansen95:CC2} based correction for the external truncated 
space which was able to reduce the errors drastically, but such an approach 
could not lead to any computational savings. Finally, we constructed a first 
order perturbed density by taking the derivative of the GSMP2D with 
respect to an external electric field, with the hope that the occupation 
numbers (ONs) generated using this density would be a better metric for 
estimating the importance of a VNO for describing the response of the 
wavefunction. However, this approach offered only marginal improvement 
in the errors. On the other hand, truncating canonical molecular orbitals (CMOs) 
with high orbital energies resulted in very small errors (less than 2\% after 
removing 50\% of the virtual space) due to intrinsic cancellation of errors.\\
On the basis of the above findings, we tried out a modified FVNO scheme, FVNO(M)
in Chapter 4, where the virtual CMOs (VCMOs) were sorted according to their diffusivity which 
wa measured by the values of their orbital spatial extents (OSEs) and GSMP2D was 
built only in the space of non-diffuse (under a given OSE cutoff) orbitals. 
In other words, the diffuse part of the virtual space remains unaffected 
by the FVNO procedure. Pilot studies on H$_2$O$_2$ molecule showed that the
FVNO(M) method yielded minimal errors compared to the original scheme 
for polarizabilities, optical rotations, rotational strengths, 
excitation energies, etc. However, more systematic studies are needed to 
make it a truly black-box method. We also looked at the perturbed density 
model in more details by first constructing the (full) CCSD first order 
perturbed density. It was found that the diagonal elements of this density 
matrix are always zeros irrespective of the perturbation. Clearly, the VNOs
obtained by the diagonalization of such densities don't carry any useful information.
Ideally, the structure of the perturbed density should be able to mimic these (linear response) properties 
which are second order in the perturbation. Hence, we proposed a new approach called
FVNO++, where a second-order perturbed density is constructed and then diagonalized
to define the virtual space. Pilot studies on H$_2$O$_2$ molecule and chiral linear chain
hydrogen helices ((P)-$(H_2)_n$) using the FVNO++ approach have shown quite promising results 
for both polarizabilities and optical rotations. The major advantage of this approach 
over FVNO(M) is that ONs are a much more robust truncation criterion compared to the OSEs.\\
%However, calculations with other choices of perturbed densities also need to be
%done, especially for bigger molecules to make the FVNO++ truncation scheme truly robust.
%We are currently in the process of a RI-CC2 linear response code\cite{} which we intend
%to use in conjunction with the FVNO++ formalism to tackle larger molecular systems.
%Furthermore, we propose to extend this approach to the reduced-scaling techniques based
%on pair natural orbitals\cite{}.
We extended the above analyses to the PNO domain in Chapter 5 as the  
the PNOs can be seen as an extension of the concept of 
natural orbitals. Consequently, the PNO(M) method was designed where
all the diffuse VCMOs were retained and every pair of occupied orbitals
inherited this global virtual space. Just as before, PNO(M) significantly
accelerates the convergence of these properties towards the full
canonical result. However, this means that every occupied pair becomes a 
strong pair and hence can't be neglected, which has an adverse affect
on the reduced scaling capabilities of the PNO scheme. Along the 
lines of FVNO++, PNO++ approach was formulated where the structure of 
the pair specific densities resemble with the contributions of a given
pair to the response functions. Preliminary results on chiral linear chain 
hydrogen helices indicate a performance similar to the PNO(M) approach
for both polarizabilities and optical rotations. However, calculations on 
two and three dimensional structures involving different forms of second-order 
perturbed densities need to be carried out in order to make PNO++ a truly robust method. \\
Encouraged by the performances of these methods, we plan to use them in 
conjunction with density-fitted CC linear response codes\cite{Friese12} to target larger
systems like solvated molecular clusters. We assume that a larger solvent shell 
around the solute molecule would require comparitively smaller number of 
snapshots for the optical rotations to converge to the experimental value. Thus, we can significantly
lower down the computational costs involved in CC calculations of optical rotations
of solvated molecules.
