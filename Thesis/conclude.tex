Chapter 2 of this work 


focuses on covering the necessary theoretical background
for deriving the coupled cluster (CC) response functions and identifying
the challenges associated with these approaches.
As a prelude to the CC response theory, the formalism for obtaining ground state CC energies
is discussed after a brief description of Hartree-Fock and Configuration Interaction models.
The next section derives the simplified gradient expressions of the CC energy and subsequently
parametrizes the left hand CC wavefunction.


On the basis of the above findings, we conclude that, in the absence of
orbital relaxation, virtual NOs are not
suited for higher-order property calculations such as dynamic
polarizabilities, and that the occupation number is not an acceptable
criterion for estimating the importance of a virtual orbital for such
calculations.  Although the use of external space corrections based on CC2
polarizabilities reduces the observed truncation errors, they are too
relatively costly to be practical for large molecular systems.  Furthermore,
the use of perturbed virtual NOs offers only slight improvement as compared to
unperturbed virtual NOs.  CMOs, on the other hand, provide a much more stable
mechanism for reducing the size of the virtual space --- with truncation of up
to 50\% of the orbitals yielding shifts of less than 2\% as compared to
full-space calculation -- but the source of their success lies in a
significant cancellation of errors.  Although further systematic studies are
needed, the dipole amplitude appears to provide a useful threshold for an {\em
a priori} truncation of the CMO virtual space.


We have developed a new formalism FVNO++ which offers a very
compact representation of the virtual space for response
property calculations. Based on plot studies on H$_2$O$_2$ using second-order
electric dipole based perturbed densities, we conclude that this
approach has faster convergence towards the full canonical result for both
polarizabilities and optical rotations than the original FVNO approach.
However, calculations with other choices of perturbed densities also need to be
done, especially for bigger molecules to make the FVNO++ truncation scheme truly robust.
We are currently in the process of a RI-CC2 linear response code\cite{} which we intend
to use in conjunction with the FVNO++ formalism to tackle larger molecular systems.
Furthermore, we propose to extend this approach to the reduced-scaling techniques based
on pair natural orbitals\cite{}.

We have a developed an extension of the PNO method, the PNO++
approach for calculating response properties. The pair
specific densities constructed in this approach strongly
resemble in structure with the underlying response equations
and preliminary results on linear chain hydrogen helices
seem to indicate that the ONs obtained by diagonalizing these densities
are a good metric for estimating the importance of a
VNO in the response of the wavefunction. we plan to use this
method with RI-CC2/CCSD linear response theory to calculate
CC level properties of large solvated clusters.
Furthermore, calculations on two and three dimensional structures
involving different forms of second-order perturbed density
are still running and are expected to give more insight
which can be used to make PNO++ a truly black-box method.
