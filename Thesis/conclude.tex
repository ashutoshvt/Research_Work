Reduced-scaling techniques have extended the application of accurate coupled cluster (CC)
methods to molecules with hundreds of heavy atoms\cite{}. However, efforts towards extension 
of these schemes to simulate chiroptical properties like optical rotations have been fairly 
limited\cite{}. McAlexander and Crawford recently compared the performances of
reduced-scaling methods based on projected atomic orbitals (PAOs)\cite{}, orbital specific virtuals (OSVs)\cite{} 
and pair natural orbitals (PNOs)\cite{} for calculating optical rotations\cite{} using the
coupled cluster linear response theory, where they found the PNOs to be the most compact 
representation of the virtual space. However, very tight cutoffs needed to be employed to 
match the canonical results. \\
In our search for an optimal virtual space for response calculations,
we investigated the behavior of the frozen virtual natural orbital (FVNO) scheme,
(chapter 3) which has been applied quite successfully to bring down the 
computational costs of CC calculations involving energies, geometry optimizations, 
ionization enthalpies, etc.\cite{} for calculating dynamic polarizabilities of 
small chiral compounds. It was found that in the absence of orbital relaxation, 
virtual NOs are not suited for higher-order property calculations such as dynamic
polarizabilities, and that the occupation number is not an acceptable
criterion for estimating the importance of a virtual orbital for such
calculations.  Although the use of external space corrections based on CC2
polarizabilities reduces the observed truncation errors, they are too
relatively costly to be practical for large molecular systems.  Furthermore,
the use of perturbed virtual NOs offers only slight improvement as compared to
unperturbed virtual NOs.  CMOs, on the other hand, provide a much more stable
mechanism for reducing the size of the virtual space --- with truncation of up
to 50\% of the orbitals yielding shifts of less than 2\% as compared to
full-space calculation -- but the source of their success lies in a
significant cancellation of errors.  Although further systematic studies are
needed, the dipole amplitude appears to provide a useful threshold for an {\em
a priori} truncation of the CMO virtual space.

Based on the above findings, we developed a new formalism called FVNO++ wmethod
(chapter 4) where second-order perturbed densities were constructed for obtaining
natural orbitals. Based on plot studies of H$_2$O$_2$ using electric dipole based 
perturbed densities, we conclude that this approach has faster convergence towards 
the full canonical result for both polarizabilities and optical rotations 
than the original FVNO approach.


%However, calculations with other choices of perturbed densities also need to be
%done, especially for bigger molecules to make the FVNO++ truncation scheme truly robust.
%We are currently in the process of a RI-CC2 linear response code\cite{} which we intend
%to use in conjunction with the FVNO++ formalism to tackle larger molecular systems.
%Furthermore, we propose to extend this approach to the reduced-scaling techniques based
%on pair natural orbitals\cite{}.
We have a developed an extension of the PNO method, the PNO++
approach for calculating response properties. The pair
specific densities constructed in this approach strongly
resemble in structure with the underlying response equations
and preliminary results on linear chain hydrogen helices
seem to indicate that the ONs obtained by diagonalizing these densities
are a good metric for estimating the importance of a
VNO in the response of the wavefunction. we plan to use this
method with RI-CC2/CCSD linear response theory to calculate
CC level properties of large solvated clusters.
Furthermore, calculations on two and three dimensional structures
involving different forms of second-order perturbed density
are still running and are expected to give more insight
which can be used to make PNO++ a truly black-box method.
