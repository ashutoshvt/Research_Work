%\section{Theory}
%\subsection{Schr\"odinger's equation}
%The most fundamental problem in quantum mechanics is to solve the time independent
%non-relativistic  Schr\"odinger equation \cite{Schrodinger26}. 
% \begin{equation}
%-\frac{1}{2}\nabla^2\psi_n(r) + V(r)\psi_n(r) = E_n\psi_n(r) . \end{equation}
%In atomic units, the Hamiltonian for N electrons and M nuclei is
%\begin{equation} \hat{H}_{elec} = -\frac{1}{2}\sum\limits_{i=1}^N\nabla_{i}^2 -
%\frac{1}{2M_A}\sum\limits_{A=1}^M\nabla_{A}^2 -
%\sum\limits_{i=1}^N\sum\limits_{A=1}^M\frac{Z_A}{r_{iA}}+\sum\limits_{i=1}^N\sum\limits_{j>i}^N\frac{1}{r_{ij}}.
%\end{equation}
%The first term in above equation is the operator for the kinetic energy of the electrons; the second term 
%is the operator fpr the kinetic energy of the nuclei; the third term represents the coulomb interaction
%between electrons and nuclei; the fourth and fifth terms represent the repulsion between electrons and
%between nuclei,respectively. Solving this eigenvalue equation gives us the wavefunction $\psi_n$, and $E_n$,
%the energy of the system. It can be solved exactly for one electron
%hydrogen-like species, while approximate solutions are obtained for higher
%electron systems.  
%\subsection{Born-Oppenheimer (BO) Approximation}
%The BO approximation was proposed by Max Born and J.Robert
%Oppenheimer\cite{Born27}. It states that the wavefunction of a molecule can be
%broken down into its electronic and nuclear components. \\ \centerline{$\psi_{total} =
%\psi_{electronic}*\psi_{nuclear}$ }\\.The rationale behind this approximation is
%the lack of coupling between the electronic and nuclear degrees of freedom in a
%molecule. Since the nuclei are a lot heavier compared to electrons, they can be
%assumed to be stationary while solving the electronic Schr\"odinger equation. The
%electronic Hamiltonian for a system of N electrons and M nuclei can be written
%as : 
%\begin{equation}
%\hat{H}_{elec} = -\frac{1}{2}\sum\limits_{i=1}^N\nabla_{i}^2-\sum\limits_{i=1}^N\sum\limits_{A=1}^M\frac{Z_A}{r_{iA}}+\sum\limits_{i=1}^N\sum\limits_{j>i}^N\frac{1}{r_{ij}}.
%\end{equation}
%Thus, the electronic wavefunction and hence the electronic
%energy depends parametrically on the nuclear coordinates and can be written as:\\
%\centerline{ $\psi_{elec} = \psi_{elec}(\left\{{r_i}\right\};\left\{{R_A}\right\})$ ,
%$E_{elec} = E_{elec}\left\{R_A\right\}.$}\\
%The total energy of the fixed nuclei system can then be written as: 
%\begin{equation}
% E_{tot} = E_{elec} + \sum\limits_{A=1}^M\sum\limits_{B>A}^M \frac{Z_AZ_B}{R_{AB}}
%\end{equation}
%Only because of this approximation can we think of the potential energy surfaces (PES) of molecules.
%\subsection{Hartree-Fock (HF) Approximation}
%The Hartree-Fock approximation assumes that the exact wavefunction of a N electron
%system can be approximated by a single slater determinant (SD) of N spin
%orbitals\cite{SzaboOstlund}. For a two electron system, the SD looks like : \begin{equation}
%\Psi = (2!)^{-1/2}
%\begin{vmatrix}
%\chi_1(x_1) & \chi_2(x_1)  \\
%\chi_1(x_2) & \chi_2(x_2)  \\
%%. & . & \; & .\\
%%\chi_1(x_N) & \chi_2(x_N) & ... &\chi_N(x_N) \\
% \end{vmatrix}
%\end{equation}
%The energy expectation value of a slater determinant is a functional of the
%spin-orbitals. By invoking the  variational principle, optimized
%spin orbitals are obtained by minimizing the energy functional with respect to
%the orbitals. This minimization procedure, coupled with the constraint of an
%orthonormal basis, gives rise to the following equation for spin orbitals:
%\begin{equation}
%\left[ h(1) + \sum\limits_{b=1}^NJ_b(1) - K_b(1)\right] \chi_a(1) = \sum\limits_{b=1}^N \epsilon_{ba}\chi_b(1) \;\;\;\;\;a = 1,2,....N
%\end{equation}
%The term in the square brackets is defined as Fock operator $f$. $J_b$ is
%called the Coulomb operator as it describes mean coulombic 
%repulsion between electrons.  \begin{equation}
%J_b(1)\chi_a(1) = \left[\int {\chi_b}^*(2){r_{12}^{-1}}\chi_b(2)\,dx_2\right]\chi_a(1) 
%\end{equation} 
%$K_b$ is known as the Exchange operator as it arises due to the antisymmetry of the wavefunction.
%\begin{equation}
%K_b(1)\chi_a(1) = \left[\int {\chi_b}^*(2){r_{12}^{-1}}\chi_a(2)\,dx_2\right]\chi_b(1) 
%\end{equation} 
%%$J_b$ and $K_b$ are called the Coulomb and exchange operators, respectively. 
%%Also, the fock operator is defined as: $ f = \left[ h(1) + \sum\limits_{b=1}^NJ_b(1) - K_b(1)\right]$
%After a unitary transformation of the basis, we get the canonical Hartree-Fock (HF) equation:
%\begin{equation}
%f\ket{\chi_a} = \epsilon_a\ket{\chi_a}
%\end{equation}
%%Thus, the many body problem gets reduced to a one electron problem in HF approximation, which is easier to solve. 
%The above equation can be solved numerically for atoms, but it becomes very
%complicated for molecules. To overcome this problem, Roothan and Hall \cite{Roothan51,Hall51} expanded the
%molecular orbital as a linear combination of known atomic basis functions,
%which casts the HF equation in the matrix form.
%%\begin{equation}
%%\chi_i = \sum\limits_{\mu=1}^K C_{\mu i}\phi_{\mu} \;\;\;\;\;\;\;\;\;\;\; \;\;\;i=1,2,...K
%%\end{equation}
%%The final equation takes the following form:
%\begin{equation}
%FC = SC \epsilon ,
%\end{equation}
%where $F$ is the Fock matrix in the given basis : $ F_{\mu\nu} = \int
%{\phi_{\mu}}^*f(1)\phi_{\nu}(1)\,dr_1 $ , C is the coefficient matrix and
%$\epsilon$ is a diagonal matrix containing orbital energies. Since the Fock
%matrix also depends on the coefficients, the above equation is solved through a
%self consistent procedure.
%  HF method is essentially a mean field theory where
%an electron moves in the average field of all the other electrons and nuclei.
%HF method recovers up to 99\% of the electronic energy and is usually the starting
%point for accurate correlated methods due to its simplicity.
%\section{Many body perturbation theory (MBPT)}
%MBPT methods try to model a multi-electron system as a perturbation applied to a simpler and solvable system and adds the appropriate %corrections in successive orders. 
%An example of MPPT is the Rayleigh-Schrodinger perturbation theory(RSPT) which treats the inter-electronic repulsions in a multi-electron %system as a perturbation.As such,it breaks down the electronic hamiltonian into two parts: 
%\begin{equation}
%\hat{H} = \hat{H_o} + \lambda\hat{V} , 
%\end{equation}
%where, $\hat{H_o}$ is the unperturbed hamiltonian, which is exactly solvable, $\hat{V}$(electron-electron repulsions) is the perturbation and %$\lambda$ is a parameter to track the perturbation.Equation(10) ensures that the energy and wavefunction of the system will also depend on %$\lambda$.Using Taylor's expansion the energy and the wavefunction of the $n^{th}$ state can be written as:
%\begin{equation}
%E_n = {E}^{(0)}_n + \lambda{E}^{(1)}_n + \lambda^2{E}^{(2)}_n + \lambda^3{E}^{(3)}_n + ...
%\end{equation}
%\begin{equation}
%\Psi_n = {\Psi}^{(0)}_n + \lambda{\Psi}^{(1)}_n + \lambda^2{\Psi}^{(2)}_n + \lambda^3{\Psi}^{(3)}_n + ...
%\end{equation}
%where, ${E}^{(k)}_n = \frac{1}{k!}\frac{d^kE_n}{d{\lambda}^k}$ and 
%${\Psi}^{(k)}_n = \frac{1}{k!}\frac{d^k\Psi_n}{d{\lambda}^k}.$
%\\ Using the above equations
%equations.First and second order equations that we get :
%\hat{H_o}{\Psi}^{(0)}_n = {E}^{(0)}_n{\Psi}^{(0)}_n
%\end{equation}
%\begin{equation}
%\hat{H_o}{\Psi}^{(1)}_n + \hat{V}{\Psi}^{(0)}_n = {E}^{(0)}_n{\Psi}^{(1)}_n + {E}^{(1)}_n{\Psi}^{(0)}_n
%\end{equation}
%\begin{equation}
%\hat{H_o}{\Psi}^{(2)}_n + \hat{V}{\Psi}^{(1)}_n = {E}^{(0)}_n{\Psi}^{(2)}_n + {E}^{(1)}_n{\Psi}^{(1)}_n +  {E}^{(2)}_n{\Psi}^{(0)}_n
%\end{equation}
%and assuming intermediate normalization : $\bra{{\Psi}^{(0)}_n}{\Psi}_n\rangle = 1$ , we obtain the corrections to the energy by projecting %${\Psi}^{(0)}_n $ to the left in the schrodinger equation.
%\begin{equation}
%{E}^{(0)}_n = \bra{{\Psi}^{(0)}_n}\hat{H_o}\ket{{\Psi}^{(0)}_n}
%\begin{equation}
%{E}^{(1)}_n = \bra{{\Psi}^{(0)}_n}\hat{V}\ket{{\Psi}^{(0)}_n}
%\end{equation}
%\begin{equation}
%{E}^{(2)}_n = \bra{{\Psi}^{(0)}_n}\hat{V}\ket{{\Psi}^{(1)}_n}
%\end{equation} 
%\begin{equation}
%{E}^{(m)}_n = \bra{{\Psi}^{(0)}_n}\hat{V}\ket{{\Psi}^{(m-1)}_n}
%\end{equation}
%t appears from the above equation that we need the ${(m-1)}^{th}$ order correction in wavefunction to get the ${m}^{th}$ order correction in %energy. However Wigner(R7) showed that only a ${m}^{th}$ order correction in wavefunction can give a ${2m+1}^{th}$ order correction in %energy.
%The wavefunction corrections are expressed as a linear combination of the eigenvectors of $\hat{H_o}$, as they form a complete space.
%\begin{equation}
%{\Psi}^{(m)}_n = \sum_l{C^{(m)}_{n,l}}{\Psi}^{(0)}_l 
%\end{equation}
%where, ${C^{(m)}_{n,l}} = \bra{{\Psi}^{(0)}_l}{\Psi}^{(m)}_n\rangle$ . The first order correction can be obtained by multiplying equation (17) on %left by $\bra{{\Psi}^{(0)}_l} $. 
%\begin{equation}
%{C^{(1)}_{n,l}} =\frac{\bra{{\Psi}^{(0)}_l}\hat{V}\ket{{\Psi}^{(0)}_n}}{({E}^{(0)}_n - {E}^{(0)}_l)}
%\end{equation}
%we can get the other higher order corrections as well using the same procedure as above.
%RSPT provides us with a framework for other perturbation based methods which differ only in the partitioning of the the electronic hamiltonian.
%\subsection{M\"oller-Plesset perturbation theory (MPPT)} MPPT named after
%Christian M\"oller and Milton S. Plesset\cite{MollerPlesset34} can be seen as a
%correction to the HF method by including electron correlation effects. It takes
%the zeroth order Hamiltonian $\hat{H_o}$ to be the sum of the one-electron Fock
%operators, and treats electron correlation as a perturbation.  \begin{equation}
%\hat{H_o} = \sum_if(i) = \sum_ih(i) + V^{HF} 
%\end{equation}
%\begin{equation}
%\hat{V} = \sum_{i>j} \frac{1}{r_{ij}} - V^{HF}   ,
%\end{equation}
%%where, ${V^{HF}_{pq}} = \sum_m\bra{pm}\ket{qm}$. 
%Thus, the zeroth order wavefunction is just the HF wavefunction and the zeroth
%order energy is the sum of orbital energies. The first order corrected energy
%is just the HF energy, \\ \centerline{$E_n = {E^{(0)}_n} + {E^{(1)}_n}= \sum_i\epsilon_i -
%\frac{1}{2}\sum_{ij}\bra{ij}\ket{ij} $.}\\
%%\begin{equation}
%%\end{equation}
%\\The wavefunction corrections can be expressed as a linear combination of
%eigenvectors of the Fock operator-slater determinants which form an
%orthogonal complete space. The first order correctionn to the wavefunction
%includes only the doubly excited determinants $\ket{{\Phi}^{ab}_{ij}}$, which can be
%obtained by replacing the occupied orbitals $i$ and $j$ in the HF wavefunction by virtual 
%orbitals $a$ and $b$ respectively. All
%other determinants do not contribute to the second order correction in energy
%because of Brillouin's theorem and the Slater Condon rules.  \begin{equation}
%{\Psi}^{(1)}_n = \sum_{i>j;a>b}C^{(1)}_{n,ijab} \Psi^{ab}_{ij} 
%\end{equation}
%and 
%\begin{equation}
%C^{(1)}_{n,ijab} = \bra{\Psi^{ab}_{ij} }{\Psi}^{(1)}_n\rangle = \sum_{i>j;a>b}\frac{\bra{\Psi^{ab}_{ij} }{\Psi}^{(0)}_n\rangle}{\epsilon_i + \epsilon_j - \epsilon_a - \epsilon_b}
%\end{equation}
%Inserting the above expression into the equation for the second order MPPT energy or MP2 energy (in the spin orbital basis) yields:
%\begin{equation}
%E^{(2)}_{0} = \sum_{i>j;a>b}\frac{{|\bra{\Phi^{(0)}_0}\frac{1}{r_{12}} } \ket{\Psi^{ab}_{ij}}^2|}{\epsilon_i + \epsilon_j - \epsilon_a - \epsilon_b}=\frac{1}{4}\sum_{ijab}\frac{{|\bra{ij}\ket{ab}|}^2}{\epsilon_i + \epsilon_j - \epsilon_a - \epsilon_b}
%\end{equation}
%\subsection{Configuration Interaction (CI)}
%CI is a matrix mechanics solution of the schr\"odinger's equation. For a
%N electron molecule, the wavefunction can be expressed as a linear combination
%of N electron basis functions. CI uses Slater determinants as the basis, since
%they are antisymmetric, orthogonal and  form a complete space.  \begin{equation}
%\ket{\Psi} = c_o\ket{\Phi_o} + \sum_{ia}{c}_{i}^{a}\ket{{\Phi}^a_i} + \sum_{i>j,a>b}c_{ij}^{ab}\ket{\Phi^{ab}_{ij}} + \sum_{i>j>k,a>b>c}c_{ijk}^{abc}\ket{\Phi^{abc}_{ijk}} + ...
%\end{equation}
%Now, just like the HF method, the variational method is invoked to minimize the
%expectation value of energy with respect to the coefficients and we get the
%final matrix eigenvalue equation: \begin{equation}
%HC = CE
%\end{equation}
%where H is the Hamiltonian matrix, $H_{ij} = \bra{\Psi_i}H\ket{\Psi_j}$ , C is
%the coefficient matrix and E is the diagonal matrix containing the energies,
%with the lowest energy being
%that of the ground state. In full CI all possible slater determinants are
%included, and we get an exact wavefunction in the basis set limit. However, Full
%CI is computationally very expensive and is not used except in benchmark
%calculations of small molecules. Instead we use truncated CI schemes like CIS and
%CISD, which only include singles and singles and doubles determinants,
%respectively. These methods can extract up to 95\% of correlation energy\cite{HarrisonHandy83} and
%can be applied to solve for excited states, open-shell systems and systems that
%are far from their equilibrium geometry, which makes them very useful. However,
%truncated CI methods are not size-extensive, which is a disadvantage.
%\subsection{Coupled Cluster Theory}
%Coupled cluster (CC) theory is one of the most accurate yet computationally
%affordable methods which is very widely used in quantum chemistry. The CC
%wavefunction \cite{Crawford00} incorporates the electron correlation effects through cluster operators 
%$\hat{T}$.  
%\begin{equation}
%\ket{\Psi_{CC}} = e^{\hat{T}}\ket{\Psi_o} , 
%\end{equation}
%\begin{equation}
% \hat{T} = \hat{T_1} + \hat{T_2} + \hat{T_3} + ... \;\hat {T_n} .
%\end{equation} 
%where $\hat{T_1}$ is the linear combination of all single excitation
%operators, which replace an occupied orbital in reference wavefunction (HF) with a
%virtual orbital. A singles excitation operator takes into account the relaxation of the one electron basis
%functions, i.e. HF molecular orbitals, when electronic motions are correlated. In
%second quantization\cite{JorgensenSimons81} formalism, $\hat{T_1}$ can be written as :
%\begin{equation}
% \hat{T_1} = \sum_{ia}t^a_i{a}^\dagger_a a_i
%\end{equation}
%In the above equation, ${a}^\dagger_a$ , the creation operator as it adds a particle (electron) to the SD.
%\begin{equation}
%{a}^\dagger_a\ket{\phi_b...\phi_d} = \ket{\phi_a\phi_b...\phi_d}
%\end{equation}
%and $a_i$ , which removes a particle from a SD is called the annihilation operator.
%\begin{equation}
%a_i\ket{\phi_i\phi_j...\phi_l} = \ket{\phi_j...\phi_l}
%\end{equation}
% The $\hat{T_2} $ operator is the linear combination of all double excitation operators. It correlates the motion of all pairs of electrons of the molecule and can be expressed in second quantization as:
%\begin{equation}
% \hat{T_2} = \frac{1}{4}\sum_{ijab}t^{ab}_{ij}a^\dagger_aa^\dagger_ba_ja_i
%\end{equation}
%Similarly, ${\hat{T_3}}$ correlates the motion of all triplets of electrons and so on. In general,
%\begin{equation}
%\hat{T_n} = {\frac{1}{n!}}^2\sum_{ij..ab..}^nt^{ab..}_{ij..}a^\dagger_aa^\dagger_b...a_ja_i
%\end{equation}
%\subsubsection{Coupled Cluster Energy}
%If we expand the exponential ansatz of the CC wavefunction,
%\begin{equation}
%\ket{\Psi_{CC}} = (1+(\hat{T_1} + \hat{T_2} + \hat{T_3} + ... ) + \frac{1}{2!}{(\hat{T_1} + \hat{T_2} + \hat{T_3} + ...)}^2 + ... )\ket{\Psi_o}
%\end{equation}
%we can see that we will get a linear combination of all possible slater determinants, which makes the Full CC wavefunction an exact wavefunction in the basis set limit, just like Full CI. 
%\begin{equation}
%\hat{H}e^{\hat{T}}\ket{\Psi_o} = Ee^{\hat{T}}\ket{\Psi_o}
%\end{equation} On multiplying the above equation by the inverse of the
%exponential operator ($e^{-\hat{T}}$) and projecting it against the HF wavefunction and excited determinants from the left, we get the energy and the amplitude
%equations of coupled cluster theory.
%\begin {equation}
%\bra{\Psi_o}e^{-\hat{T}}\hat{H}e^{\hat{T}}\ket{\Psi_o} = E
%\end{equation} 
%\begin{equation}
%\bra{\Psi^{ab..}_{ij..}}e^{-\hat{T}}\hat{H}e^{\hat{T}}\ket{\Psi_o} = E\cancelto{0}{\bra{\Psi^{ab..}_{ij..}}\Psi_o\rangle} = 0 .
%\end{equation}
%The similarity transformed hamiltonian, $e^{-\hat{T}}\hat{H}e^{\hat{T}}$, also written as $\bar{H}$, can be expressed in Hausdorff expansion \cite{Merzbacher70}as:
%\begin{equation}
%e^{-\hat{T}}\hat{H}e^{\hat{T}} = \hat{H} + \lbrack\hat{H},\hat{T}\rbrack + \frac{1}{2!}\lbrack\lbrack\hat{H},\hat{T}\rbrack,\hat{T}\rbrack + \frac{1}{3!}\lbrack\lbrack\lbrack\hat{H},\hat{T}\rbrack,\hat{T}\rbrack,\hat{T}\rbrack + \frac{1}{4!}\lbrack\lbrack\lbrack\lbrack\hat{H},\hat{T}\rbrack,\hat{T}\rbrack,\hat{T}\rbrack,\hat{T}\rbrack + ...
%\end{equation}
%Also, the second quantized form of the hamiltonian can be written as:
%\begin{equation}
%\hat{H} = \sum_{pq}h_{pq}a^\dagger_pa_q + \frac{1}{4}\sum_{pqrs}\bra{pq}\ket{rs}a^\dagger_pa^\dagger_qa_sa_r
%\end{equation}Using the above two equations and Wick's theorem\cite{Wick50}, the coupled cluster energy equation can be written simply as:
%\begin{equation}
%E_{CC} = E_o + \sum_{ia}f_{ia}t^a_i + \frac{1}{4}\sum_{aibj}\bra{ij}\ket{ab}t^{ab}_{ij} + \frac{1}{2}\sum_{aibj}\bra{ij}\ket{ab}t^a_it^b_j
%\end{equation}
%where the amplitudes $t^a_i$ and $t^{ab}_{ij}$ can be obtained by solving the
%amplitude equations.\\Since full CC is computationally impractical, truncated CC
%methods like CCSD, CCSD(T), CCSDT and CCSDTQ are used.\\ \centerline{CCSD: $\hat{T} =
%\hat{T_1} + \hat{T_2} $ ,   CCSDT: $\hat{T} = \hat{T_1} + \hat{T_2} + \hat{T_3}$
%,   CCSDTQ: $\hat{T} = \hat{T_1} + \hat{T_2} + \hat{T_3} + \hat{T_4}.$}\\The CCSD(T)
%method includes an approximate treatment of triples using perturbation theory.\\
%Using eq. 28, the CCSD amplitude equations can be obtained.

%\section{Introduction}
%Theoretical and computational chemistry has grown by leaps and bounds over the
%past several decades. Ab initio quantum chemical methods can now accurately
%predict a variety of molecular properties. More recently, these methods have
%been extended to calculate chiroptical
%properties like optical rotation. Theoretical calculations of optical rotation
%can be very useful in solving experimental chemical problems.  Specifically, they
%can provide a computational tool to modern organic chemists to determine
%the absolute stereochemical configuration of a chiral compound, thus alleviating the experimental difficulties
%involved.\\ Rosenfeld laid the theoretical foundations for calculating optical
%roatation in 1929\cite{Rosenfeld29}. He showed that in the presence of an
%external field, the induced dipole moment in a molecule can be written as:
%\begin{equation}\langle\vec{\mu}\rangle = \alpha\vec{E} + \frac{1}{\omega}\textbf{G}^\prime\frac{\partial\vec{B}}{\partial t}
%\end{equation} where $\vec{E} $ and $\vec{B}$ are the time dependent electric and
%magnetic field vectors respectively, $\omega$ is the frequency of the filed, $\alpha$ is the electric dipole
%polarizability tensor and the $\textbf{G}^\prime$ tensor, also known as the Rosenfeld
%tensor contains the recipe for calculating optical rotation.
%\begin{equation}
%\textbf{G}^{\prime}(\omega) = -\frac{1}{\hbar} Im\sum_{n \neq 0}\left[\frac{\bra
%{\psi_o}\vec{\mu}\ket{\psi_n}\bra{\psi_n}\vec{m}\ket{\psi_o}}{\omega_{no} - \omega - i\Gamma_{no}} + \frac{\bra
%{\psi_o}\vec{m}\ket{\psi_n}\bra{\psi_n}\vec{\mu}\ket{\psi_o}}{\omega_{no} + \omega + i\Gamma_{no}} \right]
%\end{equation}
%Here, $\vec{\mu}$ and $\vec{m}$ are electric and magnetic dipole operators,
%%: $\mu =
%%\sum_ir_i,\;\; m= \sum_ir_i\times p_i$ (in atomic units).  
% $\omega$ is the frequency of light, $\omega_{no}$ is the excitation energy of
%the state $\psi_n$ and $\Gamma_{no}$ is the dephasing rate between the states
%$\psi_o$ and $\psi_n$.
%% taken to be zero when the field frequency is far from
%%resonance
%\cite{CrawfordTamJPA07}. $Im$ means the imaginary part of the equation and the
%summation runs over all the excited states $\psi_n$. The trace of this tensor
%is related to the specific rotation, usually denoted as
%${\lbrack\alpha\rbrack}_\omega$ in
%deg dm$^{\text{-1}}$(g/mL)$^{\text{-1}}$
%%After averaging over all possible
%%orientations of the molecule,the following expression is obtained
%\cite{Crawford06}.
%\begin{equation}
%{\lbrack\alpha\rbrack}_{\omega} = \frac{(72.0 \times 10^6){\hbar}^2 N_A\;\omega}{c^2{m_e}^2 M} \times \left[ \frac{1}{3}Tr(\textbf{G}^\prime)\right]
%\end{equation}
%Here, $\textbf{G}^\prime$ and $\omega$ are in atomic units, c is the speed of light (m/s), m$_{\text{e}}$ is the mass of
%electron (kg), M is the molecular mass (amu) and $N_A$ is Avogadro's
%number. Calculating the $\textbf{G}^\prime$ tensor using equation (2) requires explicit calculations of a large
%number of excited states, which is very computationally expensive. Most ab initio methods
%use the linear response formalism\cite{Koch90,Kobayashi94} instead to calculate
%${\lbrack\alpha\rbrack}_\omega$.
%\begin{equation}
%\textbf{G}^{\prime}(\omega) = Im\langle\langle\mu;m\rangle\rangle
%\end{equation} The linear response approach focuses on the perturbation of the
%ground state wavefunction in the presence of an external field, avoiding
%excited state calculations. Different levels of theory have been employed
%over the years for calculating optical rotation. Polavarapu was the first to
%calculate ab-initio optical rotation of gas phase molecules using the time
%dependent Hartree Fock method\cite{Polavarapu96}. The signs of the calculated optical
%rotation values matched with that of the experiment for most of his structures,
%while their magnitudes differed generally by a factor of two. Inspired by
%Polavarapu's success, Cheeseman et al. \cite{Cheeseman00,Stephens01} included correlation
%effects in the calculations by applying density functional theory.
%Using large basis sets with diffuse functions like aug-cc-pVDZ and aug-cc-pVTZ,\cite{Dunning89} they were able to match the experimental values very closely, with a deviation of 20-25 degrees
%for a set of 28 chiral molecules. Ruud et al. extended the calculations to the
%coupled cluster level using the coupled cluster response
%theory (CCLR) developed by Koch and J{\o}rgensen\cite{Koch90},
%and obtained promising results\cite{Ruud03}. These initial breakthroughs have allowed optical
%rotation calculations to be performed on a large number of chiral molecules over the years.  As
%such, the gas phase calculations have become very reliable now and are being
%increasingly used to determine absolute configurations of many chiral molecules.\cite{Kondru99}
%\\However, since most of the experimental measurements of optical rotation are
%performed in solutions, we need to be able to calculate this property for solutions
%as well. But a larger number of molecules and stronger intermolecular interactions
%makes the modelling of solution phase optical rotation much more challenging and
%complicated than that of the gas phase. In recent years, the ab inito methods
%mentioned above have been combined with various solvation based
%approaches\cite{Neugebauer05,Neugebauer09,Mennucci02,Tomasi05,JensenGordon96}
%to calculate the optical rotation in solution. This review condiders some of
%the popular approaches that are being used and describes their working
%mechanisms and performances in detail. In the current/future work section of
%this review, some reduced scaling techniques are proposed with the aim of
%making these methods more efficient and computationally practical. Also, some
%of the essential, underlying coupled cluster and response theory used for
%optical rotation calculations are presented and discussed.
%for the sake of completeness, I also talk about the copuled cluster theory and
%derive the working equations of the CCLR theory, which our group uses for
%calculating optical rotation.


\section{Theoretical tools}
%\subsection{Schrodinger's equation}
The most basic problem in quantum mechanics is to solve the time independent
non-relativistic Schr\"odinger equation \cite{Schrodinger26}.
% \begin{equation}
%-\frac{1}{2}\nabla^2\psi_n(r) + V(r)\psi_n(r) = E_n\psi_n(r) , \end{equation}
%Solving this eigen value equation gives us the wavefunction $\psi_n$, and $E_n$, the energy of the system.
It can be solved exactly for one electron hydrogen-like species and only approximate
solutions can be obtained for many electron systems.
%\subsection{Born-Oppenheimer(BO) Approximation}
%BO Approximation was proposed by Max Born and J.Robert
%Oppenheimer\cite{Born27}.It states that the wavefunction of a molecule can be
%broken down into it's electronic and nuclear components. $\psi_{total} =
%\psi_{electronic}*\psi_{nuclear}$.The rationale behind this approximation is
%the lack of coupling between the electronic and nuclear degrees of freedom in a
%molecule.Since, the nuclei are a lot more heavier compared to electrons, they
%can be assumed to be stationary while solving the electronic schrodinger
%equation.So, the electronic hamiltonian for a system of N electrons and M
%nuclei can be written as : \begin{equation}
%\hat{H}_{elec} = -\frac{1}{2}\sum\limits_{i=1}^N\nabla_{i}^2-\sum\limits_{i=1}^N\sum\limits_{A=1}^M\frac{Z_A}{r_{iA}}+\sum\limits_{i=1}^N\sum\limits_{j>i}^N\frac{1}{r_{ij}},
%\end{equation}
%where the symbols have their usual meanings.Thus, the electronic wavefunction and hence the electronic energy depends parametrically on the nuclear coordinates and can be written as:
%$\psi_{elec} = \psi_{elec}(\left\{{r_i}\right\};\left\{{R_A}\right\})$ , $E_{elec} = E_{elec}\left\{R_A\right\}.$
%The total energy of the fixed nuclei system can then be written as: 
%\begin{equation}
% E_{tot} = E_{elec} + \sum\limits_{A=1}^M\sum\limits_{B>A}^M \frac{Z_AZ_B}{R_{AB}}
%\end{equation}
%Only because of this approximation, can we think of Potential energy surfaces of molecules.
%\subsection{Hartree-Fock Approximation}
Under the Born-Oppenheimer approximation\cite{Born27}, the Hartree-Fock (HF) theory\cite{SzaboOstlund} is one of the simplest
approximation methods. The HF wavefunction is a single slater determinant of spin orbitals which
naturally obeys the Pauli's antisymmetry principle.
%It assumes that the exact wavefunction of a N electron
%system can be approximated by a single slater determinant(SD)of N-spin
%orbitals\cite{SzaboOstlund}.For a two electron system, the SD looks like : \begin{equation}
%\Psi = (2!)^{-1/2}
%\begin{vmatrix}
%\chi_1(x_1) & \chi_2(x_1)  \\
%\chi_1(x_2) & \chi_2(x_2)  \\
%%. & . & \; & .\\
%%\chi_1(x_N) & \chi_2(x_N) & ... &\chi_N(x_N) \\
% \end{vmatrix}
%\end{equation}
Since the energy expectation value of a slater determinant is a functional of
the spin-orbitals, optimized spin orbitals can be obtained by minimizing this
functional with respect to the orbitals by invoking the variational principle.
This minimization procedure is usually coupled with the constraints of an
orthonormal basis using the Lagrangian method of undetermined multipliers.
Finally, a set of non-linear equations are solved through a self consistent
procedure, because of which HF is also called self consistent field theory
(SCF). The HF method recovers up to 99\% of the electronic energy but fails to take
into account the electron correlation effects which constitute the remaining
1\% and are very important for accurate calculations of properties. 
However, due to its simplicity it is usually the starting point for other
correlation methods like the coupled cluster method. \\\\
%\subsection{Coupled cluster theory} 
\paragraph{Coupled cluster theory} ~ \\\\
Coupled cluster\cite{Crawford00} (CC) theory is a very accurate electronic
structure method that has been widely used in quantum chemistry.
It takes into account the electron correlation effects by expressing the
wavefunction as a linear combination of slater determinants within an
exponential framework. This is achieved by the use of cluster operators $\hat{T}$ 
acting on the HF wavefunction.
%is one of the most accurate yet computationally
%affordable method, which is very widely used in quantum chemistry.CC
%wavefunction is an exponential ansatz\cite{Crawford00} which incorporates the electron
%correlation effects through cluster operators 
%$\hat{T}$.  
\begin{equation}
\ket{\Psi_{CC}} = e^{\hat{T}}\ket{\Psi_{o}} , 
\end{equation}
where
\begin{equation}
 \hat{T} = \hat{T_1} + \hat{T_2} + \hat{T_3} + ... \;\hat {T_n} 
\end{equation}
$\ket{\Psi_{o}}$ is the reference wavefunction, usually taken as the HF
wavefunction, and the $\hat{T_1}$ operator is the linear combination of all single excitation
operators, which replace an occupied orbital in the HF wavefunction with a
virtual orbital. It takes into account the relaxation of the one electron basis
functions, i.e HF molecular orbitals, due to the correlation of the electronic
motions. In second quantization\cite{JorgensenSimons81} formalism, $\hat{T_1}$
can be written as :
\begin{equation}
\hat{T_1} = \sum_{ia}t^a_i\{{a}^\dagger_a a_i\}
\end{equation}
where the indices i, j, k ... refer to occupied orbitals while
the indices a, b, c ... refer to virtual spin orbitals.
In the above equation, ${a}^\dagger_a$ is called the creation operator as it
creates a new particle state (virtual orbital) when it acts on a slater determinant.
\begin{equation}
{a}^\dagger_a\ket{\phi_b...\phi_d} = \ket{\phi_a\phi_b...\phi_d}
\end{equation}
$a_i$ is the hermitian conjugate of the creation operator and is called
the annihilation operator as it removes a hole state (occupied orbital) when it acts on a
slater determinant.
\begin{equation}
a_i\ket{\phi_i\phi_j...\phi_l} = \ket{\phi_j...\phi_l}
\end{equation}
The $\hat{T_2} $ operator is the linear combination of all double excitation operators.
It correlates the motion of all pairs of electrons of the molecule and can be expressed
in second quantization as:
\begin{equation}
\hat{T_2} = \frac{1}{4}\sum_{ijab}t^{ab}_{ij}\{a^\dagger_aa^\dagger_ba_ja_i\}
\end{equation}
Similarly, the ${\hat{T_3}}$ operator correlates the motion of all triplets of
electrons and so on. In general, \begin{equation}
\hat{T_n} = {\frac{1}{n!}}^2\sum_{ij..ab..}^nt^{ab..}_{ij..}\{a^\dagger_aa^\dagger_b...a_ja_i\}
\end{equation}
%\subsubsection{Coupled Cluster energy}
%\paragraph{Coupled Cluster energy}~ \\
If we expand the exponential ansatz of the CC wavefunction,
\begin{equation}
\ket{\Psi_{CC}} = (1+(\hat{T_1} + \hat{T_2} + \hat{T_3} + ... ) + \frac{1}{2!}{(\hat{T_1} + \hat{T_2} + \hat{T_3} + ...)}^2 + ... )\ket{\Psi_o}
\end{equation}
we get a linear combination of all possible Slater determinants, which makes CC
wavefunction an exact wavefunction in the basis set limit, just like full
configuration interaction. Thus, the full CC
wavefunction is a solution of the Schr\"odinger equation.
\begin{equation}
\hat{H}e^{\hat{T}}\ket{\Psi_o} = E_{cc}\;e^{\hat{T}}\ket{\Psi_o}
\end{equation}
On multiplying the above equation by the inverse of the exponential operator i.e
$e^{-\hat{T}}$ and projecting it against the reference and excited
determinants, we get the energy and the amplitude equations of coupled cluster
theory.
\begin {equation}
\bra{\Psi_o}e^{-\hat{T}}\hat{H}e^{\hat{T}}\ket{\Psi_o} = E_{cc}
\end{equation}
\begin{equation}
%\bra{\Psi^{ab..}_{ij..}}e^{-\hat{T}}\hat{H}e^{\hat{T}}\ket{\Psi_o} = E\cancelto{0}{\bra{\Psi^{ab..}_{ij..}}\Psi_o\rangle} = 0 .
\bra{\Psi^{a.}_{i.}}e^{-\hat{T}}\hat{H}e^{\hat{T}}\ket{\Psi_o} = 0 .
\end{equation} 
Here $\Psi^{a.}_{i.}$ can refer to any excited Slater determinant singles, doubles etc. 
The similarity transformed Hamiltonian $e^{-\hat{T}}\hat{H}e^{\hat{T}}$ also
written as $\bar{H}$ can be expressed in the Hausdorf expansion
\cite{Merzbacher70}as:
\begin{equation}
e^{-\hat{T}}\hat{H}e^{\hat{T}} = \hat{H} + \lbrack\hat{H},\hat{T}\rbrack + \frac{1}{2!}\lbrack\lbrack\hat{H},\hat{T}\rbrack,\hat{T}\rbrack + \frac{1}{3!}\lbrack\lbrack\lbrack\hat{H},\hat{T}\rbrack,\hat{T}\rbrack,\hat{T}\rbrack + \frac{1}{4!}\lbrack\lbrack\lbrack\lbrack\hat{H},\hat{T}\rbrack,\hat{T}\rbrack,\hat{T}\rbrack,\hat{T}\rbrack + ...
\end{equation}
Also, the second quantized form of the Hamiltonian can be written as\cite{Crawford00}:
\begin{equation}
\hat{H} = \sum_{pq}h_{pq}\{a^\dagger_pa_q\} + \frac{1}{4}\sum_{pqrs}\bra{pq}\ket{rs}\{a^\dagger_pa^\dagger_qa_sa_r\}
\end{equation}
Using the above two equations and the Wick's theorem\cite{Wick50}, the coupled cluster energy
equation gets simplified as:
\begin{equation}
E_{cc} = E_o + \sum_{ia}f_{ia}t^a_i + \frac{1}{4}\sum_{aibj}\bra{ij}\ket{ab}t^{ab}_{ij} + \frac{1}{2}\sum_{aibj}\bra{ij}\ket{ab}t^a_it^b_j
\end{equation} The amplitudes $t^a_i$ and $t^{ab}_{ij}$ can be obtained by
solving the coupled cluster amplitude equations. However, Full CC is
computationally impractical and truncated CC methods like CCSD:\;$\hat{T} =
\hat{T_1} + \hat{T_2}$, CCSDT:\;$\hat{T} = \hat{T_1} + \hat{T_2} + \hat{T_3}$ 
are used. The CCSD(T)\cite{Shen12} method, which treats the triples approximately using
perturbation theory is also a popular method.
%is also very popular and uses an approximate 
%treatment of triples using perturbation theory.
%CCSD: $\hat{T} = \hat{T_1} + \hat{T_2} $ , CCSDT: $\hat{T} = \hat{T_1} + \hat{T_2} + \hat{T_3}$
%, CCSDTQ: $\hat{T} = \hat{T_1} + \hat{T_2} + \hat{T_3} + \hat{T_4}$.
%\\%CCSD(T) method uses an approximate treatment of triples using perturbation theory.
%For the CCSD method the amplitude equations are:
%\begin{equation}
%\bra{\Psi^{a}_{i}}e^{-\hat{T}}\hat{H}e^{\hat{T}}\ket{\Psi_o} = 0
%\end{equation}
%\begin{equation}
%\bra{\Psi^{ab}_{ij}}e^{-\hat{T}}\hat{H}e^{\hat{T}}\ket{\Psi_o} = 0
%\end{equation}
%CC truncation methods are more effective compared to that of the
%CI. For example, although CCSD has the same number of coefficients
%($t^{ab}_{ij}$ and $t^a_i$) as CISD ($c^{ab}_{ij}$ and $c^a_i$), CCSD
%implicitly includes higher excitations than CISD because of the products of
%cluster operators like $\hat{T_1}\hat{T_2}$, ${(\hat{T_2})}^2$ which capture
%the triple and quadruple excitation contributions to the singles and doubles
%amplitudes.
%The exponential structure of the CC wavefunction makes more advantageous than the 
%linear CI method. Because of this, truncated CC methods like CCSD are size extensive, which means they
The exponential structure of the wavefunction makes the truncated CC methods more efficient 
compared to that of the linear CI methods. CCSD wavefunction implicitly includes the triples
and quadruples excitation contributions to its singles and doubles amplitude equations because 
of the products of cluster operators like $\hat{T_1}\hat{T_2}$, ${(\hat{T_2})}^2$ etc., unlike 
the CISD method which can only include singles and doubles excitation contributions. 
Another advantage of this exponential ansatz is the property of size consistency\cite{Crawford00}.
It means that the sum of the CC energies of non-interacting fragments (each calculated 
separately) is equal to the energy of the supermolecule when all fragments are included 
in the calculations. CC methods give very accurate results for molecules at their equilibrium 
geometries\cite{Bartlett}. CCSD(T) is often referred as the "gold standard" method of electronic structure 
theory. 
%Also, one of the major advantages with the CC truncation methods is their
%size-extensivity which makes them very useful. 
However, CC methods are computationally expensiver: CCSD scales as $O(N^6)$, CCSD(T) as
$O(N^7)$, CCSDT as $O(N^8)$ and so on, where $N$ is the number of basis functions.\\\\
%and is routinely used for accurate results only for small molecules. While CCSD(T)
%gives very good results for molecules at their equilibrium geometry, it fails
%to describe diradical species and bond-breaking. CCSDT and CCSDTQ methods are
%used virtually exclusively for high accuracy calculations of small molecules
%as they are very computationally expensive.
%\paragrpah{\bf{CC analytic derivatives}} ~\\\\
\paragraph{CC analytic derivatives} ~ \\\\
Molecular properties like dipole
moments, IR intensities, force constants etc., depend upon the gradients of the
molecular energy with respect to external perturbations. 
In this section, a derivation of first order gradient expressions of the CC method is presented.\\
%If we take the derivative of the CC energy directly with respect to any perturbation 
%X,
%, we get
%\begin{equation}
%\frac{\partial{E_{cc}}}{\partial X} = \bra{\phi_0}\frac{\partial{\bar H}}{\partial X}\ket{\phi_0},
%\end{equation}
%If we use the above equation for calculating gradients, 
%we need to calculate
%$\frac{\partial{t^a_i}}{\partial X}, \frac{\partial{t^{ab}_{ij}}}{\partial X}$
%, we need to calculate the derivatives of non-linear amplitude equations, 
%which makes it very computationally expensive. As a result, a different 
%approach is used where one needs to solve some linear equations i.e. 
%the Lambda equations, which are independent of the perturbation. 
%In this section, a derivation of first order gradient expressions 
%of CC method is presented using this approach.\\
The gradient of the CC energy with respect to any external perturbation 
X is written as:
\begin{equation}
\frac{\partial{E_{cc}}}{\partial X} = \bra{\Psi_o}\frac{\partial{\bar H}}{\partial X}\ket{\Psi_o} = \bra{\Psi_o}{\bar{H}}^X + \lbrack\bar H , \frac{\partial{\hat T}}{\partial X}\rbrack\ket{\Psi_o} 
\end{equation}
where, $ {\bar{H}}^X =  e^{-\hat{T}}\frac{\partial{\hat H}}{\partial X}e^{\hat{T}}$ contains the 
derivatives of the basis functions, MO coefficients etc. Invoking the 
resolution of identity (RI),
\begin{equation}
 1 = \ket{\Psi_o}\bra{\Psi_o} + \sum_{ia}\ket{{\Psi}^a_i}\bra{{\Psi}^a_i} + \frac{1}{4}\sum_{ijab}\ket{{\Psi}^{ab}_{ij}}\bra{{\Psi}^{ab}_{ij}} + ...
\end{equation}
equation (19) can be expressed as:
\begin{equation}
\frac{\partial{E_{cc}}}{\partial X} = \bra{\Psi_o}{\bar{H}}^X\ket{\Psi_o} + \sum_{i.a.}\bra{\Psi_o}\bar{H}\ket{{\Psi}^{a.}_{i.}}\bra{{\Psi}^{a.}_{i.}}\frac{\partial{\hat T}}{\partial X}\ket{\Psi_o}.
\end{equation}
The second term of the above equation involves calculating derivatives of
amplitudes i.e. $\bra{{\Psi}^{a}_{i}}\frac{\partial{\hat T}}{\partial
X}\ket{\Psi_o} = \frac{\partial{t^a_i}}{\partial X}$, which can be very
computationally expensive if calculated directly. As a result, we would like to
recast the above equation in something which is computationally practical. We
start with the CC amplitude equations, say singles equation and take
its derivative with respect to X.
\begin{equation} 
0 = \bra{{\Psi}^a_i}{\bar{H}}^X + \lbrack\bar H , \frac{\partial{\hat T}}{\partial
X}\rbrack\ket{\Psi_o} 
\end{equation} 
Using RI as before, we can write the above equation in the folllowing form:
\begin{equation}
\sum_{j.b.}\bra{{\Psi}^a_i}\bar{H}-E_{cc}\ket{{\Psi}^{b.}_{j.}}\bra{{\Psi}^{b.}_{j.}}\frac{\partial{\hat T}}{\partial X}\ket{\Psi_o} = - \bra{{\Psi}^a_i}{\bar{H}}^X\ket{\Psi_o} 
\end{equation}
or,
\begin{equation} \bra{{\Psi}^{b.}_{j.}}\frac{\partial{\hat T}}{\partial
X}\ket{\Psi_o} = -\sum_{ia}
\bra{{\Psi}^{b.}_{j.}}{(\bar{H}-E_{cc})}^{-1}\ket{{\Psi}^a_i}\bra{{\Psi}^a_i}{\bar{H}}^X\ket{\Psi_o}
\end{equation}
The above equation was derived by considering the gradient of just the singles equation.
Including the gradient of all the other amplitude equations and plugging the modified 
equation into the expression for the gradient of energy eq.(21), we obtain:
\begin{equation}
\frac{\partial{E_{cc}}}{\partial X} = \bra{\Psi_o}{\bar{H}}^X\ket{\Psi_o} -
\sum_{i.a.}\bra{\Psi_o}\bar{H}\ket{{\Psi}^{a.}_{i.}}\sum_{j.b.}
\bra{{\Psi}^{a.}_{i.}}{(\bar{H}-E_{cc})}^{-1}\ket{{\Psi}^{b.}_{j.}}
\bra{{\Psi}^{b.}_{j.}}{\bar{H}}^X\ket{\Psi_o}
\end{equation}
We define a perturbation independent $\Lambda$ operator such that:
\begin{equation}
\bra{\Psi_o}\Lambda\ket{\Psi^{b.}_{j.}} = -\sum_{i.a.}\bra{\Psi_o}\bar{H}\ket{{\Psi}^{a.}_{i.}}
\bra{{\Psi}^{a.}_{i.}}{(\bar{H}-E_{cc})}^{-1}\ket{{\Psi}^{b.}_{j.}}.
\end{equation}
This $\Lambda$ operator can be seen as a de-excitation operator and can be written as:
\begin{equation}
\Lambda = \Lambda_1 + \Lambda_2 + \Lambda_3 + ...
\end{equation} 
where $\Lambda_1 = \sum_{ia}\lambda^i_a\{{a}^\dagger_i a_a\}$ is the singles de-excitation
operator, $\Lambda_2$ is the doubles de-excitation operator and so on.  
Using the lambda operator, the gradient expression gets simplified as:
\begin{equation}
\frac{\partial{E_{cc}}}{\partial X} = \bra{\Psi_o}(1 + \Lambda){\bar{H}}^X\ket{\Psi_o} = \bra{\Psi_o}(1 + \Lambda)e^{-\hat{T}}\frac{\partial{\hat H}}{\partial X}\ket{\Psi_{cc}}
\end{equation}
The governing equation for calculating the lambda amplitudes eq. (26) can be written
in a more compact form:
\begin{equation}
\bra{\Psi_o}(1 + \Lambda)(\bar{H} - E_{cc})\ket{{\Psi}^{a.}_{i.}} = 0
\end{equation}
Thus instead of taking the gradients of the non-linear $T$ amplitude
equations with respect to perturbations, we solve the linear perturbation-independent 
$\Lambda$ equations for calculating the energy gradient.\\
Also, $\bra{\Psi_o}(1 + \Lambda)e^{-\hat{T}}$ is the CC left hand wavefunction and 
eq. (28) is a generalized Hellman-Feynman equation\cite{Feynman39}. For full CC the left and 
right hand wavefunctions are Hermitian conjugates of each other, but this is not true for 
the truncated CC methods. This is because of the non-hermiticity of the $\bar{H}$ operator.
However, for second order derivatives, we do need to calculate the first derivatives of 
either the $T$ or $\Lambda$ amplitudes with respect to a perturbation.
%The derivative of the similarity transformed hamiltonian can be expressed as:
%\begin{equation}
%\frac{\partial{\bar{H}}}{\partial X} = {\bar{H}}^X + \lbrack\bar H , \frac{\partial{\hat T}}{\partial X}\rbrack, 
%\end{equation}

%\\By combining the CC energy and amplitude equations, we can write:
%\begin{equation}
%E_{CC} = \bra{\Lambda}\hat{H}\ket{\Psi_{CC}}
%\end{equation}
%where
%\begin{equation}
%\bra{\Lambda} = \bra{\Psi_o} + \sum_{\mu}{\zeta_{\mu}}\bra{\mu}e^{-\hat{T}}
%\end{equation} In the above equation, $\bra{\mu}$ represents any excited Slater determinant.
%$\zeta_{\mu}$ can be obtained if the bra state (Lambda) obeys the schr\"odinger's
%equation.
%\begin{equation}
%\bra{\Lambda}\hat{H}e^{\hat{T}} = \bra{\Lambda}e^{\hat{T}}E_{cc}
%\end{equation}
%Right projecting the above equation onto the subspace \{$\ket{\nu}$\} (excited Slater determinants):
%\begin{equation}
%\sum_{\mu}\zeta_{\mu}A_{\mu\nu} = -\bra{\Psi_o}\lbrack\hat{H},\hat{\tau_{\nu}}\rbrack\ket{\Psi_{cc}}
%\end{equation}
%where $\hat{\tau_{\nu}}$ is an excitation operator:$\;\;\;\hat{\tau_{\nu}}\ket{\Psi_o} = \ket{\nu}$ , and 
%\begin{equation}
%A_{\mu\nu} = \bra{\mu}e^{-\hat{T}}\lbrack\hat{H},\hat{\tau_{\nu}}\rbrack\ket{\Psi_{cc}}
%\end{equation}
%We can solve for the parameters $\zeta_{\mu}$ from the above equation to get $\bra{\Lambda}.$ In the presence of a time-independent perturbation described by $\alpha \hat{V}$ with $\alpha$ as the strength parameter, the gradient of coupled cluster energy with respect to $\alpha$ at zero perturbation strength can be shown as:
%\begin{equation}
%\frac{\partial{E_{cc}}}{\partial {\alpha}}|_{\alpha=0} \;\;= \frac{\partial}{\partial{\alpha}}\bra{\Lambda (\alpha)}\hat{H}+\alpha \hat{V}\ket{\Psi_{CC}(\alpha)}|_{\alpha=0}\;\; = \;\;\;\bra{\Lambda}\hat{V}\ket{\Psi_{CC}}
%\end{equation}
%We are able to obtain the first order gradients of CC just by solving the linear
%lambda equations to get the $\zeta_{\mu}$ parameters and plugging them in the
%above expression. The $\bra{\Lambda}$ state is nothing the left hand coupled cluster wave
%function, as it is a solution to the Schr\"odinger equation and the above equation satisfies the 
%generalized Hellmann-Feynman theorem\cite{Feynman39}.
%%For full CC, $\bra{\Lambda}$ and $\ket{\Psi_{cc}}$ are just hermitian
%%conjugate, but they are different in case of truncated CC methods because %of
%%the non-hermiticity of $\bar{H}$ operator.
%\subsubsection{Coupled Cluster Linear Response (CCLR)}
%\paragraph{Exact states}~\\
\paragraph{Coupled cluster linear response (CCLR)}~\\\\
CCLR method proposed by Koch and J{\o}rgensen in 1990\cite{Koch90} is a recipe for accurate
calculations of response properties like dynamic polarizabilities, optical
rotations, etc. In this section, we will outline important steps for the
derivations of both general and CCLR response functions and also talk about
different gauge representations used for calculating the response functions.\\
The operator $V(t)$ which describes the interaction between the molecule and 
an external time-dependent field can be expressed in the frequency domain as: 
\begin{equation}
V(t) = \int_{-\infty}^{\infty}d\omega\;\;V(\omega) e^{(-i\omega + \Gamma)t} ,
\end{equation}
The full Hamiltonian can then be written as: $H = H_o + V(t)$, where $H_o$ is the
time independent unperturbed Hamiltonian. Assuming that $\Psi_o$ is an eigenstate of $H_o$
%An exact wavefunction satisfies the Schr\"odinger equation: $H_o\ket{O} =
%E_o\ket{O}$ where $H_o$ is the time independent Hamiltonian and $\ket{O}$ is an
%eigenstate of the system. 
and that the molecule is in state $\Psi_o$ when the perturbation starts at $t = -\infty$, 
the $\ket{\Psi_o}$ state evolves in time as $\ket{\Psi_o(t)}$ according to the time-dependent 
Schr\"odinger equation.
\begin{equation}
i\frac{d}{dt}\ket{\Psi_o(t)} = (H_o + V^t)\ket{\Psi_o(t)}
\end{equation}
%It can be shown that the time dependent state $\ket{\Psi_o(t)}$ is related to 
Following the work of Olsen\cite{Olsen85}, the time dependent state $\ket{\Psi_o(t)}$
can be written as:
\begin{equation}
\ket{\Psi_o(t)} = \ket{\bar{\Psi_o}}e^{i\epsilon(t)}
\end{equation}
where $\epsilon$ is a phase factor and the state $\ket{\bar{\Psi_o}}$ can be expressed as\cite{Koch90}:
%using perturbation theory. Specifically,
\begin{equation}
\ket{\bar{\Psi_o}} = \ket{\Psi_o} + {\ket{\Psi_o}}^{(1)}+ {\ket{\Psi_o}}^{(2)} + ...
\end{equation}
where first order correction ${\ket{\Psi_o}}^{(1)}$ and others can be determined 
from the time dependent perturbation theory. For calculating any response property, the
expectation value of the respective time independent  operator is calculated in
the presence of an external field. The response functions can be seen as the
coefficients of terms which appear in different orders of perturbation in $V(t)$
in the expansion of the expectation value.\cite{Koch90} 
%Furthermore, the expectation value of any time dependent operator $A$ can
%expanded as\cite{Koch90} : 
\begin{equation}
\bra{\Psi_o(t)}A\ket{\Psi_o(t)} = \bra{\Psi_o}A\ket{\Psi_o} + \int_{-\infty}^{\infty}d\omega_1{\langle\langle A;V({\omega_1})\rangle\rangle}_{\omega_1 + i\Gamma}e^{(-i\omega_1 + \Gamma)t} + .....
\end{equation}
where we only considered the linear response function $\langle\langle A;V({\omega_1})\rangle\rangle$ 
and neglected the higher order terms. If the perturbation field is composed of a 
single frequency ($\omega_1$) , the linear response function can be written as:
\begin{equation}
\langle\langle A;V({\omega_1})\rangle\rangle = \sum_k\{
\frac{\bra{\Psi_o}A\ket{\Psi_k}\bra{\Psi_k}V({\omega_1})\ket{\Psi_o}}{\omega_1
- \omega_k + i\Gamma}  -
\frac{\bra{\Psi_o}V({\omega_1})\ket{\Psi_k}\bra{\Psi_k}A\ket{\Psi_o}}{\omega_1 + \omega_k +
i\Gamma}\} \end{equation}
%The above equation is called the sum of states (SOS) equation , 
where $\omega_k$ is the excitation energy between the states $\Psi_o$ and $\Psi_k$ 
and the summation runs over all the solutions of the time independent Schr\"odinger equation $\Psi_k$.
It can be seen from eq.(2) that $\textbf{G}^{\prime}(\omega)$ is nothing but the imaginary part of this 
response function if we take $A$ as the dipole moment operator with $V$ as a magnetic field.
However, this approach is very computationally expensive for CC methods and we approximate the above 
linear response function using CCLR theory.
%\paragraph{CC linear response function}~\\\\
Using the same approach as above, the right and left hand CC wavefunction evolve in time as\cite{Koch90}:
\begin{equation}
\ket{\Psi_{cc}(t)} = e^{\hat{T}(t)}\ket{\Psi_o}e^{i\epsilon(t)}
\end{equation}
\begin{equation} \bra{\Lambda(t)} = \{\bra{\Psi_o} +
\sum_{i.a.}\lambda^{a.}_{i.}(t)\bra{\Psi^{a.}_{i.}}e^{-\hat{T}(t)}\}e^{-i\epsilon(t)}
\end{equation}
where $e^{\pm i\epsilon(t)}$ is a time dependent phase factor
and we have time dependent $T$ and $\lambda^{a.}_{i.}$ amplitudes. The governing equation for 
the time dependence of these amplitudes is the time-dependent Schr\"odinger equation.
\begin{equation}
i\frac{d}{dt}\ket{\Psi_{cc}(t)} = (H_o + V(t))\ket{\Psi_{cc}(t)} 
\end{equation}
\begin{equation}
\frac{d}{dt}\bra{\Lambda(t)} = i\bra{\Lambda(t)}(H_o + V(t))
\end{equation}
On multiplying eq. (38) by $e^{-\hat{T}(t)}$ on both sides and projecting 
it against $\bra{\Psi^{a.}_{i.}}$, we obtain the expression for the time evolution 
of $t^{a.}_{i.}$ amplitudes.
\begin{equation}
\frac{dt^{a.}_{i.}}{dt} = -i\bra{\Psi^{a.}_{i.}}e^{-\hat{T}(t)}(H_o + V(t))e^{\hat{T}(t)}\ket{\Psi_o}
\end{equation}
On multiplying eq.(38) by $e^{i\epsilon(t)}$ and invoking the RI, the expression for 
time evolution of $\lambda^{b.}_{j.}$ amplitudes can be obtained:
\begin{equation}
\frac{d\lambda^{b.}_{j.}}{dt} =
i\bra{{\tilde{\Lambda}}(t)}\lbrack H_o + V(t),\{{a}^\dagger_b.
a_j.\}\rbrack\ket{{\widetilde{\Psi}_{cc}}(t)}
\end{equation} 
where $\sim$ denotes the usual expressions without the phase factor $e^{\pm i\epsilon(t)}$. 
Expanding the $t^{a.}_{i.}$ amplitudes in orders of perturbation of V(t), the 
expressions for the different orders can be obtained using eq. (38). The first order
derivative can be written as:
%\begin{equation}
%t_\mu(t) = {t}^{(0)}_\mu(t) +  {t}^{(1)}_\mu(t) +  {t}^{(2)}_\mu(t) + ...
%\end{equation}
%\begin{equation}
%i\frac{d\;{t^{a.}_{i.}}^{(0)}}{dt} = \bra{\Psi^{a.}_{i.}}e^{-\hat{T}(0)}H_o\ket{\Psi_{cc}} = 0 ,
%\end{equation}
\begin{equation}
i\frac{d\;{t^{a.}_{i.}}^{(1)}}{dt} = \bra{\Psi^{a.}_{i.}}e^{-\hat{T}(0)}V(t)\ket{\Psi_{cc}} + \bra{\Psi^{a.}_{i.}}e^{-\hat{T}(0)}\lbrack H_o,T^{(1)}\rbrack\ket{\Psi_{cc}}
\end{equation}
The amplitude $t^{a.}_{i.}$ can be expressed in terms of its Fourier transform as: 
\begin{equation}
{t^{a.}_{i.}}^{(1)}(t) = \int_{-\infty}^{\infty} d\omega_1{X^{a.}_{i.}}^{(1)}(\omega_1 + i\Gamma)e^{(-i\omega_1 + \Gamma) t}
\end{equation}
where
\begin{equation}
{X^{a.}_{i.}}^{(1)}(\omega_1 + i\Gamma)  = \sum_{j.b.}{\left\{{(-A + (\omega_1 + i \Gamma) I )}^{-1}\right\}}^{j.b.}_{i.a.}{\lambda^{b.}_{j.}}^{(1)}(\omega_1),
\end{equation}
and  
\begin{equation}
{\lambda^{b.}_{j.}}^{(1)}(\omega_1) = \bra{\Psi^{b.}_{j.}}e^{-\hat{T}(0)}V(\omega_1)\ket{\Psi_{cc}}.
\end{equation}
$A$ is the coupled cluster Jacobian matrix defined as:
\begin{equation}
A^{j.b.}_{i.a.} = \bra{\Psi^{a.}_{i.}}e^{-\hat{T}}\lbrack H_o,\{{a}^\dagger_b.a_j.\}\rbrack\ket{\Psi_{cc}}
\end{equation}
The above process is repeated with $\lambda^{j.}_{b.}$ amplitudes to obtain the derivative expressions and the Fourier transform $Y$.
\begin{equation}
\frac{d\;{\lambda^{b.}_{j.}}^{(1)}}{dt} = i\bra{\Lambda}(\lbrack\lbrack H_o,\{{a}^\dagger_b.a_j.\}\rbrack,T^{(1)}\rbrack + \lbrack V(t),\{{a}^\dagger_b.a_j.\}\rbrack)\ket{\Psi_{cc}} + i\sum_{i.a.} {\lambda^{a.}_{i.}}^{(1)}A^{j.b.}_{i.a.} 
\end{equation}
\begin{equation} 
{Y^{a.}_{i.}}^{(1)}(\omega_1 + i\Gamma) = - \sum_{j.b.}{\eta^{b.}_{j.}}^{(1)}(\omega_1) + \sum_{k.c.}F^{k.c.}_{j.b.} {X^{c.}_{k.}}^{(1)}(\omega_1 + i\Gamma)) \times\{{(A + (\omega_1 + i\Gamma)I)}^{-1}\}^{i.a.}_{j.b.}
\end{equation}
where
\begin{equation}
{\eta^{b.}_{j.}}^{(1)}(\omega_1) = \bra{\Lambda}\lbrack V(\omega_1),\{{a}^\dagger_b.a_j.\}\rbrack\ket{\Psi_{cc}} ,
\end{equation}
and 
\begin{equation}
F^{k.c.}_{j.b.} = \bra{\Lambda}\lbrack\lbrack H_o,\{{a}^\dagger_b.a_j.\}\rbrack,\{{a}^\dagger_c.a_k.\}\rbrack\ket{\Psi_{cc}}  
\end{equation}
The expectation value of any time dependent operator $A$ in CC can be expanded in orders of perturbation.
\begin{equation}
\langle A \rangle = \bra{\Lambda}A\ket{\Psi_{cc}} + \sum_{i.a.}{\lambda^{a.}_{i.}}^{(1)}\bra{\Psi^{a.}_{i.}}e^{-\hat{T}(0)}A\ket{\Psi_{cc}} + \bra{\Lambda}\lbrack A,T^{(1)}\rbrack\ket{\Psi_{cc}} + .....
\end{equation}
Comparing this with equation (34), the CC linear response function can be determined.
\begin{equation}
\int_{-\infty}^{\infty}d\omega_1{\langle\langle A;V(\omega_1)\rangle\rangle}_{\omega_1 + i \Gamma}e^{(-i\omega_1 + \Gamma)t} \equiv \sum_{i.a.}{\lambda^{a.}_{i.}}^{(1)}\bra{\Psi^{a.}_{i.}}e^{-\hat{T}(0)}A\ket{\Psi_{cc}} + \bra{\Lambda}\lbrack A,T^{(1)}\rbrack\ket{\Psi_{cc}}
\end{equation}
In more general terms, the response function can be written as:
\begin{equation}
{\langle\langle A;B\rangle\rangle}_{\omega_1} = \sum_{i.a.} \bra{\Lambda}\lbrack A,\{{a}^\dagger_a.a_i.\}\rbrack\ket{ \Psi_{cc}}{X^{a.}_{i.}}(B,\omega_1) + \sum_{i.a.}\left\{\bra{\Lambda}\lbrack B,\{{a}^\dagger_a.a_i.\}\rbrack\ket{\Psi_{cc}} + \sum_{k.c.}F^{k.c.}_{i.a.}{X^{c.}_{k.}}(B,\omega_1)\right\}{X^{a.}_{i.}}(A,-\omega_1).
\end{equation}
where \centerline{${X^{a.}_{i.}}(B,\omega_1) = \sum_{j.b.}\left\{ {( -A + \omega_1I)}^{-1}\right\}^{j.b.}_{i.a.}\;B^{b.}_{j.},$}\\and \centerline{ $B^{b.}_{j.} = \bra{\Psi^{a.}_{i.}}e^{-\hat{T}(0)}B\ket{\Psi_{cc}}$ .}\\\\
Thus for calculating the response functions, we need to solve two sets of
linear equations for obtaining ${X^{a.}_{i.}}(B,\omega_1)$ and
${X^{a.}_{i.}}(A,-\omega_1)$.
%\newpage
%\paragraph{Optical rotation calculations}~\\
%Rosenfeld,\cite{Rosenfeld29} using semi-classical electrodynamic theory, showed
%that the induced electric dipole moment can be written as: \begin{equation}
%\langle\vec{\mu}\rangle = \alpha\vec{E} +
%\frac{1}{\omega}\textbf{G}^\prime\frac{\partial\vec{B}}{\partial t}
%\end{equation}
%where $\vec{E} $ and $\vec{B}$ are electric and magnetic field vectors
%respectively, $\alpha$ is the electric dipole polarizability tensor while
%$\textbf{G}^\prime$ tensor is the key quantity for calculating optical
%rotation.  \begin{equation}
%\textbf{G}^{\prime}_{xy}(\omega) = -\frac{2}{\hbar} Im\sum_{n \neq 0}\frac{\omega\; \bra{\psi_o}\mu_x\ket{\psi_n}\bra{\psi_n}m_y\ket{\psi_o}}{\omega^{2}_{n0}-\omega^2}
%\end{equation}
%where $\mu$ and $m$ are electric and magnetic dipole operators: $\mu = \sum_i
%r_i , m= \sum_ir_i\times p_i$ in atomic units. $\omega$ is the frequency of
%light, $\omega_{no}$ is the excitation energy for the $n^{th}$ state ($\psi_n$) and
%$Im$ means the imaginary part of the equation. The trace of this tensor is related
%to the specific rotation, usually denoted as ${\lbrack\alpha\rbrack}_\omega $ in
%deg dm$^{\text{-1}}$(g/mL)$^{\text{-1}}$. After averaging over all possible
%orientations of the molecule, the following expression is obtained\cite{Crawford06}. \begin{equation}
%{\lbrack\alpha\rbrack}_{\omega} = \frac{(72.0 \times 10^6){\hbar}^2 N_A\;\omega}{c^2{m_e}^2 M} \times \left[ \frac{1}{3}Tr(\textbf{G}^\prime)\right]
%\end{equation}
%where $\textbf{G}^\prime$ and $\omega$ are in atomic units, $N_A$ is Avogadro's number, c is the speed of light in (m/s), m$_{\text{e}}$ is the mass of electron (kg) and M is the molecular mass (amu). The $\textbf{G}^\prime$ tensor can be obtained from using the CCLR method.
Thus, we can use this linear response formalism and can calculate the $\textbf{G}^\prime$ tensor
using eq.(4) in which A is the dipole moment operator and B is the magnetic field. However, there 
can be different representations of an operator which are referred to as gauges. It is observed that 
in the truncated CCLR method if the length gauge representation of the dipole operator is used, i.e. $\mu \equiv r$ (in atomic units), the optical rotation results are origin dependent. This problem is overcome in CCLR by using the velocity
gauge representation where $\mu \equiv r\times p$. However, both these gauges are equivalent for 
exact wavefunctions. However, the velcity gauge calculated optical rotation does not decay to zero 
at the static limit, i.e. when the external field is zero, which is an unphysical result. 
To correct this problem, another representation called the modified velocity gauge (MVG)
proposed by Pedersen \cite{Pedersen04} is used which shifts the values obtained
by velocity gauge by its static limit. 

%\begin{equation}
%\textbf{G}^{\prime}(\omega) = \text{-Im}\langle\langle\mu;m\rangle\rangle \equiv \text{- Im}\langle\langle r;r\times p\rangle\rangle
%\end{equation} 
%The above results obtained using the length gauge representation ($\mu
%\equiv r$) above, however is origin dependent. This is fixed by using the velocity
%gauge representation where the dipole moment operator is written as a momentum operator ($\mu
%\equiv {\frac{p}{\omega}}$) . Length and velocity gauge representations are equivalent for exact 
%wavefunctions.
%\begin{equation}
%\text{Tr}{\langle\langle r;r\times p\rangle\rangle}_{\omega} = \frac{1}{\omega}\text{Tr}{\langle\langle p;r\times p\rangle\rangle}_{\omega} 
%\end{equation}
%However, the velocity gauge calculations does not decay to zero at the static
%limit, i.e when the external field is zero, which is an unphysical result. To
%overcome this, another representation called modified velocity gauge (MVG)
%proposed by Pedersen \cite{Pedersen04} is used which shifts the values obtained
%by velocity gauge by it's static limit value.

\section{Solvation Models}
For the purpose of computational practicality, the QM solvation models aim to reduce the degrees of freedom of
the solvent molecules without compromising the quality of description of the solvent-solute and the solvent-solvent
interactions. \\
In this regard, the molecular mechanics (MM) based models, broadly known as QM/MM methods, usually represent the
solvent molecules in the form of a classical potential by replacing them with point charges, polarizabilities etc.\cite{}
This potential polarizes the charge distribution of the solute which in turn creates a new potential or field because of
which induced dipole moments appear in solvents (depending on the point polarizabilities), resulting in a new
potential which again polarizes the solute and so on. Thus, the final one-electron solvent potential is attained in a
self-consistent procedure and gets added to the solute's Hamiltonian for further response calculations. Fluctuating
charge model(FQ)\cite{}, the drude oscillator model\cite{} and effective fragmentation potential (EFP)\cite{} are some
other popular examples of QM/MM methods.

Quantum continuum models like the polarizable continuum model (PCM)\cite{}
on the other hand place the solute in a cavity and replace the solvent molecules by a dielectric medium. The
solute-solvent interactions are then captured by adding charges on the surface on the cavity (apparent surface charges)
which are solved in a self-consistent manner as described above.\\


Mennucci et al. extended PCM to work with the linear response formalism of TD-DFT\cite{Mennucci02}
and calculated optical rotations for chiral molecules in different solvents. They were able to obtain promising results

More recently, Cappelli implemented the
QM/FQ/PCM approach for calculating optical rotations of methyloxirane


with molecular dynamics (MD) simulations



 a series of snapshots from the MD is extracted, which constitute the actual model system on which the QM/FQ/PCM calculations of the desired properties are performed and then averaged out to obtain the final results. Actually, the computational cost of the procedure is mainly determined by the cost of the single QM/FQ/PCM calculation, which basically depends on the QM level and the kind of property (first, second, third derivative, etc.) to be calculated, together with the number of representative snapshots necessary to allow the final results to converge to a sensible value.



FQ/PCM/MM has been combined by

In recent years, the ab inito methods
mentioned above have been combined with various solvation based
approaches\cite{Neugebauer05,Neugebauer09,Mennucci02,Tomasi05,JensenGordon96}
to calculate the optical rotation in solution. This review condiders some of
the popular approaches that are being used and describes their working
mechanisms and performances in detail. In the current/future work section of
this review, some reduced scaling techniques are proposed with the aim of
making these methods more efficient and computationally practical. Also, some
of the essential, underlying coupled cluster and response theory used for
optical rotation calculations are presented and discussed.

%1. couple solvent response to the electromangnetoic field
%2. include vibrational contributions in solvents
%3. explicit + implicit models
