\section{Hartree-Fock Theory}
Central to the field of quantum mechanics is the equation proposed by Erwin Schr\"odinger 
in 1926 or the Schr\"odinger's equation \cite{Schrodinger26}. One of the biggest factors 
contributing to the rich diversity of quantum chemical theories is the fact that Schr\"odinger's equation (SE)
is exactly solvable for only one electron systems like hydrogen atom. As such, numerous approximations
have been proposed over the years to solve the equation for many electron systems.
The Hartree-Fock (HF) theory\cite{} is one of the simplest approximation in this regard. This
model attempts to transform the many body problem of SE into a one body problem where each electron
only interacts with a mean field created by the other electrons. For an N-electron system, the HF 
wavefunction is a slater determinant (obeys Pauli's antisymmetry principle naturally) composed of N 
spin orbitals, i.e. the one electron solutions of the HF equation. Following is a brief outline
of the HF procedure: A Lagrangian is constructed with the HF energy (the expectation value of the 
slater determinant) coupled with the constraint of the spin orbitals always remaining 
orthonormal to each other. The Lagrangian thus can be seen as a functional of the spin 
orbitals. In the next step, the variational principle is employed to set the first order 
change in the Lagrangian with respect to spin orbitals to zero. Finally, one obtains 
a non-linear eigenvalue equation where the spin orbitals are the eigenfunctions of 
the Fock operator which itself depends on the spin orbitals themselves. 
Thus the equation is solved in a self-consistent manner usually in conjunction with 
the linear combination of atomic orbital model (LCAO) approach, where the spin 
orbitals are expressed as a linear combination of basis functions and the 
coefficients are determined from the matrix based formulation of the above eigenvalue equation.\\
Even with this mean field approach, one can extract almost 95-98\% of the total 
electronic energy. The remaining energy, also known as the correlation energy,
however, is very essential if one wants to attain ``chemical accuracy", ex.  
1kcal/mol for interaction energies. Inspite of this, the simple yet robust 
structure of the HF procedure makes it one of the most popular reference 
wavefunctions for more complicated and accurate theoretical models.
\subsection{Configuration interaction Method}
CI is a matrix mechanics solution of the schrodinger's equation.For a
N-electron molecule, the wavefunction can be expressed as a linear combination
of N-electron basis functions.CI uses slater determinants as the basis, since
they are antisymmetric, orthogonal and  form a complete space.  \begin{equation}
\ket{\Psi} = c_o\ket{\Phi_o} + \sum_{ia}{c}_{i}^{a}\ket{{\Phi}^a_i} + \sum_{i>j,a>b}c_{ij}^{ab}\ket{\Psi^{ab}_{ij}} + \sum_{i>j>k,a>b>c}c_{ijk}^{abc}\ket{\Psi^{abc}_{ijk}} + ...
\end{equation}
 Now, just like HF method, the variational method is invoked to minimize the
expectation value of energy with respect to the coefficients.We get the final
matrix eigenvalue equation : \begin{equation}
HC = CE
\end{equation}
where , $H_{ij} = \bra{\Psi_i}H\ket{\Psi_j}$ , C is the coefficient matrix and
E is the diagonal matrix containing the energies, with the lowest energy, being
that of the ground state.In full CI, all possible slater determinants are
included, and we get an exact wavefunction in the basis set limit.However, Full
CI is computationally very expensive and is not used except in benchmark
calculations of small molecules.So, we use truncated CI schemes like CIS and
CISD which only include singles and singles and doubles determinants
respectively.These methods can extract up to 95\% of correlation energy\cite{HarrisonHandy83} and
can be applied to solve for excited states, open-shell systems and systems that
are far from their equilibrium geometry, which makes them very useful.However,
truncated CI methods are not size-extensive,which is a big disadvantage.
\subsection{Coupled Cluster Theory} 
%\paragraph{Coupled cluster theory} ~ \\\\
Coupled cluster\cite{Crawford00} (CC) theory is a very accurate electronic
structure method that has been widely used in quantum chemistry.
It takes into account the electron correlation effects by expressing the
wavefunction as a linear combination of slater determinants within an
exponential framework. This is achieved by the use of cluster operators $\hat{T}$ 
acting on the HF wavefunction.
%is one of the most accurate yet computationally
%affordable method, which is very widely used in quantum chemistry.CC
%wavefunction is an exponential ansatz\cite{Crawford00} which incorporates the electron
%correlation effects through cluster operators 
%$\hat{T}$.  
\begin{equation}
\ket{\Psi_{CC}} = e^{\hat{T}}\ket{\Psi_{o}} , 
\end{equation}
where
\begin{equation}
 \hat{T} = \hat{T_1} + \hat{T_2} + \hat{T_3} + ... \;\hat {T_n} 
\end{equation}
$\ket{\Psi_{o}}$ is the reference wavefunction, usually taken as the HF
wavefunction, and the $\hat{T_1}$ operator is the linear combination of all single excitation
operators, which replace an occupied orbital in the HF wavefunction with a
virtual orbital. It takes into account the relaxation of the one electron basis
functions, i.e HF molecular orbitals, due to the correlation of the electronic
motions. In second quantization\cite{JorgensenSimons81} formalism, $\hat{T_1}$
can be written as :
\begin{equation}
\hat{T_1} = \sum_{ia}t^a_i\{{a}^\dagger_a a_i\}
\end{equation}
where the indices i, j, k ... refer to occupied orbitals while
the indices a, b, c ... refer to virtual spin orbitals.
In the above equation, ${a}^\dagger_a$ is called the creation operator as it
creates a new particle state (virtual orbital) when it acts on a slater determinant.
\begin{equation}
{a}^\dagger_a\ket{\phi_b...\phi_d} = \ket{\phi_a\phi_b...\phi_d}
\end{equation}
$a_i$ is the hermitian conjugate of the creation operator and is called
the annihilation operator as it removes a hole state (occupied orbital) when it acts on a
slater determinant.
\begin{equation}
a_i\ket{\phi_i\phi_j...\phi_l} = \ket{\phi_j...\phi_l}
\end{equation}
The $\hat{T_2} $ operator is the linear combination of all double excitation operators.
It correlates the motion of all pairs of electrons of the molecule and can be expressed
in second quantization as:
\begin{equation}
\hat{T_2} = \frac{1}{4}\sum_{ijab}t^{ab}_{ij}\{a^\dagger_aa^\dagger_ba_ja_i\}
\end{equation}
Similarly, the ${\hat{T_3}}$ operator correlates the motion of all triplets of
electrons and so on. In general, \begin{equation}
\hat{T_n} = {\frac{1}{n!}}^2\sum_{ij..ab..}^nt^{ab..}_{ij..}\{a^\dagger_aa^\dagger_b...a_ja_i\}
\end{equation}
%\subsubsection{Coupled Cluster energy}
%\paragraph{Coupled Cluster energy}~ \\
If we expand the exponential ansatz of the CC wavefunction,
\begin{equation}
\ket{\Psi_{CC}} = (1+(\hat{T_1} + \hat{T_2} + \hat{T_3} + ... ) + \frac{1}{2!}{(\hat{T_1} + \hat{T_2} + \hat{T_3} + ...)}^2 + ... )\ket{\Psi_o}
\end{equation}
we get a linear combination of all possible Slater determinants, which makes CC
wavefunction an exact wavefunction in the basis set limit, just like full
configuration interaction. Thus, the full CC
wavefunction is a solution of the Schr\"odinger equation.
\begin{equation}
\hat{H}e^{\hat{T}}\ket{\Psi_o} = E_{cc}\;e^{\hat{T}}\ket{\Psi_o}
\end{equation}
On multiplying the above equation by the inverse of the exponential operator i.e
$e^{-\hat{T}}$ and projecting it against the reference and excited
determinants, we get the energy and the amplitude equations of coupled cluster
theory.
\begin {equation}
\bra{\Psi_o}e^{-\hat{T}}\hat{H}e^{\hat{T}}\ket{\Psi_o} = E_{cc}
\end{equation}
\begin{equation}
%\bra{\Psi^{ab..}_{ij..}}e^{-\hat{T}}\hat{H}e^{\hat{T}}\ket{\Psi_o} = E\cancelto{0}{\bra{\Psi^{ab..}_{ij..}}\Psi_o\rangle} = 0 .
\bra{\Psi^{a.}_{i.}}e^{-\hat{T}}\hat{H}e^{\hat{T}}\ket{\Psi_o} = 0 .
\end{equation} 
Here $\Psi^{a.}_{i.}$ can refer to any excited Slater determinant singles, doubles etc. 
The similarity transformed Hamiltonian $e^{-\hat{T}}\hat{H}e^{\hat{T}}$ also
written as $\bar{H}$ can be expressed in the Hausdorf expansion
\cite{Merzbacher70}as:
\begin{equation}
e^{-\hat{T}}\hat{H}e^{\hat{T}} = \hat{H} + \lbrack\hat{H},\hat{T}\rbrack + \frac{1}{2!}\lbrack\lbrack\hat{H},\hat{T}\rbrack,\hat{T}\rbrack + \frac{1}{3!}\lbrack\lbrack\lbrack\hat{H},\hat{T}\rbrack,\hat{T}\rbrack,\hat{T}\rbrack + \frac{1}{4!}\lbrack\lbrack\lbrack\lbrack\hat{H},\hat{T}\rbrack,\hat{T}\rbrack,\hat{T}\rbrack,\hat{T}\rbrack + ...
\end{equation}
Also, the second quantized form of the Hamiltonian can be written as\cite{Crawford00}:
\begin{equation}
\hat{H} = \sum_{pq}h_{pq}\{a^\dagger_pa_q\} + \frac{1}{4}\sum_{pqrs}\bra{pq}\ket{rs}\{a^\dagger_pa^\dagger_qa_sa_r\}
\end{equation}
Using the above two equations and the Wick's theorem\cite{Wick50}, the coupled cluster energy
equation gets simplified as:
\begin{equation}
E_{cc} = E_o + \sum_{ia}f_{ia}t^a_i + \frac{1}{4}\sum_{aibj}\bra{ij}\ket{ab}t^{ab}_{ij} + \frac{1}{2}\sum_{aibj}\bra{ij}\ket{ab}t^a_it^b_j
\end{equation} The amplitudes $t^a_i$ and $t^{ab}_{ij}$ can be obtained by
solving the coupled cluster amplitude equations. However, Full CC is
computationally impractical and truncated CC methods like CCSD:\;$\hat{T} =
\hat{T_1} + \hat{T_2}$, CCSDT:\;$\hat{T} = \hat{T_1} + \hat{T_2} + \hat{T_3}$ 
are used. The CCSD(T)\cite{Shen12} method, which treats the triples approximately using
perturbation theory is also a popular method.
%is also very popular and uses an approximate 
%treatment of triples using perturbation theory.
%CCSD: $\hat{T} = \hat{T_1} + \hat{T_2} $ , CCSDT: $\hat{T} = \hat{T_1} + \hat{T_2} + \hat{T_3}$
%, CCSDTQ: $\hat{T} = \hat{T_1} + \hat{T_2} + \hat{T_3} + \hat{T_4}$.
%\\%CCSD(T) method uses an approximate treatment of triples using perturbation theory.
%For the CCSD method the amplitude equations are:
%\begin{equation}
%\bra{\Psi^{a}_{i}}e^{-\hat{T}}\hat{H}e^{\hat{T}}\ket{\Psi_o} = 0
%\end{equation}
%\begin{equation}
%\bra{\Psi^{ab}_{ij}}e^{-\hat{T}}\hat{H}e^{\hat{T}}\ket{\Psi_o} = 0
%\end{equation}
%CC truncation methods are more effective compared to that of the
%CI. For example, although CCSD has the same number of coefficients
%($t^{ab}_{ij}$ and $t^a_i$) as CISD ($c^{ab}_{ij}$ and $c^a_i$), CCSD
%implicitly includes higher excitations than CISD because of the products of
%cluster operators like $\hat{T_1}\hat{T_2}$, ${(\hat{T_2})}^2$ which capture
%the triple and quadruple excitation contributions to the singles and doubles
%amplitudes.
%The exponential structure of the CC wavefunction makes more advantageous than the 
%linear CI method. Because of this, truncated CC methods like CCSD are size extensive, which means they
The exponential structure of the wavefunction makes the truncated CC methods more efficient 
compared to that of the linear CI methods. CCSD wavefunction implicitly includes the triples
and quadruples excitation contributions to its singles and doubles amplitude equations because 
of the products of cluster operators like $\hat{T_1}\hat{T_2}$, ${(\hat{T_2})}^2$ etc., unlike 
the CISD method which can only include singles and doubles excitation contributions. 
Another advantage of this exponential ansatz is the property of size consistency\cite{Crawford00}.
It means that the sum of the CC energies of non-interacting fragments (each calculated 
separately) is equal to the energy of the supermolecule when all fragments are included 
in the calculations. CC methods give very accurate results for molecules at their equilibrium 
geometries\cite{Bartlett}. CCSD(T) is often referred as the "gold standard" method of electronic structure 
theory. 
%Also, one of the major advantages with the CC truncation methods is their
%size-extensivity which makes them very useful. 
However, CC methods are computationally expensiver: CCSD scales as $O(N^6)$, CCSD(T) as
$O(N^7)$, CCSDT as $O(N^8)$ and so on, where $N$ is the number of basis functions.\\\\
%and is routinely used for accurate results only for small molecules. While CCSD(T)
%gives very good results for molecules at their equilibrium geometry, it fails
%to describe diradical species and bond-breaking. CCSDT and CCSDTQ methods are
%used virtually exclusively for high accuracy calculations of small molecules
%as they are very computationally expensive.
%\paragrpah{\bf{CC analytic derivatives}} ~\\\\
\subsection{Coupled Cluster Analytic Derivatives}
%\paragraph{CC analytic derivatives} ~ \\\\
Molecular properties like dipole
moments, IR intensities, force constants etc., depend upon the gradients of the
molecular energy with respect to external perturbations. 
In this section, a derivation of first order gradient expressions of the CC method is presented.\\
%If we take the derivative of the CC energy directly with respect to any perturbation 
%X,
%, we get
%\begin{equation}
%\frac{\partial{E_{cc}}}{\partial X} = \bra{\phi_0}\frac{\partial{\bar H}}{\partial X}\ket{\phi_0},
%\end{equation}
%If we use the above equation for calculating gradients, 
%we need to calculate
%$\frac{\partial{t^a_i}}{\partial X}, \frac{\partial{t^{ab}_{ij}}}{\partial X}$
%, we need to calculate the derivatives of non-linear amplitude equations, 
%which makes it very computationally expensive. As a result, a different 
%approach is used where one needs to solve some linear equations i.e. 
%the Lambda equations, which are independent of the perturbation. 
%In this section, a derivation of first order gradient expressions 
%of CC method is presented using this approach.\\
The gradient of the CC energy with respect to any external perturbation 
X is written as:
\begin{equation}
\frac{\partial{E_{cc}}}{\partial X} = \bra{\Psi_o}\frac{\partial{\bar H}}{\partial X}\ket{\Psi_o} = \bra{\Psi_o}{\bar{H}}^X + \lbrack\bar H , \frac{\partial{\hat T}}{\partial X}\rbrack\ket{\Psi_o} 
\end{equation}
where, $ {\bar{H}}^X =  e^{-\hat{T}}\frac{\partial{\hat H}}{\partial X}e^{\hat{T}}$ contains the 
derivatives of the basis functions, MO coefficients etc. Invoking the 
resolution of identity (RI),
\begin{equation}
 1 = \ket{\Psi_o}\bra{\Psi_o} + \sum_{ia}\ket{{\Psi}^a_i}\bra{{\Psi}^a_i} + \frac{1}{4}\sum_{ijab}\ket{{\Psi}^{ab}_{ij}}\bra{{\Psi}^{ab}_{ij}} + ...
\end{equation}
equation (19) can be expressed as:
\begin{equation}
\frac{\partial{E_{cc}}}{\partial X} = \bra{\Psi_o}{\bar{H}}^X\ket{\Psi_o} + \sum_{i.a.}\bra{\Psi_o}\bar{H}\ket{{\Psi}^{a.}_{i.}}\bra{{\Psi}^{a.}_{i.}}\frac{\partial{\hat T}}{\partial X}\ket{\Psi_o}.
\end{equation}
The second term of the above equation involves calculating derivatives of
amplitudes i.e. $\bra{{\Psi}^{a}_{i}}\frac{\partial{\hat T}}{\partial
X}\ket{\Psi_o} = \frac{\partial{t^a_i}}{\partial X}$, which can be very
computationally expensive if calculated directly. As a result, we would like to
recast the above equation in something which is computationally practical. We
start with the CC amplitude equations, say singles equation and take
its derivative with respect to X.
\begin{equation} 
0 = \bra{{\Psi}^a_i}{\bar{H}}^X + \lbrack\bar H , \frac{\partial{\hat T}}{\partial
X}\rbrack\ket{\Psi_o} 
\end{equation} 
Using RI as before, we can write the above equation in the folllowing form:
\begin{equation}
\sum_{j.b.}\bra{{\Psi}^a_i}\bar{H}-E_{cc}\ket{{\Psi}^{b.}_{j.}}\bra{{\Psi}^{b.}_{j.}}\frac{\partial{\hat T}}{\partial X}\ket{\Psi_o} = - \bra{{\Psi}^a_i}{\bar{H}}^X\ket{\Psi_o} 
\end{equation}
or,
\begin{equation} \bra{{\Psi}^{b.}_{j.}}\frac{\partial{\hat T}}{\partial
X}\ket{\Psi_o} = -\sum_{ia}
\bra{{\Psi}^{b.}_{j.}}{(\bar{H}-E_{cc})}^{-1}\ket{{\Psi}^a_i}\bra{{\Psi}^a_i}{\bar{H}}^X\ket{\Psi_o}
\end{equation}
The above equation was derived by considering the gradient of just the singles equation.
Including the gradient of all the other amplitude equations and plugging the modified 
equation into the expression for the gradient of energy eq.(21), we obtain:
\begin{equation}
\frac{\partial{E_{cc}}}{\partial X} = \bra{\Psi_o}{\bar{H}}^X\ket{\Psi_o} -
\sum_{i.a.}\bra{\Psi_o}\bar{H}\ket{{\Psi}^{a.}_{i.}}\sum_{j.b.}
\bra{{\Psi}^{a.}_{i.}}{(\bar{H}-E_{cc})}^{-1}\ket{{\Psi}^{b.}_{j.}}
\bra{{\Psi}^{b.}_{j.}}{\bar{H}}^X\ket{\Psi_o}
\end{equation}
We define a perturbation independent $\Lambda$ operator such that:
\begin{equation}
\bra{\Psi_o}\Lambda\ket{\Psi^{b.}_{j.}} = -\sum_{i.a.}\bra{\Psi_o}\bar{H}\ket{{\Psi}^{a.}_{i.}}
\bra{{\Psi}^{a.}_{i.}}{(\bar{H}-E_{cc})}^{-1}\ket{{\Psi}^{b.}_{j.}}.
\end{equation}
This $\Lambda$ operator can be seen as a de-excitation operator and can be written as:
\begin{equation}
\Lambda = \Lambda_1 + \Lambda_2 + \Lambda_3 + ...
\end{equation} 
where $\Lambda_1 = \sum_{ia}\lambda^i_a\{{a}^\dagger_i a_a\}$ is the singles de-excitation
operator, $\Lambda_2$ is the doubles de-excitation operator and so on.  
Using the lambda operator, the gradient expression gets simplified as:
\begin{equation}
\frac{\partial{E_{cc}}}{\partial X} = \bra{\Psi_o}(1 + \Lambda){\bar{H}}^X\ket{\Psi_o} = \bra{\Psi_o}(1 + \Lambda)e^{-\hat{T}}\frac{\partial{\hat H}}{\partial X}\ket{\Psi_{cc}}
\end{equation}
The governing equation for calculating the lambda amplitudes eq. (26) can be written
in a more compact form:
\begin{equation}
\bra{\Psi_o}(1 + \Lambda)(\bar{H} - E_{cc})\ket{{\Psi}^{a.}_{i.}} = 0
\end{equation}
Thus instead of taking the gradients of the non-linear $T$ amplitude
equations with respect to perturbations, we solve the linear perturbation-independent 
$\Lambda$ equations for calculating the energy gradient.\\
Also, $\bra{\Psi_o}(1 + \Lambda)e^{-\hat{T}}$ is the CC left hand wavefunction and 
eq. (28) is a generalized Hellman-Feynman equation\cite{Feynman39}. For full CC the left and 
right hand wavefunctions are Hermitian conjugates of each other, but this is not true for 
the truncated CC methods. This is because of the non-hermiticity of the $\bar{H}$ operator.
However, for second order derivatives, we do need to calculate the first derivatives of 
either the $T$ or $\Lambda$ amplitudes with respect to a perturbation.
%The derivative of the similarity transformed hamiltonian can be expressed as:
%\begin{equation}
%\frac{\partial{\bar{H}}}{\partial X} = {\bar{H}}^X + \lbrack\bar H , \frac{\partial{\hat T}}{\partial X}\rbrack, 
%\end{equation}

%\\By combining the CC energy and amplitude equations, we can write:
%\begin{equation}
%E_{CC} = \bra{\Lambda}\hat{H}\ket{\Psi_{CC}}
%\end{equation}
%where
%\begin{equation}
%\bra{\Lambda} = \bra{\Psi_o} + \sum_{\mu}{\zeta_{\mu}}\bra{\mu}e^{-\hat{T}}
%\end{equation} In the above equation, $\bra{\mu}$ represents any excited Slater determinant.
%$\zeta_{\mu}$ can be obtained if the bra state (Lambda) obeys the schr\"odinger's
%equation.
%\begin{equation}
%\bra{\Lambda}\hat{H}e^{\hat{T}} = \bra{\Lambda}e^{\hat{T}}E_{cc}
%\end{equation}
%Right projecting the above equation onto the subspace \{$\ket{\nu}$\} (excited Slater determinants):
%\begin{equation}
%\sum_{\mu}\zeta_{\mu}A_{\mu\nu} = -\bra{\Psi_o}\lbrack\hat{H},\hat{\tau_{\nu}}\rbrack\ket{\Psi_{cc}}
%\end{equation}
%where $\hat{\tau_{\nu}}$ is an excitation operator:$\;\;\;\hat{\tau_{\nu}}\ket{\Psi_o} = \ket{\nu}$ , and 
%\begin{equation}
%A_{\mu\nu} = \bra{\mu}e^{-\hat{T}}\lbrack\hat{H},\hat{\tau_{\nu}}\rbrack\ket{\Psi_{cc}}
%\end{equation}
%We can solve for the parameters $\zeta_{\mu}$ from the above equation to get $\bra{\Lambda}.$ In the presence of a time-independent perturbation described by $\alpha \hat{V}$ with $\alpha$ as the strength parameter, the gradient of coupled cluster energy with respect to $\alpha$ at zero perturbation strength can be shown as:
%\begin{equation}
%\frac{\partial{E_{cc}}}{\partial {\alpha}}|_{\alpha=0} \;\;= \frac{\partial}{\partial{\alpha}}\bra{\Lambda (\alpha)}\hat{H}+\alpha \hat{V}\ket{\Psi_{CC}(\alpha)}|_{\alpha=0}\;\; = \;\;\;\bra{\Lambda}\hat{V}\ket{\Psi_{CC}}
%\end{equation}
%We are able to obtain the first order gradients of CC just by solving the linear
%lambda equations to get the $\zeta_{\mu}$ parameters and plugging them in the
%above expression. The $\bra{\Lambda}$ state is nothing the left hand coupled cluster wave
%function, as it is a solution to the Schr\"odinger equation and the above equation satisfies the 
%generalized Hellmann-Feynman theorem\cite{Feynman39}.
%%For full CC, $\bra{\Lambda}$ and $\ket{\Psi_{cc}}$ are just hermitian
%%conjugate, but they are different in case of truncated CC methods because %of
%%the non-hermiticity of $\bar{H}$ operator.
%\subsubsection{Coupled Cluster Linear Response (CCLR)}
%\paragraph{Exact states}~\\
%\paragraph{Coupled cluster linear response (CCLR)}~\\\\
\subsection{Coupled Cluster Linear Response (CCLR) Theory}
CCLR method proposed by Koch and J{\o}rgensen in 1990\cite{Koch90} is a recipe for accurate
calculations of response properties like dynamic polarizabilities, optical
rotations, etc. In this section, we will outline important steps for the
derivations of both general and CCLR response functions and also talk about
different gauge representations used for calculating the response functions.\\
The operator $V(t)$ which describes the interaction between the molecule and 
an external time-dependent field can be expressed in the frequency domain as: 
\begin{equation}
V(t) = \int_{-\infty}^{\infty}d\omega\;\;V(\omega) e^{(-i\omega + \Gamma)t} ,
\end{equation}
The full Hamiltonian can then be written as: $H = H_o + V(t)$, where $H_o$ is the
time independent unperturbed Hamiltonian. Assuming that $\Psi_o$ is an eigenstate of $H_o$
%An exact wavefunction satisfies the Schr\"odinger equation: $H_o\ket{O} =
%E_o\ket{O}$ where $H_o$ is the time independent Hamiltonian and $\ket{O}$ is an
%eigenstate of the system. 
and that the molecule is in state $\Psi_o$ when the perturbation starts at $t = -\infty$, 
the $\ket{\Psi_o}$ state evolves in time as $\ket{\Psi_o(t)}$ according to the time-dependent 
Schr\"odinger equation.
\begin{equation}
i\frac{d}{dt}\ket{\Psi_o(t)} = (H_o + V^t)\ket{\Psi_o(t)}
\end{equation}
%It can be shown that the time dependent state $\ket{\Psi_o(t)}$ is related to 
Following the work of Olsen\cite{Olsen85}, the time dependent state $\ket{\Psi_o(t)}$
can be written as:
\begin{equation}
\ket{\Psi_o(t)} = \ket{\bar{\Psi_o}}e^{i\epsilon(t)}
\end{equation}
where $\epsilon$ is a phase factor and the state $\ket{\bar{\Psi_o}}$ can be expressed as\cite{Koch90}:
%using perturbation theory. Specifically,
\begin{equation}
\ket{\bar{\Psi_o}} = \ket{\Psi_o} + {\ket{\Psi_o}}^{(1)}+ {\ket{\Psi_o}}^{(2)} + ...
\end{equation}
where first order correction ${\ket{\Psi_o}}^{(1)}$ and others can be determined 
from the time dependent perturbation theory. For calculating any response property, the
expectation value of the respective time independent  operator is calculated in
the presence of an external field. The response functions can be seen as the
coefficients of terms which appear in different orders of perturbation in $V(t)$
in the expansion of the expectation value.\cite{Koch90} 
%Furthermore, the expectation value of any time dependent operator $A$ can
%expanded as\cite{Koch90} : 
\begin{equation}
\bra{\Psi_o(t)}A\ket{\Psi_o(t)} = \bra{\Psi_o}A\ket{\Psi_o} + \int_{-\infty}^{\infty}d\omega_1{\langle\langle A;V({\omega_1})\rangle\rangle}_{\omega_1 + i\Gamma}e^{(-i\omega_1 + \Gamma)t} + .....
\end{equation}
where we only considered the linear response function $\langle\langle A;V({\omega_1})\rangle\rangle$ 
and neglected the higher order terms. If the perturbation field is composed of a 
single frequency ($\omega_1$) , the linear response function can be written as:
\begin{equation}
\langle\langle A;V({\omega_1})\rangle\rangle = \sum_k\{
\frac{\bra{\Psi_o}A\ket{\Psi_k}\bra{\Psi_k}V({\omega_1})\ket{\Psi_o}}{\omega_1
- \omega_k + i\Gamma}  -
\frac{\bra{\Psi_o}V({\omega_1})\ket{\Psi_k}\bra{\Psi_k}A\ket{\Psi_o}}{\omega_1 + \omega_k +
i\Gamma}\} \end{equation}
%The above equation is called the sum of states (SOS) equation , 
where $\omega_k$ is the excitation energy between the states $\Psi_o$ and $\Psi_k$ 
and the summation runs over all the solutions of the time independent Schr\"odinger equation $\Psi_k$.
It can be seen from eq.(2) that $\textbf{G}^{\prime}(\omega)$ is nothing but the imaginary part of this 
response function if we take $A$ as the dipole moment operator with $V$ as a magnetic field.
However, this approach is very computationally expensive for CC methods and we approximate the above 
linear response function using CCLR theory.
%\paragraph{CC linear response function}~\\\\
Using the same approach as above, the right and left hand CC wavefunction evolve in time as\cite{Koch90}:
\begin{equation}
\ket{\Psi_{cc}(t)} = e^{\hat{T}(t)}\ket{\Psi_o}e^{i\epsilon(t)}
\end{equation}
\begin{equation} \bra{\Lambda(t)} = \{\bra{\Psi_o} +
\sum_{i.a.}\lambda^{a.}_{i.}(t)\bra{\Psi^{a.}_{i.}}e^{-\hat{T}(t)}\}e^{-i\epsilon(t)}
\end{equation}
where $e^{\pm i\epsilon(t)}$ is a time dependent phase factor
and we have time dependent $T$ and $\lambda^{a.}_{i.}$ amplitudes. The governing equation for 
the time dependence of these amplitudes is the time-dependent Schr\"odinger equation.
\begin{equation}
i\frac{d}{dt}\ket{\Psi_{cc}(t)} = (H_o + V(t))\ket{\Psi_{cc}(t)} 
\end{equation}
\begin{equation}
\frac{d}{dt}\bra{\Lambda(t)} = i\bra{\Lambda(t)}(H_o + V(t))
\end{equation}
On multiplying eq. (38) by $e^{-\hat{T}(t)}$ on both sides and projecting 
it against $\bra{\Psi^{a.}_{i.}}$, we obtain the expression for the time evolution 
of $t^{a.}_{i.}$ amplitudes.
\begin{equation}
\frac{dt^{a.}_{i.}}{dt} = -i\bra{\Psi^{a.}_{i.}}e^{-\hat{T}(t)}(H_o + V(t))e^{\hat{T}(t)}\ket{\Psi_o}
\end{equation}
On multiplying eq.(38) by $e^{i\epsilon(t)}$ and invoking the RI, the expression for 
time evolution of $\lambda^{b.}_{j.}$ amplitudes can be obtained:
\begin{equation}
\frac{d\lambda^{b.}_{j.}}{dt} =
i\bra{{\tilde{\Lambda}}(t)}\lbrack H_o + V(t),\{{a}^\dagger_b.
a_j.\}\rbrack\ket{{\widetilde{\Psi}_{cc}}(t)}
\end{equation} 
where $\sim$ denotes the usual expressions without the phase factor $e^{\pm i\epsilon(t)}$. 
Expanding the $t^{a.}_{i.}$ amplitudes in orders of perturbation of V(t), the 
expressions for the different orders can be obtained using eq. (38). The first order
derivative can be written as:
%\begin{equation}
%t_\mu(t) = {t}^{(0)}_\mu(t) +  {t}^{(1)}_\mu(t) +  {t}^{(2)}_\mu(t) + ...
%\end{equation}
%\begin{equation}
%i\frac{d\;{t^{a.}_{i.}}^{(0)}}{dt} = \bra{\Psi^{a.}_{i.}}e^{-\hat{T}(0)}H_o\ket{\Psi_{cc}} = 0 ,
%\end{equation}
\begin{equation}
i\frac{d\;{t^{a.}_{i.}}^{(1)}}{dt} = \bra{\Psi^{a.}_{i.}}e^{-\hat{T}(0)}V(t)\ket{\Psi_{cc}} + \bra{\Psi^{a.}_{i.}}e^{-\hat{T}(0)}\lbrack H_o,T^{(1)}\rbrack\ket{\Psi_{cc}}
\end{equation}
The amplitude $t^{a.}_{i.}$ can be expressed in terms of its Fourier transform as: 
\begin{equation}
{t^{a.}_{i.}}^{(1)}(t) = \int_{-\infty}^{\infty} d\omega_1{X^{a.}_{i.}}^{(1)}(\omega_1 + i\Gamma)e^{(-i\omega_1 + \Gamma) t}
\end{equation}
where
\begin{equation}
{X^{a.}_{i.}}^{(1)}(\omega_1 + i\Gamma)  = \sum_{j.b.}{\left\{{(-A + (\omega_1 + i \Gamma) I )}^{-1}\right\}}^{j.b.}_{i.a.}{\lambda^{b.}_{j.}}^{(1)}(\omega_1),
\end{equation}
and  
\begin{equation}
{\lambda^{b.}_{j.}}^{(1)}(\omega_1) = \bra{\Psi^{b.}_{j.}}e^{-\hat{T}(0)}V(\omega_1)\ket{\Psi_{cc}}.
\end{equation}
$A$ is the coupled cluster Jacobian matrix defined as:
\begin{equation}
A^{j.b.}_{i.a.} = \bra{\Psi^{a.}_{i.}}e^{-\hat{T}}\lbrack H_o,\{{a}^\dagger_b.a_j.\}\rbrack\ket{\Psi_{cc}}
\end{equation}
The above process is repeated with $\lambda^{j.}_{b.}$ amplitudes to obtain the derivative expressions and the Fourier transform $Y$.
\begin{equation}
\frac{d\;{\lambda^{b.}_{j.}}^{(1)}}{dt} = i\bra{\Lambda}(\lbrack\lbrack H_o,\{{a}^\dagger_b.a_j.\}\rbrack,T^{(1)}\rbrack + \lbrack V(t),\{{a}^\dagger_b.a_j.\}\rbrack)\ket{\Psi_{cc}} + i\sum_{i.a.} {\lambda^{a.}_{i.}}^{(1)}A^{j.b.}_{i.a.} 
\end{equation}
\begin{equation} 
{Y^{a.}_{i.}}^{(1)}(\omega_1 + i\Gamma) = - \sum_{j.b.}{\eta^{b.}_{j.}}^{(1)}(\omega_1) + \sum_{k.c.}F^{k.c.}_{j.b.} {X^{c.}_{k.}}^{(1)}(\omega_1 + i\Gamma)) \times\{{(A + (\omega_1 + i\Gamma)I)}^{-1}\}^{i.a.}_{j.b.}
\end{equation}
where
\begin{equation}
{\eta^{b.}_{j.}}^{(1)}(\omega_1) = \bra{\Lambda}\lbrack V(\omega_1),\{{a}^\dagger_b.a_j.\}\rbrack\ket{\Psi_{cc}} ,
\end{equation}
and 
\begin{equation}
F^{k.c.}_{j.b.} = \bra{\Lambda}\lbrack\lbrack H_o,\{{a}^\dagger_b.a_j.\}\rbrack,\{{a}^\dagger_c.a_k.\}\rbrack\ket{\Psi_{cc}}  
\end{equation}
The expectation value of any time dependent operator $A$ in CC can be expanded in orders of perturbation.
\begin{equation}
\langle A \rangle = \bra{\Lambda}A\ket{\Psi_{cc}} + \sum_{i.a.}{\lambda^{a.}_{i.}}^{(1)}\bra{\Psi^{a.}_{i.}}e^{-\hat{T}(0)}A\ket{\Psi_{cc}} + \bra{\Lambda}\lbrack A,T^{(1)}\rbrack\ket{\Psi_{cc}} + .....
\end{equation}
Comparing this with equation (34), the CC linear response function can be determined.
\begin{equation}
\int_{-\infty}^{\infty}d\omega_1{\langle\langle A;V(\omega_1)\rangle\rangle}_{\omega_1 + i \Gamma}e^{(-i\omega_1 + \Gamma)t} \equiv \sum_{i.a.}{\lambda^{a.}_{i.}}^{(1)}\bra{\Psi^{a.}_{i.}}e^{-\hat{T}(0)}A\ket{\Psi_{cc}} + \bra{\Lambda}\lbrack A,T^{(1)}\rbrack\ket{\Psi_{cc}}
\end{equation}
In more general terms, the response function can be written as:
\begin{equation}
{\langle\langle A;B\rangle\rangle}_{\omega_1} = \sum_{i.a.} \bra{\Lambda}\lbrack A,\{{a}^\dagger_a.a_i.\}\rbrack\ket{ \Psi_{cc}}{X^{a.}_{i.}}(B,\omega_1) + \sum_{i.a.}\left\{\bra{\Lambda}\lbrack B,\{{a}^\dagger_a.a_i.\}\rbrack\ket{\Psi_{cc}} + \sum_{k.c.}F^{k.c.}_{i.a.}{X^{c.}_{k.}}(B,\omega_1)\right\}{X^{a.}_{i.}}(A,-\omega_1).
\end{equation}
where \centerline{${X^{a.}_{i.}}(B,\omega_1) = \sum_{j.b.}\left\{ {( -A + \omega_1I)}^{-1}\right\}^{j.b.}_{i.a.}\;B^{b.}_{j.},$}\\and \centerline{ $B^{b.}_{j.} = \bra{\Psi^{a.}_{i.}}e^{-\hat{T}(0)}B\ket{\Psi_{cc}}$ .}\\\\
Thus for calculating the response functions, we need to solve two sets of
linear equations for obtaining ${X^{a.}_{i.}}(B,\omega_1)$ and
${X^{a.}_{i.}}(A,-\omega_1)$.
%\newpage
%\paragraph{Optical rotation calculations}~\\
%Rosenfeld,\cite{Rosenfeld29} using semi-classical electrodynamic theory, showed
%that the induced electric dipole moment can be written as: \begin{equation}
%\langle\vec{\mu}\rangle = \alpha\vec{E} +
%\frac{1}{\omega}\textbf{G}^\prime\frac{\partial\vec{B}}{\partial t}
%\end{equation}
%where $\vec{E} $ and $\vec{B}$ are electric and magnetic field vectors
%respectively, $\alpha$ is the electric dipole polarizability tensor while
%$\textbf{G}^\prime$ tensor is the key quantity for calculating optical
%rotation.  \begin{equation}
%\textbf{G}^{\prime}_{xy}(\omega) = -\frac{2}{\hbar} Im\sum_{n \neq 0}\frac{\omega\; \bra{\psi_o}\mu_x\ket{\psi_n}\bra{\psi_n}m_y\ket{\psi_o}}{\omega^{2}_{n0}-\omega^2}
%\end{equation}
%where $\mu$ and $m$ are electric and magnetic dipole operators: $\mu = \sum_i
%r_i , m= \sum_ir_i\times p_i$ in atomic units. $\omega$ is the frequency of
%light, $\omega_{no}$ is the excitation energy for the $n^{th}$ state ($\psi_n$) and
%$Im$ means the imaginary part of the equation. The trace of this tensor is related
%to the specific rotation, usually denoted as ${\lbrack\alpha\rbrack}_\omega $ in
%deg dm$^{\text{-1}}$(g/mL)$^{\text{-1}}$. After averaging over all possible
%orientations of the molecule, the following expression is obtained\cite{Crawford06}. \begin{equation}
%{\lbrack\alpha\rbrack}_{\omega} = \frac{(72.0 \times 10^6){\hbar}^2 N_A\;\omega}{c^2{m_e}^2 M} \times \left[ \frac{1}{3}Tr(\textbf{G}^\prime)\right]
%\end{equation}
%where $\textbf{G}^\prime$ and $\omega$ are in atomic units, $N_A$ is Avogadro's number, c is the speed of light in (m/s), m$_{\text{e}}$ is the mass of electron (kg) and M is the molecular mass (amu). The $\textbf{G}^\prime$ tensor can be obtained from using the CCLR method.
Thus, we can use this linear response formalism and can calculate the $\textbf{G}^\prime$ tensor
using eq.(4) in which A is the dipole moment operator and B is the magnetic field. However, there 
can be different representations of an operator which are referred to as gauges. It is observed that 
in the truncated CCLR method if the length gauge representation of the dipole operator is used, i.e. $\mu \equiv r$ (in atomic units), the optical rotation results are origin dependent. This problem is overcome in CCLR by using the velocity
gauge representation where $\mu \equiv r\times p$. However, both these gauges are equivalent for 
exact wavefunctions. However, the velcity gauge calculated optical rotation does not decay to zero 
at the static limit, i.e. when the external field is zero, which is an unphysical result. 
To correct this problem, another representation called the modified velocity gauge (MVG)
proposed by Pedersen \cite{Pedersen04} is used which shifts the values obtained
by velocity gauge by its static limit. 

%\begin{equation}
%\textbf{G}^{\prime}(\omega) = \text{-Im}\langle\langle\mu;m\rangle\rangle \equiv \text{- Im}\langle\langle r;r\times p\rangle\rangle
%\end{equation} 
%The above results obtained using the length gauge representation ($\mu
%\equiv r$) above, however is origin dependent. This is fixed by using the velocity
%gauge representation where the dipole moment operator is written as a momentum operator ($\mu
%\equiv {\frac{p}{\omega}}$) . Length and velocity gauge representations are equivalent for exact 
%wavefunctions.
%\begin{equation}
%\text{Tr}{\langle\langle r;r\times p\rangle\rangle}_{\omega} = \frac{1}{\omega}\text{Tr}{\langle\langle p;r\times p\rangle\rangle}_{\omega} 
%\end{equation}
%However, the velocity gauge calculations does not decay to zero at the static
%limit, i.e when the external field is zero, which is an unphysical result. To
%overcome this, another representation called modified velocity gauge (MVG)
%proposed by Pedersen \cite{Pedersen04} is used which shifts the values obtained
%by velocity gauge by it's static limit value.

