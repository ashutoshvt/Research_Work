This chapter focuses on covering the necessary theoretical background 
for deriving the coupled cluster (CC) response functions and identifying
the challenges associated with these approaches.
As a prelude to the CC response theory, the formalism for obtaining ground state CC energies 
is discussed after a brief description of Hartree-Fock and Configuration Interaction models.
The next section derives the simplified gradient expressions of the CC energy and subsequently 
parametrizes the left hand CC wavefunction.
\section{Hartree-Fock Theory}
Central to the field of quantum mechanics is the equation proposed by Erwin Schr\"odinger 
in 1926 or the Schr\"odinger's equation \cite{Schrodinger26}. One of the biggest factors 
contributing to the rich diversity of quantum chemical theories is the fact that Schr\"odinger's equation (SE)
is exactly solvable for only one electron systems like hydrogen atom. As such, numerous approximations
have been proposed over the years to solve this equation for many electron systems.
The Hartree-Fock (HF) theory\cite{} is one of the simplest approximation in this regard
under the Born-Oppenheimer approximation\cite{} in non-relativistic regimes\cite{}. This
model attempts to transform the many body problem of SE into a one body problem where each electron
only interacts with a mean field created by the other electrons. For an N-electron system, the HF 
wavefunction is a Slater determinant (obeys Pauli's antisymmetry principle naturally) composed of N 
spin orbitals, i.e. the one electron solutions of the HF equation. Following is a brief outline
of the HF procedure: A Lagrangian is constructed with the HF energy (the expectation value of the 
Slater determinant) coupled with the constraint of the spin orbitals always remaining 
orthonormal to each other. The Lagrangian thus can be seen as a functional of the spin 
orbitals. In the next step, the variational principle is employed to set the first order 
change in the Lagrangian with respect to spin orbitals to zero. Finally, one obtains 
a non-linear eigenvalue equation where the spin orbitals are the eigenfunctions of 
the Fock operator which itself depends on the spin orbitals themselves. 
Thus the equation is solved in a self-consistent manner usually in conjunction with 
the linear combination of atomic orbital model (LCAO) approach, where the spin 
orbitals are expressed as a linear combination of basis functions and the 
coefficients are determined from the solving a generalized eigenvalue problem.
The eigenvectors with the lowest N eigenvalues are the occupied orbitals, usually 
denoted by symbols $i,j,k,l,..$. If a basis set of size $K$ is employed, the remaining $K-N$
orbitals are termed as unoccupied or virtual orbitals usually denoted by symbols $a,b,c,d,..$.
\\
Even with this mean field approach, one can extract almost 95-98\% of the total 
electronic energy. The remaining energy, also known as the correlation energy,
however, is very essential if one wants to attain ``chemical accuracy", ex.  
1kcal/mol for interaction energies. Inspite of this, the simple yet robust 
structure of the HF procedure makes it one of the most popular reference 
wavefunctions for more complicated and accurate theoretical models.
\subsection{Configuration Interaction Method}
Configuration interaction (CI) method attempts to solve the time independent 
Schr\"odinger's (TISE) by recasting it into a matrix eigenvalue problem, 
\begin{equation}
HC = EC
\end{equation}
where H is the matrix representation of the Hamiltonian, C is the coefficient matrix
whose columns are the eigenvectors of the Hamiltonian and E is the diagonal matrix 
containing the eigenvalues or electronic energies (The lowest energy corresponds to 
the ground state wavefunction). If the vector space used to represent the Hamiltonian 
is complete, the CI approach, then popularly known as the Full CI method gives exact 
wavefunctions and electronic energies. Specifically, this model uses a vector space 
composed of substituted or excited Slater determiants which can form a complete
space in the limit of an infinite one electron basis. Furthermore, this vector space 
also have the desired properties of antisymmetry and orthogonality. Indeed the Full 
CI wavefunction is a linear combination of all possible Slater determinants for a given basis set,
\\
\begin{equation}
\ket{\Psi} = c_o\ket{\Phi_o} + \sum_{ia}{c}_{i}^{a}\ket{{\Phi}^a_i} + \sum_{i>j,a>b}c_{ij}^{ab}\ket{\Psi^{ab}_{ij}} + \sum_{i>j>k,a>b>c}c_{ijk}^{abc}\ket{\Psi^{abc}_{ijk}} + ...
\end{equation}
\\
where $\ket{\Phi_o}$ is the reference wavefunction (usuallly HF), $\ket{{\Phi}^a_i}$ refers to a singly excited determinant where an occupied orbital $i$ of the reference wavefunction is replaced by a virtual orbital $a$
and so on.\\
However, Full CI method is computationally very expensive and scales factorially with 
system size. Unsurprsingly, it is mostly used in benchmark calculations of small molecules and
the largest Full CI calculation to date has been on nitrogen molecule\cite{} involving 
billions of determinants. In practice, truncated CI methods like CISD, CISDT where the vector 
space is restricted to include up to doubly and triply excited determinants 
is used. Unlike the exact wavefunction, truncated CI wavefunctions, however, 
do not posess the property of size-extensivity i.e. the CIS[D/T] energy doesn't scale 
linearly with the number of electrons in the asymptotic limit. Another notable deficiency 
is the lack of size-consistency which means that within these approaches, the sum of the 
energies of non-interacting fragments (each calculated separately) is not equal to the energy 
of the supermolecule when all the fragments are included in the calculations.
As a result, the accuracy of these methods decrease progressively as the size of the system increases.
%.These methods can extract up to 95\% of correlation energy\cite{HarrisonHandy83} and
%can be applied to solve for excited states, open-shell systems and systems that
%are far from their equilibrium geometry, which makes them very useful.
\subsection{Coupled Cluster Theory} 
%\paragraph{Coupled cluster theory} ~ \\\\
Coupled cluster (CC) theory\cite{} is an alternative formulation of TDSE which attempts to 
reproduce the Full CI wavefunction through an exponential parametrization
of the wavefunction. The CC wavefunction can be obtained by the operation of cluster
operators $\hat{T}$ acting on the reference Slater determinant. 
%is one of the most accurate yet computationally
%affordable method, which is very widely used in quantum chemistry.CC
%wavefunction is an exponential ansatz\cite{Crawford00} which incorporates the electron
%correlation effects through cluster operators 
%$\hat{T}$.  
\\
\begin{equation}
\ket{\Psi_{CC}} = e^{\hat{T}}\ket{\Psi_{o}} , 
\end{equation}
where,
\begin{equation}
 \hat{T} = \hat{T_1} + \hat{T_2} + \hat{T_3} + ... \;\hat {T_n}.
\end{equation}
\\
Here $\ket{\Psi_{o}}$ is the reference wavefunction, usually taken as the HF
wavefunction. One of the most popular tools used for the derivation of the 
complicated CC equations is second-quantization\cite{JorgensenSimons81}.
The $\hat{T_2}$ operator in SQ form can be written as
\\
\begin{equation}
\hat{T_2} = \frac{1}{4}\sum_{ijab}t^{ab}_{ij}\{a^\dagger_aa^\dagger_ba_ja_i\}
\end{equation}
\\
%\begin{equation}
%\hat{T_1} = \sum_{ia}t^a_i\{{a}^\dagger_a a_i\}
%\end{equation}
where ${a}^\dagger_a$ (or ${a}^\dagger_b)$ is called a creation operator as it 
creates a new particle state (virtual orbital) when it acts on a Slater determinant.
\begin{equation}
{a}^\dagger_a\ket{\phi_b...\phi_d} = \ket{\phi_a\phi_b...\phi_d}
\end{equation}
Here, $\ket{\phi_a\phi_b...\phi_d}$ is a shorthand notation (Dirac) for a Slater determinant
with orbitals $a,b,..d$. The $a_i$ (or $a_j$) operator on other hand is called 
as annihilation operator as it removes a hole state (occupied orbital) when it 
acts on a Slater determinant.
\begin{equation}
a_i\ket{\phi_i\phi_j...\phi_l} = \ket{\phi_j...\phi_l}
\end{equation}
Thus, the action of the $\hat{T_2} $ operator on a Slater determinant creates a 
linear combination of all doublly excited determinants with corresponding coefficients
$t^{ab}_{ij}$ which can be seen as the contribution of virtual orbitals $a$ and $b$
to the pair correlation function $f_{ij}$ which correlates the motions of any two 
electrons associated with occupied orbitals $i$ and $j$. 
\\
\begin{equation}
\begin{split}
& \hat{T_2} = \sum_{ij}f_{ij} \\
& f_{ij} = \frac{1}{4}\sum_{ab}t^{ab}_{ij}\{a^\dagger_aa^\dagger_ba_ja_i\}\\
\end{split}
\end{equation}
\\
Similarly, the ${\hat{T_3}}$ operator correlates the motion of all triplets of electrons. 
The $\hat{T_1}$ operator on the other hand are one electron operators that capture the 
``adjustment of the one-electron basis"\cite{} as the effect of other correlation operators 
are added to the wavefunction.
\\
\begin{equation}
\hat{T_1} = \sum_{ia}t^a_i\{{a}^\dagger_a a_i\}
\end{equation}
\\
In general, the structure of these cluster operators can be shown as,
\\
\begin{equation}
\hat{T_n} = {\bigg(\frac{1}{n!}\bigg)}^2\sum_{ij..ab..}^nt^{ab..}_{ij..}\{a^\dagger_aa^\dagger_b...a_ja_i\}
\end{equation}
\\
%\subsubsection{Coupled Cluster energy}
%\paragraph{Coupled Cluster energy}~ \\
Expanding the ``exponential ansatz" of the CC wavefunction,
\\
\begin{equation}
\ket{\Psi_{CC}} = (1+(\hat{T_1} + \hat{T_2} + \hat{T_3} + ... ) + \frac{1}{2!}{(\hat{T_1} + \hat{T_2} + \hat{T_3} + ...)}^2 + ... )\ket{\Psi_o}
\end{equation}
\\
a linear combination of all possible Slater determinants is obtained, which in the basis set limit,  
should be an exact solution of the TISE just like the Full CI wavefunction. 
\\
\begin{equation}
\hat{H}e^{\hat{T}}\ket{\Psi_o} = E_{cc}\;e^{\hat{T}}\ket{\Psi_o}
\end{equation}
\\
The Hamiltonian is also expressed in second-quantized form \cite{Crawford00}:
\\
\begin{equation}
\hat{H} = \sum_{pq}h_{pq}\{a^\dagger_pa_q\} + \frac{1}{4}\sum_{pqrs}\bra{pq}\ket{rs}\{a^\dagger_pa^\dagger_qa_sa_r\}
\end{equation}
\\
where, $h_{pq} = \langle h_{pq} \rangle$ and $\langle pq||rs \rangle = \langle pq|rs \rangle - \langle pq|sr \rangle$ are the one and two electron components of the Hamiltonian repectively.
The CC equations for calculating the amplitudes and the energy can be  
obtained by multiplying the above equation by the inverse of the exponential operator i.e
$e^{-\hat{T}}$ and projecting it against the reference and excited
determinants.
\\
\begin {equation}
\bra{\Psi_o}e^{-\hat{T}}\hat{H}e^{\hat{T}}\ket{\Psi_o} = E_{cc}
\end{equation}
\begin{equation}
%\bra{\Psi^{ab..}_{ij..}}e^{-\hat{T}}\hat{H}e^{\hat{T}}\ket{\Psi_o} = E\cancelto{0}{\bra{\Psi^{ab..}_{ij..}}\Psi_o\rangle} = 0 .
\bra{\Psi^{a.}_{i.}}e^{-\hat{T}}\hat{H}e^{\hat{T}}\ket{\Psi_o} = 0 .
\end{equation} 
\\
Here $\Psi^{a.}_{i.}$ can refer to any excited Slater determinant: singles, doubles etc. 
The similarity transformed Hamiltonian $e^{-\hat{T}}\hat{H}e^{\hat{T}}$, also written as 
written as $\bar{H}$ can be expressed in terms of commutators of the Hamiltonian with the 
cluster operators $\hat{T}$ by using the Campbell-Baker-Hausdorff formula\cite{Merzbacher70}.
\\
\begin{equation}
e^{-\hat{T}}\hat{H}e^{\hat{T}} = \hat{H} + \lbrack\hat{H},\hat{T}\rbrack + \frac{1}{2!}\lbrack\lbrack\hat{H},\hat{T}\rbrack,\hat{T}\rbrack + \frac{1}{3!}\lbrack\lbrack\lbrack\hat{H},\hat{T}\rbrack,\hat{T}\rbrack,\hat{T}\rbrack + \frac{1}{4!}\lbrack\lbrack\lbrack\lbrack\hat{H},\hat{T}\rbrack,\hat{T}\rbrack,\hat{T}\rbrack,\hat{T}\rbrack + ...
\end{equation}
\\
The expansion truncates naturally at the four nested commutator term since the Hamiltonian is at most a two 
electron operator and the cluster operators commute among themselves\cite{}.
%Also, the second quantized form of the Hamiltonian can be written as\cite{Crawford00}:
%\begin{equation}
%\hat{H} = \sum_{pq}h_{pq}\{a^\dagger_pa_q\} + \frac{1}{4}\sum_{pqrs}\bra{pq}\ket{rs}\{a^\dagger_pa^\dagger_qa_sa_r\}
%\end{equation}
Invoking Wick's theorem\cite{Wick50}, the CC energy equation gets simplified as:
\\
\begin{equation}
E_{cc} = E_o + \sum_{ia}f_{ia}t^a_i + \frac{1}{4}\sum_{aibj}\bra{ij}\ket{ab}t^{ab}_{ij} + \frac{1}{2}\sum_{aibj}\bra{ij}\ket{ab}t^a_it^b_j
\end{equation} 
\\
The non-linear amplitude equations are solved iteratively until the change in energies
fall below a convergence threshold. However, just like its CI counterpart, Full CC is 
computationally impractical and hence truncated CC methods like CCSD:\;$\hat{T} =
\hat{T_1} + \hat{T_2}$, CCSDT:\;$\hat{T} = \hat{T_1} + \hat{T_2} + \hat{T_3}$ 
are used. The exponential structure of the CC wavefunction makes the truncated CC methods 
more efficient and accurate than the corresponding linear CI methods. For example,
the CCSD wavefunction implicitly includes the triples and quadruples excitation contributions to 
its singles and doubles amplitude equations because of the products of cluster operators like 
$\hat{T_1}\hat{T_2}$, ${(\hat{T_2})}^2$ etc., unlike the CISD method which can only include 
singles and doubles excitation contributions. Furthermore, CC energies have the desired 
properties of size-extensivity and size-consistency (provided the reference wavefunction is
size-consistent). Unsurprisingly, the CCSD(T)\cite{Shen12} method, which approximates the triples using
perturbation theory is considered to be the ``gold standard" of quantum chemistry.
%Also, one of the major advantages with the CC truncation methods is their
%size-extensivity which makes them very useful. 
However, CC methods just like their CI counterparts, are computationally expensive: 
CCSD scales as $O(N^6)$, CCSD(T) as $O(N^7)$, CCSDT as $O(N^8)$ and so on, where $N$ 
is the number of basis functions.\\
%and is routinely used for accurate results only for small molecules. While CCSD(T)
%gives very good results for molecules at their equilibrium geometry, it fails
%to describe diradical species and bond-breaking. CCSDT and CCSDTQ methods are
%used virtually exclusively for high accuracy calculations of small molecules
%as they are very computationally expensive.
%\paragrpah{\bf{CC analytic derivatives}} ~\\\\
\subsection{Coupled Cluster Analytic Gradients}
%\paragraph{CC analytic derivatives} ~ \\\\
%Molecular properties like dipole
%moments, IR intensities, force constants etc., depend upon the gradients of the
%molecular energy with respect to external perturbations. 
%In this section, a derivation of first order gradient expressions of the CC method is presented.\\
%If we take the derivative of the CC energy directly with respect to any perturbation 
%X,
%, we get
%\begin{equation}
%\frac{\partial{E_{cc}}}{\partial X} = \bra{\phi_0}\frac{\partial{\bar H}}{\partial X}\ket{\phi_0},
%\end{equation}
%If we use the above equation for calculating gradients, 
%we need to calculate
%$\frac{\partial{t^a_i}}{\partial X}, \frac{\partial{t^{ab}_{ij}}}{\partial X}$
%, we need to calculate the derivatives of non-linear amplitude equations, 
%which makes it very computationally expensive. As a result, a different 
%approach is used where one needs to solve some linear equations i.e. 
%the Lambda equations, which are independent of the perturbation. 
%In this section, a derivation of first order gradient expressions 
%of CC method is presented using this approach.\\
The gradient of the CC energy (equation 2.14) with respect to any external perturbation X 
can be written as:
\\
\begin{equation}
\frac{\partial{E_{cc}}}{\partial X} = \bra{\Psi_o}\frac{\partial{\bar H}}{\partial X}\ket{\Psi_o} = \bra{\Psi_o}{\bar{H}}^X + \lbrack\bar H , \frac{\partial{\hat T}}{\partial X}\rbrack\ket{\Psi_o} 
\end{equation}
where, 
\begin{equation}
{\bar{H}}^X =  e^{-\hat{T}}\frac{\partial{\hat H}}{\partial X}e^{\hat{T}}.
\end{equation}
\\
Invoking the resolution of identity (RI),
\\
\begin{equation}
 1 = \ket{\Psi_o}\bra{\Psi_o} + \sum_{ia}\ket{{\Psi}^a_i}\bra{{\Psi}^a_i} + \frac{1}{4}\sum_{ijab}\ket{{\Psi}^{ab}_{ij}}\bra{{\Psi}^{ab}_{ij}} + ...
\end{equation}
\\
equation (2.18) can be seen to involve the derivatives of the amplitudes i.e $\frac{\partial{t^{a.}_{i.}}}{\partial X}$
\\
\begin{equation}
\frac{\partial{E_{cc}}}{\partial X} = \bra{\Psi_o}{\bar{H}}^X\ket{\Psi_o} + \sum_{i.a.}\bra{\Psi_o}\bar{H}\ket{{\Psi}^{a.}_{i.}} \frac{\partial{t^{a.}_{i.}}}{\partial X}
%\bra{{\Psi}^{a.}_{i.}}\frac{\partial{\hat T}}{\partial X}\ket{\Psi_o}.
\end{equation}
\\
Calculating gradients using this approach would require taking the derivative of the non-linear equations
used to solve for the amplitudes, which could be computationally demanding. However, explicit gradient 
calculations of the amplitudes can be avoided altogether through an alternative formulation.
Taking the derivative of the CC amplitude equations with respect to X, 
\\
\begin{equation} 
0 = \bra{{\Psi}^{a.}_{i.}}{\bar{H}}^X + \lbrack\bar H , \frac{\partial{\hat T}}{\partial
X}\rbrack\ket{\Psi_o}.
\end{equation} 
\\
Using the RI method again, the above equation can be simplified.
\\
\begin{equation}
\sum_{j.b.}\bra{{\Psi}^{a.}_{i.}}\bar{H}-E_{cc}\ket{{\Psi}^{b.}_{j.}} \frac{\partial{t^{b.}_{j.}}}{\partial X}
= - \bra{{\Psi}^{a.}_{i.}}{\bar{H}}^X\ket{\Psi_o}
%\bra{{\Psi}^{b.}_{j.}}\frac{\partial{\hat T}}{\partial X}\ket{\Psi_o} = - \bra{{\Psi}^a_i}{\bar{H}}^X\ket{\Psi_o} 
\end{equation}
or,
\\
\begin{equation} 
%\bra{{\Psi}^{b.}_{j.}}\frac{\partial{\hat T}}{\partial X}\ket{\Psi_o} 
\frac{\partial{t^{b.}_{j.}}}{\partial X} = -\sum_{i.a.}
\bra{{\Psi}^{b.}_{j.}}{(\bar{H}-E_{cc})}^{-1}\ket{{\Psi}^{a.}_{i.}}\bra{{\Psi}^{a.}_{i.}}{\bar{H}}^X\ket{\Psi_o}
\end{equation}
\\
%The above equation was derived by considering the gradient of just the singles equation.
%Including the gradient of all the other amplitude equations and plugging the modified 
Plugging this gradient expression back into equation (2.21), we obtain the following equation,
\\
\begin{equation}
\frac{\partial{E_{cc}}}{\partial X} = \bra{\Psi_o}{\bar{H}}^X\ket{\Psi_o} -
\sum_{i.a.}\bra{\Psi_o}\bar{H}\ket{{\Psi}^{a.}_{i.}}\sum_{j.b.}
\bra{{\Psi}^{a.}_{i.}}{(\bar{H}-E_{cc})}^{-1}\ket{{\Psi}^{b.}_{j.}}
\bra{{\Psi}^{b.}_{j.}}{\bar{H}}^X\ket{\Psi_o}.
\end{equation}
\\
The second term of the RHS of the above equation involves an inverted 
$\bar{H}$ matrix which needs to be avoided at all costs as the dimensions
of this matrix can be very large. As such, a linear operator $\hat{\Lambda}$
is defined in the following manner.
\\
\begin{equation}
\bra{\Psi_o}\hat{\Lambda}\ket{\Psi^{b.}_{j.}} = -\sum_{i.a.}\bra{\Psi_o}\bar{H}\ket{{\Psi}^{a.}_{i.}}
\bra{{\Psi}^{a.}_{i.}}{(\bar{H}-E_{cc})}^{-1}\ket{{\Psi}^{b.}_{j.}}.
\end{equation}
\\
or, 
\begin{equation}
\sum_{i.a.}\bra{\Psi_o}\hat{\Lambda}\ket{\Psi^{a.}_{i.}}\bra{{\Psi}^{a.}_{i.}}{(\bar{H}-E_{cc})}\ket{{\Psi}^{b.}_{j.}}
 = - \bra{\Psi_o}\bar{H}\ket{{\Psi}^{b.}_{j.}}
%\bra{{\Psi}^{a.}_{i.}}{(\bar{H}-E_{cc})}^{-1}\ket{{\Psi}^{b.}_{j.}}.
\end{equation}
\\
It can be easily seen from above equations that $\hat{\Lambda}$ is a linear de-excitation operator,
\begin{equation}
\hat{\Lambda} = \hat{\Lambda}_1 + \hat{\Lambda}_2 + \hat{\Lambda}_3 + ...
\end{equation} 
\\
where $\hat{\Lambda}_1 = \sum\limits_{ia}\lambda^i_a\{{a}^\dagger_i a_a\}$, $\hat{\Lambda}_2=\sum\limits_{ijab}
\lambda^{ij}_{ab}\{{a}^\dagger_i {a}^\dagger_j a_b a_a\}$ 
are the singles and doubles de-excitation operators respectively. The CC gradient expression can now be expressed 
in terms of the lambda operator,
\\
\begin{equation}
\frac{\partial{E_{cc}}}{\partial X} = \bra{\Psi_o}(1 + \hat{\Lambda}){\bar{H}}^X\ket{\Psi_o} = \bra{\Psi_o}(1 + \hat{\Lambda})e^{-\hat{T}}\frac{\partial{\hat H}}{\partial X}\ket{\Psi_{cc}}
\end{equation}
\\
and the governing equation for sloving the lambda amplitudes eq. (2.27) can be written in a more compact form as,
\\
\begin{equation}
\bra{\Psi_o}(1 + \hat{\Lambda})(\bar{H} - E_{cc})\ket{{\Psi}^{a.}_{i.}} = 0.
\end{equation}
\\
Thus instead of taking the gradients of the non-linear $T$ amplitude
equations with respect to perturbations, we solve linear perturbation-independent 
$\hat{\Lambda}$ equations for calculating the CC energy gradient.
Furthermore, CC gradients satisfy the generalized Hellman-Feynman equation\cite{Feynman39}
if $\bra{\Psi_o}(1 + \hat{\Lambda})e^{-\hat{T}}$ is defined as the CC left hand wavefunction
as due to the non-hermiticity of the $\bar{H}$ operator or the non-variational nature of the 
CC method, the left and right hand wavefunctions are not simple hermitian conjugates of each other.
%However, for second order derivatives, we do need to calculate the first derivatives of 
%either the $T$ amplitudes with respect to a perturbation.
%The derivative of the similarity transformed hamiltonian can be expressed as:
%\begin{equation}
%\frac{\partial{\bar{H}}}{\partial X} = {\bar{H}}^X + \lbrack\bar H , \frac{\partial{\hat T}}{\partial X}\rbrack, 
%\end{equation}

%\\By combining the CC energy and amplitude equations, we can write:
%\begin{equation}
%E_{CC} = \bra{\Lambda}\hat{H}\ket{\Psi_{CC}}
%\end{equation}
%where
%\begin{equation}
%\bra{\Lambda} = \bra{\Psi_o} + \sum_{\mu}{\zeta_{\mu}}\bra{\mu}e^{-\hat{T}}
%\end{equation} In the above equation, $\bra{\mu}$ represents any excited Slater determinant.
%$\zeta_{\mu}$ can be obtained if the bra state (Lambda) obeys the schr\"odinger's
%equation.
%\begin{equation}
%\bra{\Lambda}\hat{H}e^{\hat{T}} = \bra{\Lambda}e^{\hat{T}}E_{cc}
%\end{equation}
%Right projecting the above equation onto the subspace \{$\ket{\nu}$\} (excited Slater determinants):
%\begin{equation}
%\sum_{\mu}\zeta_{\mu}A_{\mu\nu} = -\bra{\Psi_o}\lbrack\hat{H},\hat{\tau_{\nu}}\rbrack\ket{\Psi_{cc}}
%\end{equation}
%where $\hat{\tau_{\nu}}$ is an excitation operator:$\;\;\;\hat{\tau_{\nu}}\ket{\Psi_o} = \ket{\nu}$ , and 
%\begin{equation}
%A_{\mu\nu} = \bra{\mu}e^{-\hat{T}}\lbrack\hat{H},\hat{\tau_{\nu}}\rbrack\ket{\Psi_{cc}}
%\end{equation}
%We can solve for the parameters $\zeta_{\mu}$ from the above equation to get $\bra{\Lambda}.$ In the presence of a time-independent perturbation described by $\alpha \hat{V}$ with $\alpha$ as the strength parameter, the gradient of coupled cluster energy with respect to $\alpha$ at zero perturbation strength can be shown as:
%\begin{equation}
%\frac{\partial{E_{cc}}}{\partial {\alpha}}|_{\alpha=0} \;\;= \frac{\partial}{\partial{\alpha}}\bra{\Lambda (\alpha)}\hat{H}+\alpha \hat{V}\ket{\Psi_{CC}(\alpha)}|_{\alpha=0}\;\; = \;\;\;\bra{\Lambda}\hat{V}\ket{\Psi_{CC}}
%\end{equation}
%We are able to obtain the first order gradients of CC just by solving the linear
%lambda equations to get the $\zeta_{\mu}$ parameters and plugging them in the
%above expression. The $\bra{\Lambda}$ state is nothing the left hand coupled cluster wave
%function, as it is a solution to the Schr\"odinger equation and the above equation satisfies the 
%generalized Hellmann-Feynman theorem\cite{Feynman39}.
%%For full CC, $\bra{\Lambda}$ and $\ket{\Psi_{cc}}$ are just hermitian
%%conjugate, but they are different in case of truncated CC methods because %of
%%the non-hermiticity of $\bar{H}$ operator.
%\subsubsection{Coupled Cluster Linear Response (CCLR)}
%\paragraph{Exact states}~\\
%\paragraph{Coupled cluster linear response (CCLR)}~\\\\
\subsection{Coupled Cluster Linear Response (CCLR) Theory}
CCLR method proposed by Koch and J{\o}rgensen in 1990\cite{Koch90} is a recipe for accurate
calculations of response properties like dynamic polarizabilities, optical
rotations, etc. In this section, we will outline important steps for the
derivations of both general and CCLR response functions and also talk about
different gauge representations used for calculating the response functions.\\
The operator $V(t)$ which describes the interaction between the molecule and 
an external time-dependent field can be expressed in the frequency domain as: 
\begin{equation}
V(t) = \int_{-\infty}^{\infty}d\omega\;\;V(\omega) e^{(-i\omega + \Gamma)t} ,
\end{equation}
The full Hamiltonian can then be written as: $H = H_o + V(t)$, where $H_o$ is the
time independent unperturbed Hamiltonian. Assuming that $\Psi_o$ is an eigenstate of $H_o$
%An exact wavefunction satisfies the Schr\"odinger equation: $H_o\ket{O} =
%E_o\ket{O}$ where $H_o$ is the time independent Hamiltonian and $\ket{O}$ is an
%eigenstate of the system. 
and that the molecule is in state $\Psi_o$ when the perturbation starts at $t = -\infty$, 
the $\ket{\Psi_o}$ state evolves in time as $\ket{\Psi_o(t)}$ according to the time-dependent 
Schr\"odinger equation.
\begin{equation}
i\frac{d}{dt}\ket{\Psi_o(t)} = (H_o + V^t)\ket{\Psi_o(t)}
\end{equation}
%It can be shown that the time dependent state $\ket{\Psi_o(t)}$ is related to 
Following the work of Olsen\cite{Olsen85}, the time dependent state $\ket{\Psi_o(t)}$
can be written as:
\begin{equation}
\ket{\Psi_o(t)} = \ket{\bar{\Psi_o}}e^{i\epsilon(t)}
\end{equation}
where $\epsilon$ is a phase factor and the state $\ket{\bar{\Psi_o}}$ can be expressed as\cite{Koch90}:
%using perturbation theory. Specifically,
\begin{equation}
\ket{\bar{\Psi_o}} = \ket{\Psi_o} + {\ket{\Psi_o}}^{(1)}+ {\ket{\Psi_o}}^{(2)} + ...
\end{equation}
where first order correction ${\ket{\Psi_o}}^{(1)}$ and others can be determined 
from the time dependent perturbation theory. For calculating any response property, the
expectation value of the respective time independent  operator is calculated in
the presence of an external field. The response functions can be seen as the
coefficients of terms which appear in different orders of perturbation in $V(t)$
in the expansion of the expectation value.\cite{Koch90} 
%Furthermore, the expectation value of any time dependent operator $A$ can
%expanded as\cite{Koch90} : 
\begin{equation}
\bra{\Psi_o(t)}A\ket{\Psi_o(t)} = \bra{\Psi_o}A\ket{\Psi_o} + \int_{-\infty}^{\infty}d\omega_1{\langle\langle A;V({\omega_1})\rangle\rangle}_{\omega_1 + i\Gamma}e^{(-i\omega_1 + \Gamma)t} + .....
\end{equation}
where we only considered the linear response function $\langle\langle A;V({\omega_1})\rangle\rangle$ 
and neglected the higher order terms. If the perturbation field is composed of a 
single frequency ($\omega_1$) , the linear response function can be written as:
\begin{equation}
\langle\langle A;V({\omega_1})\rangle\rangle = \sum_k\{
\frac{\bra{\Psi_o}A\ket{\Psi_k}\bra{\Psi_k}V({\omega_1})\ket{\Psi_o}}{\omega_1
- \omega_k + i\Gamma}  -
\frac{\bra{\Psi_o}V({\omega_1})\ket{\Psi_k}\bra{\Psi_k}A\ket{\Psi_o}}{\omega_1 + \omega_k +
i\Gamma}\} \end{equation}
%The above equation is called the sum of states (SOS) equation , 
where $\omega_k$ is the excitation energy between the states $\Psi_o$ and $\Psi_k$ 
and the summation runs over all the solutions of the time independent Schr\"odinger equation $\Psi_k$.
It can be seen from eq.(2) that $\textbf{G}^{\prime}(\omega)$ is nothing but the imaginary part of this 
response function if we take $A$ as the dipole moment operator with $V$ as a magnetic field.
However, this approach is very computationally expensive for CC methods and we approximate the above 
linear response function using CCLR theory.
%\paragraph{CC linear response function}~\\\\
Using the same approach as above, the right and left hand CC wavefunction evolve in time as\cite{Koch90}:
\begin{equation}
\ket{\Psi_{cc}(t)} = e^{\hat{T}(t)}\ket{\Psi_o}e^{i\epsilon(t)}
\end{equation}
\begin{equation} \bra{\Lambda(t)} = \{\bra{\Psi_o} +
\sum_{i.a.}\lambda^{a.}_{i.}(t)\bra{\Psi^{a.}_{i.}}e^{-\hat{T}(t)}\}e^{-i\epsilon(t)}
\end{equation}
where $e^{\pm i\epsilon(t)}$ is a time dependent phase factor
and we have time dependent $T$ and $\lambda^{a.}_{i.}$ amplitudes. The governing equation for 
the time dependence of these amplitudes is the time-dependent Schr\"odinger equation.
\begin{equation}
i\frac{d}{dt}\ket{\Psi_{cc}(t)} = (H_o + V(t))\ket{\Psi_{cc}(t)} 
\end{equation}
\begin{equation}
\frac{d}{dt}\bra{\Lambda(t)} = i\bra{\Lambda(t)}(H_o + V(t))
\end{equation}
On multiplying eq. (38) by $e^{-\hat{T}(t)}$ on both sides and projecting 
it against $\bra{\Psi^{a.}_{i.}}$, we obtain the expression for the time evolution 
of $t^{a.}_{i.}$ amplitudes.
\begin{equation}
\frac{dt^{a.}_{i.}}{dt} = -i\bra{\Psi^{a.}_{i.}}e^{-\hat{T}(t)}(H_o + V(t))e^{\hat{T}(t)}\ket{\Psi_o}
\end{equation}
On multiplying eq.(38) by $e^{i\epsilon(t)}$ and invoking the RI, the expression for 
time evolution of $\lambda^{b.}_{j.}$ amplitudes can be obtained:
\begin{equation}
\frac{d\lambda^{b.}_{j.}}{dt} =
i\bra{{\tilde{\Lambda}}(t)}\lbrack H_o + V(t),\{{a}^\dagger_b.
a_j.\}\rbrack\ket{{\widetilde{\Psi}_{cc}}(t)}
\end{equation} 
where $\sim$ denotes the usual expressions without the phase factor $e^{\pm i\epsilon(t)}$. 
Expanding the $t^{a.}_{i.}$ amplitudes in orders of perturbation of V(t), the 
expressions for the different orders can be obtained using eq. (38). The first order
derivative can be written as:
%\begin{equation}
%t_\mu(t) = {t}^{(0)}_\mu(t) +  {t}^{(1)}_\mu(t) +  {t}^{(2)}_\mu(t) + ...
%\end{equation}
%\begin{equation}
%i\frac{d\;{t^{a.}_{i.}}^{(0)}}{dt} = \bra{\Psi^{a.}_{i.}}e^{-\hat{T}(0)}H_o\ket{\Psi_{cc}} = 0 ,
%\end{equation}
\begin{equation}
i\frac{d\;{t^{a.}_{i.}}^{(1)}}{dt} = \bra{\Psi^{a.}_{i.}}e^{-\hat{T}(0)}V(t)\ket{\Psi_{cc}} + \bra{\Psi^{a.}_{i.}}e^{-\hat{T}(0)}\lbrack H_o,T^{(1)}\rbrack\ket{\Psi_{cc}}
\end{equation}
The amplitude $t^{a.}_{i.}$ can be expressed in terms of its Fourier transform as: 
\begin{equation}
{t^{a.}_{i.}}^{(1)}(t) = \int_{-\infty}^{\infty} d\omega_1{X^{a.}_{i.}}^{(1)}(\omega_1 + i\Gamma)e^{(-i\omega_1 + \Gamma) t}
\end{equation}
where
\begin{equation}
{X^{a.}_{i.}}^{(1)}(\omega_1 + i\Gamma)  = \sum_{j.b.}{\left\{{(-A + (\omega_1 + i \Gamma) I )}^{-1}\right\}}^{j.b.}_{i.a.}{\lambda^{b.}_{j.}}^{(1)}(\omega_1),
\end{equation}
and  
\begin{equation}
{\lambda^{b.}_{j.}}^{(1)}(\omega_1) = \bra{\Psi^{b.}_{j.}}e^{-\hat{T}(0)}V(\omega_1)\ket{\Psi_{cc}}.
\end{equation}
$A$ is the coupled cluster Jacobian matrix defined as:
\begin{equation}
A^{j.b.}_{i.a.} = \bra{\Psi^{a.}_{i.}}e^{-\hat{T}}\lbrack H_o,\{{a}^\dagger_b.a_j.\}\rbrack\ket{\Psi_{cc}}
\end{equation}
The above process is repeated with $\lambda^{j.}_{b.}$ amplitudes to obtain the derivative expressions and the Fourier transform $Y$.
\begin{equation}
\frac{d\;{\lambda^{b.}_{j.}}^{(1)}}{dt} = i\bra{\Lambda}(\lbrack\lbrack H_o,\{{a}^\dagger_b.a_j.\}\rbrack,T^{(1)}\rbrack + \lbrack V(t),\{{a}^\dagger_b.a_j.\}\rbrack)\ket{\Psi_{cc}} + i\sum_{i.a.} {\lambda^{a.}_{i.}}^{(1)}A^{j.b.}_{i.a.} 
\end{equation}
\begin{equation} 
{Y^{a.}_{i.}}^{(1)}(\omega_1 + i\Gamma) = - \sum_{j.b.}{\eta^{b.}_{j.}}^{(1)}(\omega_1) + \sum_{k.c.}F^{k.c.}_{j.b.} {X^{c.}_{k.}}^{(1)}(\omega_1 + i\Gamma)) \times\{{(A + (\omega_1 + i\Gamma)I)}^{-1}\}^{i.a.}_{j.b.}
\end{equation}
where
\begin{equation}
{\eta^{b.}_{j.}}^{(1)}(\omega_1) = \bra{\Lambda}\lbrack V(\omega_1),\{{a}^\dagger_b.a_j.\}\rbrack\ket{\Psi_{cc}} ,
\end{equation}
and 
\begin{equation}
F^{k.c.}_{j.b.} = \bra{\Lambda}\lbrack\lbrack H_o,\{{a}^\dagger_b.a_j.\}\rbrack,\{{a}^\dagger_c.a_k.\}\rbrack\ket{\Psi_{cc}}  
\end{equation}
The expectation value of any time dependent operator $A$ in CC can be expanded in orders of perturbation.
\begin{equation}
\langle A \rangle = \bra{\Lambda}A\ket{\Psi_{cc}} + \sum_{i.a.}{\lambda^{a.}_{i.}}^{(1)}\bra{\Psi^{a.}_{i.}}e^{-\hat{T}(0)}A\ket{\Psi_{cc}} + \bra{\Lambda}\lbrack A,T^{(1)}\rbrack\ket{\Psi_{cc}} + .....
\end{equation}
Comparing this with equation (34), the CC linear response function can be determined.
\begin{equation}
\int_{-\infty}^{\infty}d\omega_1{\langle\langle A;V(\omega_1)\rangle\rangle}_{\omega_1 + i \Gamma}e^{(-i\omega_1 + \Gamma)t} \equiv \sum_{i.a.}{\lambda^{a.}_{i.}}^{(1)}\bra{\Psi^{a.}_{i.}}e^{-\hat{T}(0)}A\ket{\Psi_{cc}} + \bra{\Lambda}\lbrack A,T^{(1)}\rbrack\ket{\Psi_{cc}}
\end{equation}
In more general terms, the response function can be written as:
\begin{equation}
{\langle\langle A;B\rangle\rangle}_{\omega_1} = \sum_{i.a.} \bra{\Lambda}\lbrack A,\{{a}^\dagger_a.a_i.\}\rbrack\ket{ \Psi_{cc}}{X^{a.}_{i.}}(B,\omega_1) + \sum_{i.a.}\left\{\bra{\Lambda}\lbrack B,\{{a}^\dagger_a.a_i.\}\rbrack\ket{\Psi_{cc}} + \sum_{k.c.}F^{k.c.}_{i.a.}{X^{c.}_{k.}}(B,\omega_1)\right\}{X^{a.}_{i.}}(A,-\omega_1).
\end{equation}
where \centerline{${X^{a.}_{i.}}(B,\omega_1) = \sum_{j.b.}\left\{ {( -A + \omega_1I)}^{-1}\right\}^{j.b.}_{i.a.}\;B^{b.}_{j.},$}\\and \centerline{ $B^{b.}_{j.} = \bra{\Psi^{a.}_{i.}}e^{-\hat{T}(0)}B\ket{\Psi_{cc}}$ .}\\\\
Thus for calculating the response functions, we need to solve two sets of
linear equations for obtaining ${X^{a.}_{i.}}(B,\omega_1)$ and
${X^{a.}_{i.}}(A,-\omega_1)$.
%\newpage
%\paragraph{Optical rotation calculations}~\\
%Rosenfeld,\cite{Rosenfeld29} using semi-classical electrodynamic theory, showed
%that the induced electric dipole moment can be written as: \begin{equation}
%\langle\vec{\mu}\rangle = \alpha\vec{E} +
%\frac{1}{\omega}\textbf{G}^\prime\frac{\partial\vec{B}}{\partial t}
%\end{equation}
%where $\vec{E} $ and $\vec{B}$ are electric and magnetic field vectors
%respectively, $\alpha$ is the electric dipole polarizability tensor while
%$\textbf{G}^\prime$ tensor is the key quantity for calculating optical
%rotation.  \begin{equation}
%\textbf{G}^{\prime}_{xy}(\omega) = -\frac{2}{\hbar} Im\sum_{n \neq 0}\frac{\omega\; \bra{\psi_o}\mu_x\ket{\psi_n}\bra{\psi_n}m_y\ket{\psi_o}}{\omega^{2}_{n0}-\omega^2}
%\end{equation}
%where $\mu$ and $m$ are electric and magnetic dipole operators: $\mu = \sum_i
%r_i , m= \sum_ir_i\times p_i$ in atomic units. $\omega$ is the frequency of
%light, $\omega_{no}$ is the excitation energy for the $n^{th}$ state ($\psi_n$) and
%$Im$ means the imaginary part of the equation. The trace of this tensor is related
%to the specific rotation, usually denoted as ${\lbrack\alpha\rbrack}_\omega $ in
%deg dm$^{\text{-1}}$(g/mL)$^{\text{-1}}$. After averaging over all possible
%orientations of the molecule, the following expression is obtained\cite{Crawford06}. \begin{equation}
%{\lbrack\alpha\rbrack}_{\omega} = \frac{(72.0 \times 10^6){\hbar}^2 N_A\;\omega}{c^2{m_e}^2 M} \times \left[ \frac{1}{3}Tr(\textbf{G}^\prime)\right]
%\end{equation}
%where $\textbf{G}^\prime$ and $\omega$ are in atomic units, $N_A$ is Avogadro's number, c is the speed of light in (m/s), m$_{\text{e}}$ is the mass of electron (kg) and M is the molecular mass (amu). The $\textbf{G}^\prime$ tensor can be obtained from using the CCLR method.
Thus, we can use this linear response formalism and can calculate the $\textbf{G}^\prime$ tensor
using eq.(4) in which A is the dipole moment operator and B is the magnetic field. However, there 
can be different representations of an operator which are referred to as gauges. It is observed that 
in the truncated CCLR method if the length gauge representation of the dipole operator is used, i.e. $\mu \equiv r$ (in atomic units), the optical rotation results are origin dependent. This problem is overcome in CCLR by using the velocity
gauge representation where $\mu \equiv r\times p$. However, both these gauges are equivalent for 
exact wavefunctions. However, the velcity gauge calculated optical rotation does not decay to zero 
at the static limit, i.e. when the external field is zero, which is an unphysical result. 
To correct this problem, another representation called the modified velocity gauge (MVG)
proposed by Pedersen \cite{Pedersen04} is used which shifts the values obtained
by velocity gauge by its static limit. 

%\begin{equation}
%\textbf{G}^{\prime}(\omega) = \text{-Im}\langle\langle\mu;m\rangle\rangle \equiv \text{- Im}\langle\langle r;r\times p\rangle\rangle
%\end{equation} 
%The above results obtained using the length gauge representation ($\mu
%\equiv r$) above, however is origin dependent. This is fixed by using the velocity
%gauge representation where the dipole moment operator is written as a momentum operator ($\mu
%\equiv {\frac{p}{\omega}}$) . Length and velocity gauge representations are equivalent for exact 
%wavefunctions.
%\begin{equation}
%\text{Tr}{\langle\langle r;r\times p\rangle\rangle}_{\omega} = \frac{1}{\omega}\text{Tr}{\langle\langle p;r\times p\rangle\rangle}_{\omega} 
%\end{equation}
%However, the velocity gauge calculations does not decay to zero at the static
%limit, i.e when the external field is zero, which is an unphysical result. To
%overcome this, another representation called modified velocity gauge (MVG)
%proposed by Pedersen \cite{Pedersen04} is used which shifts the values obtained
%by velocity gauge by it's static limit value.

